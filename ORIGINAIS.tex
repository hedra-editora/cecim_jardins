\clearpage{\pagestyle{empty}\cleardoublepage}
\addcontentsline{toc}{chapter}{Les jardins et la nuit, \emph{traduction de Cláudia Vitalouca}}
\movetooddpage
\part*{Voyage a Andara oO livre invisible\\
\bigskip
\bigskip
\bigskip
\bigskip
\HUGE{Les jardins et la nuit}\\
\Large{Vicente Franz Cecim}\\
\bigskip
\bigskip
\normalsize{La liberté est une nuit noire?}}

\chapter*{}

%\addcontentsline{toc}{chapter}{Tradução para o francês, de Cláudia Vitalouca}

\selectlanguage{french}

\forceindent{}\textbf{Que contiennent les animaux?}

Qu'est-ce qui contient les animaux?\\

L'étrange oiseau qui aveugle les hommes. Ne protège pas ses yeux. La
liberté est-elle une nuit obscure?

Une fois encore, bêtes et hommes

\pagebreak

\vspace*{4cm}

Le labyrantre.

C'est Andara. Andara c'est là où Santa Maria do Grão est née, comme on
le verra maintenant dans ces jardins, en faisant s'ouvrir la forêt, puis
en la faisant reculer, et c'est par Andara que la forêt revient.

Tous ces enfants ici

Andara est-elle le lieu d'un mythe entrevu?

Peut-être, mais pas seulement cela peut-être.

De toute façon, c'est là que s'enlacent, que se nient la nature et cette
autre chose inquiétante qui a un nom. Civilisation.

Tous ces enfants écoutent

La voix. Du labiantre

\pagebreak

\vspace*{4cm}

Seras-tu aussi attentif à ce qu'Andara n'est pas?

Andara c'est le voyage hors de soi et elle devra continuer à être ainsi,
un geste sans geste, elle sera ailleurs

Cela restera en blanc. Le vertige. Elle retire la terre sous nos pieds
et pourtant nous ne la perdons pas de vue.

Les yeux qui autrefois ont lu l'histoire du Nazaréen et le livre de
l'aveugle Dias, eux aussi lisaient un livre qu'ils ne lisaient pas.

Peut-être savent-ils maintenant, non sans surprise, ou déjà
savaient-ils? Que sous les livres il y a les non-livres.

Voyage à Andara, le livre invisible.

\pagebreak

\vspace*{4cm}

Bon.

Dans le labyrinthe, on doit dire aux autres: sans un texte, il n'y pas
de temps

Et ainsi, il y a d'infinies manières de lire Andara.

Un jeu de dislocations et, parfois, de rapprochements arbitraires, selon
chacun, et en chacun, le temps d'une émotion.

Une danse pour les intuitions.

\pagebreak

\vspace*{4cm}

Dans cette danse, la trame des livres d'Andara est abolie. Et on peut
commencer le voyage par le chemin qui s'ouvrira en premier

--- O fleur Andara, où les rêves

N'ont pas de limites.

Où tout est déréglé. Plus tard, une lecture d'Andara trouvera d'autres
combinaisons dans la mémoire. Voilà un idéal: l'imagination dans la
pénombre, libérée par l'absence physique d'un texte

Les événements curieux qui alors se produiront: qui vole à côté de
Caminá? Est-ce l'oiseau Curau? Et: existe-t-il un livre appelé Les
jours aveugles, où l'on raconte l'histoire de l'aveugle Dias?

Mais il y a plus: ayant arraché les pages de tous les livres,

et ce serait lumineux si cela arrivait dans un moment de révolte,

le lecteur, avec l'aide d'un vent inattendu qui d'abord égare puis
réorganise le voyage à Andara. Je m'arrête. Les fenêtres doivent rester
toujours ouvertes pour que les vents entrent sans cesse et qu'aucun
ordre final ne s'installe. Rigoureux.

\pagebreak

\clearpage
\thispagestyle{empty}
\movetooddpage

\vspace*{4cm}

Ce voyage vers la chose humaine la nuit\\

Et j'ai ouvertement voué mon cœur à la terre grave et souffrante et,
souvent, dans la nuit sacrée, j'ai promis de l'aimer fidèlement jusqu'à
ma mort, sans crainte, avec son pesant fardeau de fatalité, et de ne
négliger aucune de ses énigmes. Hoelderlin dit

Cette nuit une fois niée,

Il y en aura une autre

Et j'ai ouvertement voué mon cœur à la terre et souvent, dans la nuit
sacrée, j'ai promis de l'aimer fidèlement et de ne négliger aucune de
ses énigmes\\

La fardeau d'une histoire avec une histoire.

Comment serait une histoire sans une seule histoire, le fardeau éclaté
en morceaux?

Le non.

Le labyrinthe.

Le non. Non à une histoire unique. Même si elle est l'histoire d'un
homme. Faire en sorte que par elle passent autant d'histoires d'autres
hommes que l'oreille en perçoit et que le vent en raconte.

Le labyrinthe. Dans celui qui s'ouvre maintenant, il y a les voix que
cet homme écoute à sa fenêtre, réelles,

elles viennent de documents anciens sur la douleur et se mêlent à la
mise en scène de la douleur imaginée faite par la voix d'Andara\\

Alors, le voici.

Le non labyrinthe.

\pagebreak

\vspace*{4cm}

Une sorte de voix murmure dans la nuit.

C'est cela l'imagination.

Elle arrive. On ne sait pas bien d'où

Voilà ce que nous avons. Un homme à sa fenêtre. On ne peut pas l'éviter.
Et des voix. Il suffit alors de se laisser aller,

ce voyage parle de la vie et ne s'arrêtera pas avant la fin.\\

Une autre fois.

Ce sera comment?

\pagebreak

\vspace*{4cm}

L'homme était là. Aveugle.

Quand l'autre est arrivé, il a dit:

--- Je m'appelle Jacinto. Je suis ici, aveugle. Ici c'est partout.

Oui. Ce genre de choses arrivent à Santa Maria.

Tu es venu pour entendre l'histoire. On t'a dit là-bas que je sais
comment tout est arrivé, alors tu es venu. Ils t'ont dit où j'habitais.
Ils t'ont mené jusqu'à cette rue et t'ont montré la porte de cette
maison. Alors entre.

L'autre homme est entré.

Il est resté à écouter.

Pendant une partie de l'après-midi, l'aveugle a parlé. Et l'autre
écoutait.

Lui, l'aveugle, a dit:

--- Il n'y a pas si longtemps.

Et je me souviens de tout. Je n'oublierai jamais.

Ecoute.

Ça s'est passé comme je te le dis maintenant, avec ces mots. Tu peux ne
pas croire que j'ai été averti avant les autres, mais c'est vrai. Il n'y
pas de quoi le jurer. Il y a seulement cette voix telle qu'elle est, et
ne prête pas l'oreille au vent de la nuit., de cette nuit sans la
moindre lueur pour éclairer les énigmes, cela se tordra avec des lames
partout, ne prête l'oreille qu'à cette voix et laisse le vent emporter,
aussi, la part qui lui revient dans ce que tu vas entendre. D'autres ont
aussi besoin de savoir comment tout cela est arrivé.\\

Ecoute.

L'oiseau était tout rouge. Il était là, arrêté. Il semblait malade.

C'est ainsi que tout a commencé.

Je lui ai donné le nom que j'ai voulu, puisque les choses qu'on connaît
ont toutes un nom commun, mais les autres, celles qui viennent on ne
sait d'où, on ne le sait jamais, on peut les nommer comme on veut, il
suffit d'ouvrir la bouche et de retenir le premier mot qui sort, il n'y
a que comme cela qu'on peut trouver le nom caché d'une chose cachée. Et
celui-ci n'était pas un oiseau comme les autres, ça je l'ai vu tout de
suite.\\

--- Curau.

C'est le premier mot qui est sorti de ma bouche. Cette bouche qui de
cette façon a pris vie. Curau. Et ce nom~lui est resté depuis ce jour de
rumeurs et de changement, le mien, où j'ai ouvert la porte et qu'il
était là. Il était vraiment malade et ne s'est pas enfui quand je me
suis approché. Il a voulu entrer dans ma maison. J'ai pris soin de lui.
Très vite il s'est rétabli. Il était fort. Grand. L'oiseau.

Il faut que je te dise: auparavant, j'avais les yeux fermés, et j'ai
tout vu.

C'était avant de trouver l'oiseau. J'ai fait un rêve.

Quand je me suis réveillé, je n'ai pas compris ce que j'avais vu sans
mes yeux. Dans mon rêve.

Maintenant je sais que cela avait été un présage. J'avais vu les jours
de peur qui plus tard allaient s'installer parmi nous

Quand je me suis réveillé, j'ai revu ces ailes immenses,

des yeux de feu qui m'épiaient du fond de la nuit,

et dans le rêve que j'avais fait il y avait une nuit dans laquelle les
hommes couraient, fuyant avec les femmes par les rues vides, des rues où
les pierres n'avaient pas de poids,

fuyant comme ils allaient fuir aussi dans les rues bien réelles de cette
ville. En criant. En cherchant à se cacher.

En me réveillant de ce rêve, je ne pouvais pas encore comprendre, vivant
comme je vivais dans l'illusion que seul ce que l'on voit les yeux
ouverts est réel, et miroir trompeur tout ce qui nous apparaît quand on
ferme les yeux. Non je ne pouvais pas. Il y avait cet oiseau qui volait
au-dessus de la ville, il plongeait sur une rue et on entendait des
cris. Les cris.

Ce fut le début. Et je l'ai vu.

C'est ainsi que la vie m'avertit que le Curau allait arriver.

Et ensuite, vois-tu, elle, la vie, est venue poser le Curau exactement
devant la porte de cette maison

Plus tard, j'ai compris

Nous avons des yeux, comme j'en ai eu, pour regarder partout. Un miroir
nous aveugle en dépit du soleil.

Est-il encore là, tout là-haut?

Existe-t-il encore?

La nuit hésite et nous rejette avec ces yeux qui ne nous servent à rien

Tout s'est passé comme je l'ai dit.

Avant de voler parmi les hommes, le Curau a volé en Moi, entre les yeux
que j'avais, fermés, et la vie là dehors. Il a d'abord volé dans un ciel
humain.

Et alors j'ai trouvé l'oiseau et je l'ai emmené chez moi.\\

La peur n'est arrivée que plus tard chez les autres.

C'est plus tard qu'ils ont commencé à entendre ses cris. Alors, le Curau
volait déjà au-dessus de la ville.

Il volait et ils criaient, le Curau. Curau. Il est revenu. Et les enfant
aussi criaient.

Pourtant ils ne devaient pas avoir peur.

La peur ne vient que chez ceux qui ont déjà de vieilles raisons d'avoir
peur,

Ceux-là avaient déjà la peur en eux et ils se sont mis alors à avoir
peur du Curau. Ceux-là ont peur de tout. Et l'oiseau était la seule
chose maintenant visible qui faisait surgir leur peur, cette peur qui
allait par les rues au pas qui nous mènera, qui mène toujours, à une
terre non-sacrée. Ils allaient ainsi dans la contrée d'un dieu sans
Visage, et dans son œil gauche ils avaient planté une Epine.\\

Vois, sauf les enfants qui ont des peurs réelles

Non. Je veux dire cela d'une autre façon. Comme je l'ai compris

Quand un enfant criait le Curau, le Curau, il ne faisait que ce que les
autres faisaient, il imitait

Si les adultes n'avaient pas peur, les enfants eux non plus n'auraient
pas peur de l'oiseau

Ils resteraient dans leurs hamacs, calmes. Et le soir ils n'auraient pas
d'appréhension pour entreprendre sans effroi leurs vols enfantins

Un Curau n'attaque jamais, jamais il ne fait du mal à un enfant. Jamais
on n'a parlé d'une attaque d'oiseau contre un enfant. Il perce seulement
les yeux des adultes. Tu n'as jamais entendu dire qu'il avait aveuglé un
enfant.\\

Les autres cependant,

ceux-là ne dormaient plus depuis l'arrivée de l'oiseau. Les adultes. Ils
tremblaient la nuit. Ils avaient des frissons d'effroi. Une chose était
certaine: à tout instant l'oiseau pouvait entrer par leurs fenêtres et
les aveugler, dans ces maisons où ensuite ils erreraient, sans but, en
se cognant contre les objets\\

Et les jours passaient\\

Il y avait la peur. Elle était à l'intérieur. Et tout autour.

J'ai entendu des choses s'écrouler.\\

Eux ne comprenaient pas.

Un Curau ne fait pas de mal.

Il fallait même demander sa venue.

--- Il faut demander qu'il vienne, c'est ce que je disais à tous.

Je leur disais:

--- Tout le monde doit demander au Curau de venir.

Tout le~monde doit le demander chaque nuit et dire comme dans une
prière:

Viens Curau, et aveugle-moi

Délivre moi des choses immuables,

Mes yeux ne veulent plus voir ces jours monotones

Curau, fais-moi tomber

dans la nuit que je suis moi-même, je le sais,

pour que tout change

pour que je retrouve le goût de la vie

Curau,

c'est ce que tous devraient dire, les yeux fermés,

moi je ne vois plus rien

Je ne vois plus les autres hommes, il y a un masque sur chaque visage

Fais aussi que ces autres se perdent dans ta nuit, perdent leurs yeux
avec moi, fais-leur du bien

Aveugle la femme qui dort à mes côtés, aveugle ces hommes autour de moi,

eux aussi ne voient qu'un masque sur mon visage et ne peuvent voir,
comme je ne peux voir en eux, sous le masque, le visage que j'avais Et
que j'ai encore. Il est caché. Souterrain. Un espace sans lumière.

Viens, et je te le demande, ne vois pas le masque sur le visage d'un
enfant. Je te le demande. Parce qu'en eux vivent les jours

Laisse pour cela leurs yeux en paix

Et ne perce que les miens, Curau, en entrant par cette fenêtre avec ta
lumière noire.

Moi je ne sais plus voir\\

Voilà ce que vous tous devriez demander

C'est ce que je leur disais.

Ils devaient se réunir dans les églises et sur les places pour demander.
Et aussi demander seuls, comme le font les saints à voix basse

Demandez la venue de l'oiseau et attendez qu'il vienne, leur disais-je.

Et qu'il vous exauce vite.\\

Regarde. Moi.

Tu es entrain d'écouter un homme dont les yeux ont déjà été percés par
le Curau, dit l'aveugle à l'homme.

Et l'autre écoutait.

Maintenant ce que je sais c'est ce que je sais. Rien.

L'autre écoutait et l'aveugle a dit la première chose que l'oiseau a
fait c'est de m'aveugler. En finir avec les yeux que j'avais. Détruits.

L'homme en face de l'aveugle a regardé par la fenêtre.

Il a eu du mal à retrouver ses forces, continua l'aveugle, après avoir
mangé ce que je lui avais donné, il a bondi sur moi et alors est arrivée
cette nuit dans laquelle maintenant je suis plongée, ici. Ici c'est
partout. Cette nuit, j'attends ce qui va arriver, les jours sans nom, je
le sais

J'attends et je sais

J'attends et j'écoute. Les voix viennent dans le vent

J'attends et je marche en tâtant les choses. Et je comprends.

Je passe cette main sur le visage de quelqu'un et je comprends qu'il est
triste et d'où vient sa tristesse, je passe cette même main sur le
visage d'un autre et je comprends sa peur et d'où vient la Peur et où va
la Peur. Et de quoi est faite la Peur.

Alors je dis d'une voix lente pour ne pas l'effrayer

Prends patience attends un jour l'oiseau va revenir\\

L'homme devant l'aveugle a eu envie de regarder à nouveau par la
fenêtre.

Il y est allé, a écouté ce que l'aveugle disait.

Il a voulu savoir la suite, ce qui était arrivé ensuite.

Un jour il s'en alla, disait l'aveugle.

On ne sait pas où. Il a dû retourner d'où il était venu.

Il sera maintenant dans un nid inamical, dans un lieu caché dans une
quelconque partie de la vie, loin ou près de nous. On ne sait pas

Maintenant, à cette fenêtre, j'attends le Jour.

Le Jour où il reviendra, dit l'aveugle. Ce sera par un après-midi comme
celui-ci peut-être. Il écartera les nuages. Quand l'oiseau reviendra,
disais-je. Et alors. Les cris reprendront. Les fuites. Moi

j'attends ici.\\

L'homme a entendu ce que l'aveugle lui disait. Reste, tu viens de loin,
et attends son retour. Le retour de l'oiseau.

L'aveugle demandait à toucher ses yeux. Il l'a laissé faire.

Viens ici, disait l'aveugle.

Laisse-moi toucher tes yeux.

Les yeux.

Ceux-là aussi. Les tiens.

Tous pareils.

Les tiens sont comme ceux que j'avais. Ils regardent hors de la vie.

Délivre-toi d'eux aussi.

Ah, mais tu as peur. Je t'ai touché. Je le sais.

Tu en verras plus d'un courir par les rues quand l'oiseau va revenir.
Cachant leurs yeux, cherchant un lieu pour se cacher, criant le Curau,
le Curau. Il arrive. Comme ils criaient. Tellement. Les cris. Et plus
tard aussi tu ne fermeras même pas les yeux la nuit, par peur de n'avoir
plus d'yeux à ouvrir le matin. De ne plus rien voir.\\

Repoussé par l'aveugle.

L'homme s'est écarté.

L'aveugle lui disait pars maintenant. J'ai tout dit. Maintenant je veux
rester seul. Rester assis à côté de la fenêtre. Aveugle.

\pagebreak

\vspace*{4cm}

J'écoute.

Je suis ici. Aveugle.

\pagebreak

\vspace*{4cm}

Une fois l'homme parti, l'aveugle était resté seul à nouveau.

\pagebreak

\vspace*{4cm}

Et cette voix qui dit dans le vent\\

--- Viens Curau. Viens emmener les hommes vers tes jardins

\pagebreak

\vspace*{4cm}

J'écoute.

Je suis ici. Aveugle.

Ici c'est partout.\\

C'est la voix de l'aveugle qui parle. Mais il y en aura d'autres, c'est
dit\\

J'attends le retour de l'oiseau. Et j'écoute les voix, dit-il. Les voix
de la terre viennent de loin. Pour entendre les voix de la terre il
suffit de se laisser aller, dit l'aveugle près de sa fenêtre.

Il dit, ceci est un homme. Un insecte me regardera sans comprendre.

Il y a toutes ces rumeurs\\

Maintenant l'aveugle écoute.

Et il dit: C'est du fond de la tête que me viennent ces histoires. Tout
vient dans le vent aussi

pour ceux qui insisteront à avancer dans dette nuit, et elle, la vie, à
son tour avancera aussi. Parfois elle crie. D'autre fois elle murmure

\pagebreak

\vspace*{4cm}

Maintenant

c'est l'heure où les ombres s'approchent des choses.

D'ici peu, personne ne pourra plus voir ses pieds, qui buteront sur les
objets, hésiteront dans les maisons impuissants à savoir où diriger le
corps. Les bras aussi s'effaceront et on pourra les étendre en avant
sans voir ce que les mains iront toucher, et tous ainsi, mutilés,
découvriront avec surprise, bien que cela arrive tous les soirs, qu'ils
disparaissent. L'obscurité envahira aussi tous les yeux jusqu'à ce qu'il
ne reste plus rien à donner à la nuit. Mais ce n'est pas le Curau qui
revient, non. C'est seulement le Curau de la vie qui fait disparaître la
vie pour un certain temps à nos yeux

La nuit. Maintenant elle arrive.\\

Un cri. C'est un oiseau.

Il dit à Jacinto qu'à Andara tout cesse d'être humain. C'est d'Andara
que vient ce cri qui arrive à sa fenêtre, à Santa Maria.

Santa Maria do Grão le soir.

Maintenant la peur, la Peur va sortir d'ici et courir urbi et orbi
emportée par le vent.\\

A sa fenêtre Jacinto attend

Un bruit de chute arrive dans le vent.

Cela commence, dit Jacinto à sa fenêtre. Et il attend.

Il y a un homme à sa fenêtre qui écoute

Et la vie lui parle

Arrivée avec le vent, elle, la vie, pour que moi qui porte le fardeau de
toutes ces histoire je l'écoute aussi\\

Plus tard, il entendra un aie

C'est le malheur

Il est en train de s'installer.

Alors Jacinto sent que tout recommence et que cette nuit va être une
nuit d'insomnie, et que seul lui, l'homme à sa fenêtre, va rester calme,
muet, tandis que tout parle et que les autres parlent

Il arrive dans le vent:

--- Je te dis qu'il a arraché les yeux avec ses mains. Il est tombé, il a
roulé sur le sol et a appelé quelqu'un. La femme est arrivée. Et alors,
pour qu'il ne souffre plus, elle a utilisé la corde pour le tuer.

Une voix dit cela.

Et à sa fenêtre l'aveugle sait que certains ne résisteront pas jusqu'au
lever du jour.

Demain, quand la lumière reviendra, ce sera un jour de plus pour les
enterrements. Mais un jour les Ailes vont revenir et cela n'arrivera
plus, dit l'homme à sa fenêtre.

Il attend.

Et il sait qu'à mesure que la nuit avance,

la force du vent augmentera jusqu'à arracher les fenêtres, tirer les
hommes des hamacs, frapper à la porte comme pour entrer de force tandis
que de l'autre côté du bois s'amoncellent les choses, les tables, les
morts, les chaises, tout pour résister au vent. Et le vent ira aussi au
bord du fleuve et coulera les barques amarrées au ponton et renversera
les chandeliers et éteindra les bougies, et ne ménagera aucun refuge de
lumière où l'on puisse courir, poursuivis, tous, par une chose sans nom
qui vient du fond de chacun de nous et se trouve dehors aussi, et dans
les rêves de ceux qui dorment et tentent ainsi de s'échapper en feignant
de ne rien savoir, les endormis, qui ne sont pas vivants, et il entre
par leurs oreilles, et même au fond des rêves personne n'est sauvé car
maintenant le vent va faire en sorte qu'ils rêvent qu'il bouleverse tout
en eux, dans leurs têtes, embroussaillant leurs cheveux et, pâles, ils
se réveilleront en voulant fuir ici, dehors, vers la vie, mais c'est
ici, en elle, qu'un véritable enfer les attend et le vent agrippe les
femmes et veut soulever leurs jupes et arracher les enfants de leurs
ventres dès qu'ils sortent leur tête et regardent dehors. Ceux qui
tenteraient de naître cette nuit. Epient la vie qui s'élance, s'élance
mais vers quoi,

Se demande Jacinto à sa fenêtre.

Il a déjà vécu des nuits comme celle-là.

La vie pourrait être autre chose, dit l'aveugle. Mais seulement pour
ceux qui sauront attendre que le jour se lève.

Il se lèvera bientôt, pense Jacicnto à sa fenêtre.

Et il attend.\\

Cette nuit, d'autres voix viendront à nouveau dans le vent\\

Il écoute.

Elle lui diront:

Que maintenant les eaux du fleuve sont devenues folles, elles coulent en
sens inverse. Elles montent par là, signalent les voix

Et elles disent, Notre Dame illumine de ta lame de lumière cette nuit

Et elles disent aussi, O merveille, O merveille. Les hallucinations
arrivent au port, elles apportent un espoir, venez tous

Voilà ce que disent les voix à l'homme à sa fenêtre.

Et elles disent plus encore, elles disent, Un fils de dieu a été enlevé
par le vent et jeté sur le pavé de l'église quand il disait ses prières.
La petite église se lézarde, elle ne résistera pas à cette nuit

Je ne supporte plus d'attendre le lever du jour,

Disent les voix.

Et l'homme à sa fenêtre écoute. Il attend.

Peut-être l'une d'elles demande-t-elle maintenant qu'on sacrifie un
innocent, se dit Jacinto.

Elle dira, C'est pour nous sauver.

Ils boiront son sang s'il se répand.

Et plus tard ils allumeront des feux dans la nuit. Ils feront une fête.
Et ensuite ils tenteront de s'endormir, écrasés par l'alcool.\\

Le jour va arriver.\\

Cette nuit va finir, murmure Jacinto.

Alors arrivent dans le vent des voix plus anciennes,

parlent d'un autre temps de tortures.

Et le vent dit: Il y a des gens, condamnés à être tirés par quatre
chevaux dans quatre directions.

Oui, dit Jacinto.

Et il pense, il va bientôt faire jour.

Et avec le vent arrive, d'autres pays, cette histoire: Là-bas, dans un
pays inconnu, une femme en a tué une autre. C'est pour cela que
maintenant elle va mourir, maintenant c'est son tour. Elle va mourir en
face de la chaise où était assise l'autre femme quand elle l'a tuée.

Jacinto peut voir cela de sa fenêtre

C'est la vengeance. Dit le vent.

Il dit et menée sur le lieu où elle a tué l'autre, sa main sera coupée
et jetée au feu pour qu'elle voie. C'est avec la main droite qu'elle a
tué et c'est précisément elle qu'elle voit se transformer en cendres

Pour qu'elle voie, répète l'aveugle à la fenêtre.

Elle mourra du même couteau, dit le vent.

L'homme à sa fenêtre répète c'est la vengeance

Le même couteau, dit le vent.

Et il s'arrête.\\

Les voix qui viennent de Santa Maria do Grão dans le vent s'arrêtent
aussi. Et plus aucune voix n'arrive jusqu'à l'homme, pas même d'Andara.

A sa fenêtre il n'entend plus rien maintenant

Ensuite, le vent revient. Et un oiseau tout noir se fracasse sur le
visage de l'homme. Il sait que celui qui a jeté l'oiseau c'est le vent.
Et cela lui importe peu.

--- Ils ont tué Mariana, ils on tué Mariana, crie quelqu'un sous la
fenêtre.

A sa fenêtre l'homme ne croit pas cette voix.

Il sait que c'est un oiseau que le vent a jeté sur son visage. Et que
Mariana n'est pas noire. Il sait aussi que de nombreuses nuits viendront
encore pour qu'il ressente cette peur d'être irrémédiablement vivant. Et
il attend que le jour arrive\\

Le matin arrive maintenant

Je le sens venir.

Il vient de ce côté, entre par la fenêtre, va éclairer d'abord les mains
des hommes pour qu'ils puissent trouver un verre d'eau et éloigner cette
nuit. Ensuite, il éclairera leurs pieds pour qu'ils vivent un jour de
plus avant que les ombres ne reviennent toucher les choses blanches

Je suis ici.

Ici c'est partout, dit l'aveugle à sa fenêtre.\\

Le jour se lève.\\

Et si l'enfance venait jusqu'à lui maintenant?

L'enfance arrive alors

Et une voix vient dire dans le vent:

--- La constellation du chien aboie à nouveau

\pagebreak

\vspace*{4cm}

--- Fin pour l'enfance,

crie, muet, l'aveugle à côté de la fenêtre.

Il est là. Aveugle.

Et il attend le retour de son oiseau. Le premier mot qui est sorti de sa
bouche. Curau.

--- Nous devons veiller nos morts, dit Jacinto. Avant j'étais avec eux.
Maintenant je ne veux plus quitter ce lieu, avancer dans l'obscur. Les
morts sont des poissons partis pour d'autres eaux, on ne sait où\\

Et dans le vent une autre voix vient lui dire:\\

cette voix venait de l'autre côté de la pièce.

C'était comme un travail d'insecte.

Ensuite elle s'est arrêtée de parler

Les uns grands les autres petits approchaient leur visage de l'homme
couché pour un ultime baiser. Il a entendu des voix d'enfants. Là dehors
un oiseau de la nuit a chanté. La brise entre maintenant par la fenêtre
et frôle ses cheveux qui semblent retrouver un restant de vie aux yeux
de ceux qui sont autour de lui

Dans la pièce quelqu'un murmure

--- Alors passons la frontière, et il n'a pas écouté la fin de ce qu'il
disait

Pendant toute la nuit nous avons veillé dans la maison

Et ensuite le mort n'était plus là.

Mais ils l'embrassaient encore sans savoir qu'il était parti avec les
lèvres froides de l'aube. Les parents

--- Jamais nous n'avons été aussi doux qu'en cet instant, répétait la voix
de l'autre côté de la pièce. Dans mon pays, il y a des pécheurs qui
amarrent les morts par les pieds pour qu'ils ne se lèvent pas et ne
retournent pas à la mer, dit cette voix

\pagebreak

\vspace*{4cm}

Je suis ici.

C'est l'aveugle.

Ceci est un homme, dit-il. Et ceci veut aussi comprendre

\pagebreak

\vspace*{4cm}

--- La constellation du chien aboie\\

--- Nous n'avons jamais été aussi doux qu'en cet instant\\

Les voix errent encore dans l'air de ce matin naissant

Tandis que dans leurs maisons les hommes dorment, soignant dans leurs
rêves les blessures de la nuit passée\\

Et une voix arrive dans le vent jusqu'à Jacinto à sa fenêtre pour dire:

--- Andara. Là-bas, les rues sont toujours vides et Cela, la forêt,
avance, s'approche de plus en plus de nous\\

Jacinto a faim.

Ceci aussi est un homme, dit-il.\\

Andara c'est l'Afrique que nous avons en nous

\pagebreak

\vspace*{4cm}

Et si une autre voix arrivait encore?

Et cette voix arrive. Et elle dit:

--- La vieille. Elle mastique lentement. Elle aussi est un animal la nuit,

comme on m'a dit qu'était cet oncle Irido qui est allé à Andara et n'en
est jamais revenu. Je ne vois pas ses yeux, elle n'a pas d'yeux pour se
voir

Jacinto écoute et pense, Agora est une de mes sœurs qui me parle\\	

Et la voix:

--- Maintenant elle marche dans le jardin. La vieille. Un insecte la voit.
Ils se regardent. Il n'y a rien à comprendre

\pagebreak

\vspace*{4cm}

Des vents qui parlent, des vents qui parlent\\

Et cette voix qui maintenant vient raconter à l'aveugle à sa fenêtre

ceci:

--- Un homme est là et regarde le visage d'une femme qui dort et
l'endormie disparaît dans sa nuit. Mais sur son visage les choses
qu'elle a vécues aujourd'hui vivent encore,

et sur ce masque qui fut son visage les choses disent

j'ai vécu encore un jour avec ses douleurs et la joie qui revient à
chacun sur cette terre que les hommes appellent ma maison.

Ensuite, sur le visage de l'endormie se dissipe jusqu'au masque des
choses. Et alors il n'y a plus rien à regarder. Le vide.

--- Où a-t-elle pu aller, demande l'homme en regardant ce vide. Qui
commence à se changer en paysage,

car d'autres choses commencent maintenant à prendre forme sur le visage
de l'endormie.

La femme s'agite. Elle écarte quelque chose. Non. Dit-elle sans voix.
Non.

Et l'homme sait qu'elle rêve à quelque chose qu'elle ne veut pas,
qu'elle refuse.

Et ce qu'elle ne veut pas c'est ce paysage qui surgit, où l'homme voit
un enfant et l'enfant voit les figures qu'il aimait retrouver sur les
murs, les taches,

Pour y voir des animaux, des armes, un peu de sang coulant au début,
puis plus de sang encore, coulant plus fort. C'est l'homme et sa faim,
il veut un animal pour sa table, disait l'enfant en regardant le mur. La
tache. Il suffisait de laisser les yeux aller et se laisser aller, avec
eux, pour voir

des hommes derrière, à sa poursuite, et l'animal fuyant sans savoir où,
pris dans la tache, sur le mur. Le piège

\pagebreak

\vspace*{4cm}

--- La faim arrive à l'heure initiale de la vie, quand vient le matin et
que les yeux s'ouvrent\\

dit une voix et le matin arrive encore.

Ce matin, le temps stagne

Et dans leurs maisons les hommes dorment\\

Alors, plus rien d'autre n'arrivera en dehors de ces voix qui racontent
des histoires à l'aveugle à sa fenêtre?

Nous ne le savons pas encore. Inquiets,

les pervers de la continuité.

\pagebreak

\vspace*{4cm}

--- En voyant le piège l'homme est allé dans sa mémoire,

raconte la voix qui parle de l'endormie.

Mais maintenant il s'agite.

C'est qu'un autre homme s'incline sur le visage de l'endormie.

--- Qui cela peut-il être, se demande l'homme. Et comme tout est sombre et
qu'il ne peut voir le visage auquel elle rêve, il pense que l'autre
pourrait bien être lui, qui sait

Une nuit arrive maintenant sur le lieu où la femme rêve, une fin
d'après-midi, et là elle ne se refuse plus, elle ne repousse pas
l'ombre.

--- Là où elle est, où ils sont, personne n'est plus, dit l'homme à côté
du lit.

Il regarde encore et voit la femme et l'autre, lui peut-être, qui se
dissimulent entre les arbres dans un jardin. Elle appuie sa tête contre
une épaule sombre, c'est la dernière chose qu'il voit

\pagebreak

\vspace*{4cm}

Une nouvelle voix arrive, elle entre par la fenêtre. L'aveugle écoute

Cela ne finit pas. Ne finit pas?

Une fois de plus

L'enfance n'a pas de fin\\

--- A Santa Maria do Grão, raconte la voix

Celui qui passe par cette rue ne sait pas pourquoi il voit une cage en
regardant la maison. Celui qui passe regarde la maison. Pour ne pas se
tourmenter avec ça, il veut trouver une explication et s'arrête, regarde
bien et ensuite s'écarte et se dit que cela n'était qu'une illusion, que
cela vient des grilles de la maison

Et l'impression part avec lui, elle accompagne des jours durant celui
qui a vu la maison

Il n'arrive pas vraiment à s'en distraire, le voilà revenu devant cette
maison. Il cherche dans sa mémoire où il a bien pu voir cette maison, se
demande celui qui est passé.

C'est ainsi.\\

L'enfance n'a-t-elle pas de fin?\\

Dans la maison demeure une femme. Il n'y a plus d'homme ici, mais il est
resté une enfant, une fille, et elle mange tout ce qui tombe d'en haut.

La mère criaille par toute la maison, elle met de l'ordre, balaie le
sol, elle a une plume noire qui sort de l'ouverture de sa robe,
derrière. C'est avec elle que maintenant la femme a terminé de balayer
le sol.

En plus, c'est une femme comme une autre. Comme les autres on peut la
voir de partout.

C'est ainsi qu'un jour l'homme l'a vue

L'homme riait, il se tenait sous la femme et sentait cette plume sur son
corps maintenant mort.

Auparavant, la fille aussi riait. Mais la femme sait utiliser la plume
avec violence et la petite n'a plus jamais ri.

Avec la plume la femme s'évente les jours de chaleur, avec elle aussi
elle ranime le feu qui meurt sous une casserole. La femme utilise la
plume pour faire beaucoup de choses. Et maintenant elle est tourne le
dos, feignant de ne rien voir mais tout à coup elle se retourne et il y
a une mouche de moins dans ce monde

Avec la plume, la femme pourchasse aussi les insectes. Et elle sait où
les trouver dans les trous et les fentes et il n'y a aucune petite
fissure où puissent fuir

Les insectes.

Belle, la plume brille.

Le soleil l'éclaire. Noire

Le soir cependant, on ne sait pas où va la femme. Tous les soirs elle
sort. Le jour se lève déjà quand elle rentre, rapporte de quoi manger,
la fille mange ce qui lui tombe du haut

C'est un oiseau. Et elle chasse au clair de lune.

C'est cela, se dit la petite, seule, à la maison

pendant que sa mère vole par les rues la nuit et qu'elle elle reste là,
elle ferme les yeux et ainsi elle voit la femme voler, ensuite vient un
sommeil calme et elle s'endort en disant, j'ai cette mère, j'ai cette
mère. Je dois me réveiller demain

Cependant, le visage de la femme a commencé à annoncer la mort.

--- Elle est fatiguée d'être vivante, dit la fillette dans sa chambre à
son miroir. Elle est fatiguée d'être vivante, a dit la fillette un jour
en regardant le visage de sa mère endormie, un après-midi

Elle a vu que la plume se tordait sans lumière.

Et elle a eu peur. Si elle part, alla-t-elle dire une autre fois à son
miroir.

La peur était arrivée.

La peur.

Mais maintenant la fillette n'a plus peur.

La peur s'en est allée.

Elle est sortie de cette maison.

C'était à l'heure du dîner, aujourd'hui

Quand elle s'est assise à table, il y avait quelque chose sous elle,
entre elle et la chaise.

Elle s'est levée, est allée regarder dans le miroir et s'est
contorsionnée pour pouvoir voir derrière. C'était une ébauche. En elle
aussi elle apparaissait

Elle est retournée à table. Et elle a mangé avec un appétit qui revenait
aussi. Elle regardait sa mère assise comme toujours de côté et s'est
assise aussi de côté. L'héritière. Et elle a dit tout bas moi et elle
riait.

Maintenant, quelqu'un passe devant la maison, celui qui passe regarde la
maison

\pagebreak

\vspace*{4cm}

--- Si vous voulez, parlez davantage de la douleur\\

C'est l'aveugle qui a dit cela. Il parle avec les voix qu'il a écoutées.
Les voix.

Elles n'omettent jamais de venir lui raconter leurs histoires, dans le
vent,

les jours ne passaient pas\\

Les voix. Ces voix, ces histoires

Elles vont venir encore longtemps pour lui dire que la vie, là dehors,
est encore un lieu de rumeurs et qu'un Non tient tout sous sa coupe.

En lui, les hommes se cherchent sans se voir, dit l'aveugle. Mais un
jour l'oiseau va revenir

Il rit.

Dans la pièce où il est il y a un miroir, mais le miroir ne reflète pas
de rire

\clearpage
\thispagestyle{empty}
\movetooddpage

\vspace*{4cm}

Ah tu as frappé à ma porte. Tu es revenu.

C'est l'homme qui entre maintenant.

Il est revenu.

Il venu voir comment je passe.

Je ne passe pas. Les jours ne passent pas. Rien de tout cela ne change
tant que le Curau ne revient pas, dit l'aveugle à l'homme.

J'ai cette fenêtre et le temps s'y est arrêté. J'y attends l'oiseau. Tu
te rappelles, je t'ai tout raconté, comment cela a commencé,

Dit l'aveugle à l'homme et l'homme était de retour.

Il était entré comme la première fois.

Et maintenant il était à nouveau en face de l'aveugle.\\

--- Oui, j'ai entendu le vent, lui dit Jacinto. Les histoires qu'il vient
me raconter. J'écoute toujours le vent. Non, tu ne pourras pas écouter
aussi ces histoires. Non. Toi tu as tes yeux. Tu veux les garder.

Si tu veux entendre une autre voix. La mienne.

\pagebreak

\vspace*{4cm}

Maintenant le vent s'est arrêté. Et l'aveugle a près de lui cet homme
qui est revenu. Et il attend le retour de l'oiseau. Il veut entendre ses
ailes dans le ciel. Les Ailes.\\

--- Quand le vent s'arrête, je me souviens d'autres histoires, a-t-il dit.
Tu veux écouter l'une d'elles?

Oui. Je t'ai déjà raconté l'histoire du Curau, maintenant tu veux en
entendre plus, toujours plus? Laisse-moi te dire alors, pendant que
j'attends. Et ne pense pas que l'histoire du Curau n'est qu'une
histoire. Je suis ici, seul. Ici c'est partout.\\

L'homme écoute.

--- Vois, dit l'aveugle, le temps c'est ça et alors je ne me rappelle déjà
plus comme elle a fini.

Maintenant c'est l'histoire de Sumiro qu'il va raconter à l'homme. Ne
pense pas que l'histoire du Curau n'est qu'une histoire, répète-t-il.

Peut-être invente-t-il maintenant une fin à cette histoire, il dit. Et
l'homme écoute. Peut-être se rappelle-t-il à mesure qu'il l'invente. On
ne sait jamais. Il y a la mémoire, cette chose la nuit. Je ne me
souviens pas non plus des noms. De toute manière, en elle,
l'Imagination, il y a des choses qui grossissent, des feux énormes, et
il y a ce qui s'éteint. Ou revient changé, au retour, quand on veut se
rappeler. Dans la Mémoire.

Je vais par ces chemins. J'imagine. Je me rappelle.

Ils sont deux. Ils se mêlent.

J'ai dit que je ne me rappelais pas les noms?

Je ne mens pas. Il n'y a que son nom que je n'ai pas oublié. Sumiro.
C'était Sumiro. L'inoubliable.\\

C'était une fin d'après-midi.

Ils sont venus et l'ont emporté, attaché.\\

A cette époque, j'avais encore mes yeux pour voir, dit l'aveugle.
C'était avant le Curau et avec eux je ne voyais rien, tu comprends.

Regarde, la terre où ils l'attachaient écrivait déjà sur son corps cette
histoire que maintenant je te raconte

en la gravant aussi dans tes oreilles,

et cela étant arrivé tandis qu'ils l'emportaient, c'était une
anticipation de toute la douleur qui allait l'atteindre plus tard,

comme elle viendra aussi plus tard pour toi qui maintenant écoute
l'histoire et son lot de douleur

Eux cependant n'emportaient pas l'homme attaché parce qu'il résistait,
non. C'est qu'ils voulaient l'emporter comme ça. Vois.

A l'endroit qu'ils avaient choisi pour faire ce qu'ils ont fait, l'un
d'eux les attendaient, en surveillant, une paire d'yeux vifs, les armes
prêtes à servir contre ceux qui voudraient les empêcher de faire ce
qu'ils allaient faire. Vois maintenant ceci, c'était un Lieu hors de la
terre, il était au-dessus de l'enfer. C'est là que tout est arrivé. A
cette heure, les oiseaux ont commencé à voler bas, annonçant la mort
arrive la mort arrive, ils volaient et les gens entendaient le
froissement de leurs ailes.

--- Prends un café, dit l'aveugle à l'homme.

Ah, ce froid, dit l'homme.

C'est la nuit, dit l'aveugle.

Non. Je ne veux pas. Cette nuit, plus tard, je veux rêver après t'avoir
tout raconté. Si les voix le permettent

Ecoute.

Alors, je disais: ils l'attachèrent comme s'il n'était pas un homme.
Sumiro. Mais c'était un homme.\\

Bien qu'il vécût toujours de cette façon, et que toujours il eût les
yeux baissés cherchant sur le sol on ne savait pas ce qu'il cherchait.
On ne l'avait jamais vu lever les yeux. Eux, les yeux de cet homme,
s'accordaient bien avec la terre. Seulement avec elle. Jours après
jours, il vivait ainsi. Il vivait à l'intérieur. Certains disaient qu'il
n'était plus un homme et qu'il n'était plus parmi nous, les autres. Pour
moi, il était là.

Je ne sais pas quelle autre chose cela pouvait être, parfois il parlait
bien qu'on ne comprît pas ce qu'il disait

S'il avait perdu quelque chose et vivait pour la chercher, c'était son
secret.

Chacun aura le sien.

Je me demande si ce ne serait-ce pas en lui un secret qui ne peut se
partager, celui qu'ils n'ont pas pardonné,

Et c'est pour cela qu'ils l'ont attaché ce soir-là pour montrer que
tous, tous nous sommes égaux, nous petits hommes, mystères qui doivent
être révélés coûte que coûte pour que tout, l'humain, reste sous la
lumière sur laquelle la lame du couteau puisse compter, qu'on puisse
refuser quand il s'agit de faire un mort de plus?

Alors.

Emporté attaché, c'était ça qui donnait envie de rire. Il allait
profiter pour continuer à chercher, tranquille, il regardait vers le sol
qui s'étalait caressait ses yeux, il voulait utiliser ses yeux jusqu'au
dernier moment, il avait encore l'espoir de trouver ce qu'il avait perdu
jusqu'à la fin. Et il ne disait rien.

Il y avait un arbre à l'endroit où il l'avaient mené.

Vois ceci maintenant: ils avaient aussi un secret.\\

Ils avaient donc fait venir une femme, ils lui avaient bouché les yeux,
et l'avaient envoyée maudire l'arbre pour qu'il ne laisse pas de fruits.
Les fruits.

Et ensuite, ils avaient cloué l'homme sur l'arbre.

Sumiro, l'homme resta cloué là

Alors arriva l'heure des couteaux,

La première heure des couteaux

Il y en avait eu d'autres après.

C'est dans cette première heure qu'ils lui avaient enlevé son sexe.

Un homme peut-il ne pas crier sa douleur?

Alors il cria. On entend encore son cri quand on passe par ce lieu à la
même heure, en fin d'après-midi. Dit-on. Moi je ne suis jamais retourné
là-bas, je ne sais pas. Il resta cloué sur l'arbre. Et après cela plus
aucun son ne s'échappa de lui.

Rien ne changeait ici.

Ils avaient emporté, traîné Sumiro et maintenant il était là, sur
l'arbre. Attend. J'essaie de me rappeler

Je me rappelle. Dans le ciel des nuages immobiles. Ce n'étaient pas des
nuages légers. Ils portaient en eux un rouge sang indélébile. Lourd. Les
gens d'ici savaient que la pluie allait tomber, mais quelle était cette
pluie, figée, là-haut, ça on ne savait pas

Nous avions le ciel au-dessus de nous et en-bas, nous étions seulement
des hommes, des femmes et il y avait aussi quelques enfants autour de
l'arbre.

Ce fut à nouveau l'heure des couteaux

Du côté de l'horizon s'éleva un grand bruit.

Il semblait que la pluie allait tomber. Mais elle ne tomba pas. Elle ne
tombait pas. Et ce fut l'heure où ils enlevèrent les yeux de l'homme.

C'est qu'il n'avait pas cessé de regarder le sol, de chercher, et il
fallait que cela s'arrête. Cela n'aurait servi à rien de lui faire subir
toutes ces choses s'il ne s'arrêtait pas de chercher.

Nous, nous regardions. Nous ne faisions rien.

Les témoins.

Ils voulaient que les yeux de l'homme cessent de chercher en roulant sur
le sol, et, ça nous l'avons vu, l'un d'eux sortit un petit sac, et y mit
les deux yeux, fermés,

Dans l'obscurité.

Comment allait-il chercher maintenant?

Je ne le savais pas.

C'est qu'à cette époque je voyais encore, j'avais mes yeux et je ne
voyais rien. Je voyais l'homme sans yeux sur l'arbre et je me demandais
comment maintenant va-t-il chercher, sans avoir de réponse~; mais lui
cherchait encore, maintenant je le sais, même s'il était autant dans
l'obscurité que ses yeux rangés dans le sac de l'autre.

--- Ne sois pas dégoûté. Il semble que c'est avec les yeux des gens,
n'est-ce pas?

Mais c'étaient ses yeux, pas les leurs, rappelle-toi.

L'homme regarda par la fenêtre et dit:

--- Seul peut être artiste celui qui aura une vision originale de
l'infini.

C'est Schlegel qui l'a dit

--- Je ne sais pas dit l'aveugle. Je ne sais rien. Je suis ici. Aveugle.

Je ne peux te dire que ce que je te dis, nous sommes tous ensemble dans
cette vie,

Hommes et dieux,

Ceux qui nous ont faits avec de la terre

Ceux de l'Eau, ceux de l'Air et ceux du Feu

Et ceux de l'Absence.

Maintenant je sais cela\\	

Dans le vent la voix passait à nouveau et disait\\

--- Viens Curau, viens emporter les hommes vers tes jardins\\

Et l'aveugle dit:

Je continue.

Plus tard, ils l'avaient mis nu, lui avaient enlevé ses vêtements, les
avaient brûlés. Les femmes n'avaient pas détourné leurs regards, c'était
une autre honte

Ils lui avaient coupé les mains, à la troisième heure des couteaux. Et
cela soulagea en lui une douleur, une douleur effaçant l'autre, tu
comprends: la douleur qu'il ressentait avant,

Celle des yeux arrachés

Mais à peine avaient-ils vu la nouvelle douleur naître et l'ancienne
disparaître du visage de Sumiro, qu'ils lui en causèrent une autre. Ils
mirent un grand clou, Noir, dans sa bouche. Et ainsi il ne pouvait plus
parler, même s'il n'avait rien dit pendant tout le temps que cela avait
duré. Cela, je le savais\\

Et cela dura.

En regardant, plus tard, on ne pouvait savoir où lui finissait et où
commençait l'arbre.

Ce fut ainsi.

C'est qu'ils avaient ouvert le tronc, à la quatrième heure des couteaux,
et mis l'homme dans l'arbre,

Une partie visible, l'autre cachée.

Ils voulaient de cette façon lui faire perdre de plus en plus sa qualité
d'homme. Tirer de lui tout mystère. Faire de lui un arbre.

Chacun comprendra à sa façon. Oui. C'est cela. Chacun est un autre.

Mais vois, c'était ce qu'ils voulaient.

C'est ainsi que je le comprends.

Je suis ici, aveugle. Ici c'est partout

Les vois de la terre viennent de loin

Un insecte m'a regardé

Qu'est-ce qu'il y a à voir?\\

Nous sommes restés là. Je regardais. Les autres aussi regardaient.

Et là, les jours allaient je ne sais où

Je suis là. Je regarde. Nous ne faisions rien.

Ils avaient mis l'homme dans l'arbre.

Ils décidèrent de lui donner à manger pour qu'il dure plus longtemps. Et
cet homme, notre ex-voisin, Sumiro, a accepté alors leur nourriture.

Les nuages étaient toujours là-haut, la pluie ne tombait pas, et il ne
mourait pas, cet homme dans l'arbre, il ne cessait pas d'être un homme
et ne devenait pas arbre une fois pour toutes. Il n'avait rien à nous
dire. Nous regardions\\

Le jour où il voulut rejeter la nourriture, il ne le put pas. Ils
avaient fermé la sortie.

Il dut souffrir aussi les douleurs communes.

Alors il ressemblait encore davantage à un homme.

Ça, ils le notèrent. Ils virent qu'ils ne devaient pas lui donner à
manger et ne lui donnèrent plus

Il était là, tout près. L'homme dans l'arbre. Et pourtant, bien loin de
ce lieu, une chose changeait en lui, elle restait dans ses rêves où rien
de tout cela n'arrivait

En lui, dans ses rêves, devaient parfois entrer l'une ou l'autre des
voix de ces hommes, de garde près de l'arbre

Mais cela se termina quand vint l'autre heure des couteaux et qu'ils
percèrent ses oreilles. Et qu'ils le laissèrent absolument seul, dans un
blanc,

Et il se fit un silence dont on ne connaissait pas l'existence.

Et ainsi il ne pouvait plus entendre quand de nouvelles heures
arrivèrent pour les couteaux, par vagues, les unes après les autres, en
résonnant,

Et ils lui arrachèrent ses jambes pour qu'il ne soit plus jamais humain

Mais, je crois, il recevait cela comme s'il était un autre, dans une
autre histoire. Et dans cette histoire ils voulaient peut-être lui
arracher son âme, son oiseau. Pour l'enfermer dans une caisse en bois.
En bois de l'arbre. L'arbre dans lequel son corps disparaissait en tant
que chose humaine. Mais dans cette caisse cette âme n'allait pas cesser
de voler, elle allait voler d'abord dans le corps de l'homme. C'est que
lui et l'arbre étaient alors devenus une seule chose.

Là-haut, le ciel s'agita. Etait-ce son sang qui finalement allait
tomber?

C'était cette pluie. Qui menaçait de tomber et qui ne tombait pas. Elle
ne tombait pas. Elle ne tomba pas et ne tombera pas

Les heures des couteaux allaient et venaient\\

Ils avaient coupé sa tête pour jouer avec. Et il comprit ce qu'était le
vide

Il avait déjà compris ce qu'était l'obscurité, n'est-ce pas?

Il avait déjà compris ce qu'était le silence

Et maintenant il comprenait ce qu'était le vide.

Ceux qui montaient la garde autour de l'arbre s'amusaient, riaient,

Est-ce que l'homme reste toujours un enfant? Je me demande. Ils riaient
et s'envoyaient la tête les uns aux autres. C'était pour passer le
temps, là. Ils la lançaient d'un côté à l'autre, la boule magique, elle
volait parmi ces hommes,

Et cela sans qu'il ne se sentît en rien humilié.

Mais les jours étaient nés. Ils étaient morts.\\

Prends encore un café. Prends. Cela va finir. Il faut que cela finisse.
Maintenant nous allons vers la partie où elle est autre, la vie\\

Ecoute.

Ils amenèrent la femme qui avait fait Sumiro. Ils la placèrent devant
l'arbre. C'est que maintenant le désespoir était tombé sur ces hommes.
Celui d'avoir perdu Sumiro dans cette vie. Ils voulaient le faire
revenir. Où pouvait-il être maintenant?

Ils se réunirent alors et parlèrent. Sous un autre arbre. A l'écart.

Ensuite ils revinrent.

Nous allons faire ainsi, dirent-ils. Et ainsi firent-ils. Pour que
Sumiro revienne, ils allèrent le chercher là où il était encore et ils
arrachèrent les vêtements de la femme, et ils cherchèrent dans son corps

Et tout cela devant l'arbre pour qu'il voie sans les yeux.

Ils allèrent chercher un enfant aussi et ils dirent, C'est un garçon.

Ils cherchèrent des traces de Sumiro en lui.

Et ils allèrent en chercher d'autres, effrayés, et ils disaient, Ce sont
des frères.

Ils firent la même chose avec tous. Et d'autres choses aussi. Celles qui
n'ont pas de nom.

Sans nom aussi ce qu'ils finirent par faire de lui.

Comme ils ne voulaient pas rester sans Sumiro,

ils se transformèrent, là devant nous, en hommes-sables, secs, et ils
burent son sang,

ils se transformèrent en hommes-chiens, ils étaient enragés, et
rongèrent ses os,

et revenant à leur état d'hommes qu'ils étaient, ils travaillèrent,
travaillèrent, organisés, précis. Et nous vîmes qu'avec la peau de
Sumiro ils avaient tissé un hamac, et qu'ils l'avaient fait sécher sous
un soleil violent qui était apparu dans le ciel entre les nuages de
cette pluie qui ne tombait pas, ne tombait pas, ne tomba pas et ne
tombera jamais, et se montra à nos yeux. Les témoins.\\

C'est ainsi que cela se passa.

Et une roue tourna. Lentement

D'abord vers la gauche.

Puis vers la droite.

Ensuite elle alla une fois vers la droite une fois vers la gauche, comme
un doute. Et à chaque tour qu'elle finissait, nous voyions

Les fruits naître dans l'arbre.

Alors nous avons dansé. Les témoins.

Et nous avons chanté. Nous applaudissions et nous buvions une boisson
amère, en tournant autour de lui enfin calme, ou serein, ou désespéré et
sans savoir pourquoi

\pagebreak

\vspace*{4cm}

Maintenant va-t-en,

Dit ensuite l'aveugle à l'homme. Et l'homme s'en alla

Je veux rester à nouveau seul.

Rester ici. Dit l'aveugle.

\pagebreak

\vspace*{4cm}

La mémoire.

C'est à nouveau elle. Mais c'est la mémoire d'un autre dans cette voix
qui vient parler à Jacinto,\\

et arrive dans le vent\\

Quand notre frère est né, l'horloge du salon s'est arrêtée. Je me
souviens

Elle est restée à l'heure qu'elle marquait.

Je me souviens

Notre père a dit, de Lui, de ce fils qui naît de moi viendra
l'allégresse. Il a dit cela et en pâlissant, tout à coup plus vieux, il
s'est retiré, s'est enfermé dans sa chambre et reste muet depuis lors,

une autre machine cassée dans cette maison.

Ce père.

C'était un enfant. Il ne pleurait pas. Il ne riait pas.

Il nous regardait seulement, distant. Le Distant. Il était là parmi nous
et n'y était pas. C'était comme s'il était toujours à l'endroit d'où il
était venu

Il est venu de notre mère, dit l'un de mes frères.

Mais moi je sais qu'il est venu de plus loin quand je vais le voir.

C'est notre dernier-né. L'Eternel. Il ne grandit jamais. Et les années
passent. Attendons

Notre père a dit cela. Et maintenant nous attendons l'allégresse.

Et cette attente nous rend inquiets. Tout est effroi. C'est cette
attente

Une bruit, une vitre qui se casse

Les accidents, les jours. Ce n'est pas cela l'allégresse.

Nous sommes allés voir le verre cassé à la cuisine et ce n'était pas
elle.

Et alors, comme rien ne s'est passé, nous avons cherché dans toute la
maison un indice de son arrivée.

Mais nous ne sommes jamais d'accord.

Elle arrivera le jour des morts, dit ma mère. L'allégresse.

Une de mes sœurs veut qu'elle arrive un jour de fête.

Nous avons ce père et il cherche dans les poches des vêtements, renverse
tout, veut trouver l'autre côté des choses, une lettre perdue, un nom
noté dont il ne sait plus où il l'a rangé. Et ses doigts tremblent. Il a
peur de toucher tout à coup la chose et cela n'arrivera pas un jour
comme les autres si elle vient, quand elle viendra et c'est à l'un de
nous de trouver.

Moi je sais.

Notre allégresse viendra quand il commencera à grandir

Il reste là dans la chambre. Et il ne grandit pas. Nous attendons.

Entourant le lieu où il est, couché, au fond d'un puits d'où nous
voulons tirer quelque chose

L'un de nous tente de deviner. On parie. Rien. Et on s'impatiente.

Parfois un autre pleure.

Ma mère crie.

Dans le salon, l'horloge est encore arrêtée. C'est son premier mouvement
que j'attends. Ce sera quand les aiguilles se remettront à bouger que je
serai avisé. Elle arrive. L'allégresse.

l'enfant est toujours à l'endroit d'où il est venu. Et pour cela je
reste aussi arrêtée dans le salon. Et je regarde l'horloge.

Dans la chambre notre nouveau-né, éternel, attend encore.

Lui seul sait quand elle arrivera. L'Allégresse.

J'espère seulement qu'elle arrivera avant notre mort

\pagebreak

\vspace*{4cm}

Là dehors maintenant c'est la nuit.

--- Cette nuit ne passe pas, dit l'aveugle à sa fenêtre.\\

Quand il fait nuit, là dehors Jacinto pense qu'il entend les ailes
rouges qui viennent du ciel. Maintenant encore il les a entendues à
nouveau. Les ailes

Elles arrivent

Elles sont en train d'arriver\\

Ce n'était pas le Curau de retour cependant.

Et il attend. Cette attente. Et les années passent

\clearpage
\thispagestyle{empty}
\movetoevenpage

\vspace*{4cm}

J'écoute.

C'est du fond de la tête. Et arrive dans le vent:\\

Hier nous sommes retournés au pont.

Sur lui le bois a perdu sa forme de planches, j'ai perdu cette forme
humaine qu'on m'a donnée, dit le bois\\

Ce pont vieillit de plus en plus. Il dit je tombe en morceaux,

et quand nous passons dessus l'un de nous tombe aussi dans l'eau, là en
bas. Pas tous ne remontent à la surface.

--- Il y a un mal avec des yeux d'enfants et des dents là en bas,
disons-nous les uns aux autres, et nous rions. Mais nous restons sur le
pont.

Il y a un mal là en bas, répétons-nous. Et nous rions.

Et nous faisons trembler le pont en sautant dessus.

Quand nous sommes arrivés au pont la nuit tombait déjà.

--- Je regarde ce pont et cela me rend triste, dit le vieux.

Il vient toujours avec nous.

Je vois l'un de vous tomber là en bas et ne plus revenir,

dit-il. Et cela me rend triste.

Et si parfois j'en vois un autre sauter de joie en tous sens, je suis
joyeux moi aussi.

Voyez, nous ne sommes pas plus que cela, et cela va être changé par la
vie.

Tout vient de dehors et entre.

Et il y a aussi des choses qui sortent de moi et entrent dans la vie.

Et nous avançons

C'est ainsi.

Nous vivons. C'est peut-être le pire.\\

Quand le vieux parle, sans qu'on comprenne toujours ce qu'il dit.

Nous écoutons. Il parlait. Ensuite il s'est éloigné de nous.

Il reste là assis sur la rive, seul, et il regarde l'eau passer sous le
pont.

C'est quand l'après-midi se termine que nous allons vers le pont. Réunis
dans une rue de la ville, tout à coup nous partons, vers le pont, vers
le pont

Et les uns disparaissent dans le chemin, peut-être sont-ils allés chez
eux peut-être pas, on ne sait

d'autres attendent pour tomber du pont quand nous arriverons, ce qui est
une autre façon de s'échapper

Ces disparitions ne sont cependant pas tout.

Il arrive aussi que quelques-uns disparaissent seulement en partie. Un
accident. Et nous avons un mutilé pour rire, il devra maintenant marcher
sur une seule jambe pour la vie entière. Celui-la saute. Ce n'est plus
un homme entier. Nous sommes ainsi. Nous nous faisons par morceaux. Nous
restons penchés sur le pont à regarder l'eau s'assombrir\\

Aujourd'hui, jusqu'à présent rien n'est arrivé. Nous attendons que la
nuit soit tombée complètement

Alors l'un de nous y va.

Maintenant il entre dans l'eau.

Nous le voyons entrer en se rappelant comment auparavant les autres sont
entrés aussi dans l'eau sombre pour la dernière fois

Et je dis:

--- Pierres. Les pierres.

Et nous allons prendre les pierres. Elles sont là, elles pèsent pour
toute la vie. Mais elles seront légères dans l'air

Et je donne l'exemple, que les autres imitent et celui qui est entré
dans l'eau sort en courant, fuit, il a du sang sur la tête, il va
s'asseoir à l'écart à côté du vieux. Il y en a un autre là. Nous
regardons. Maintenant à l'écart ils sont deux.

Nous sommes sur ce pont.

Nous attendons. Et nous voyons ces autres là à l'écart.

Ils sont seuls. Côte à côte. Maintenant ils regardent l'eau couler,
sombre

Sur la rive celui qui a reçu les pierres pleure. Un lapidé

Nous sommes tous ainsi, il n'y a rien à faire\\

Alors: cela doit arriver, dit l'un de nous maintenant.

Oui

Je l'ai senti à cet instant aussi.

Là sur la rive, celui que nous avons bombardé de pierres s'arrête de
pleurer. Il a levé les yeux. Il voit autour de lui des arbres immobiles,
il n'y a pas de vent. Et il court, vient à nouveau vers nous, sur le
pont, nous nous serrons les uns contre les autres et nous attendons
qu'il arrive. Mais nous ne quittons pas le pont. Et seul le vieux reste
là-bas sur la rive à regarder l'eau couler

Nous ne savons pas avec lequel de nous ce sera.

Ensuite nous serons prêts à revenir demain, et tous les autres jours de
notre vie aussi.

Et pour partir d'ici.

Et cela est déjà en train d'arriver

L'un de nous ne sait plus où mettre les pieds, comment rester à la
surface, le bois se casse sous lui, nous ne ferons rien pour le retenir.
Et les eaux passent sous le pont. Nous sommes ainsi. Il n'y a rien à
faire. Cette fois c'était lui alors

Il s'en est allé. L'un de nous.

Nous regardons. Et l'eau est calme.

Et nous nous écartons à nouveau les uns des autres. Le groupe se défait.
Chacun pour soi dans cette vie que certains appellent un pont. D'autres
l'eau. D'autres une noyade.

Demain il faudra venir à nouveau, en fin d'après-midi.

Sur la rive, le vieux se lève .

Nous rentrons.

Personne ne dit rien. Nous rentrons sans parler, dans la nuit, et nous
nous séparons aux angles des rues, pour rentrer chez nous. Et nous
disparaissons dans les rues. Il n'y pas de quoi se lamenter

\clearpage
\thispagestyle{empty}
\movetoevenpage

\vspace*{4cm}

Le vent a déjà fait le tour de toute la terre avec l'histoire que
l'aveugle a racontée

Et maintenant il passe à nouveau par la fenêtre de Jacinto\\

--- Ils sont venus et l'ont enlevé, attaché. Vois, par où ils l'ont traîné
en écrivant sur son corps cette histoire que je te raconte maintenant en
l'inscrivant aussi dans ton oreille

Dit le vent, une seule rafale qui passe par la fenêtre de Jacinto en
apportant un morceau de l'histoire de Sumiro, qui reste dans l'air après
que l'aveugle l'a contée.

Et il est passé. Et de ce morceau de voix de vent il ne reste plus rien\\

Une autre voix arrive ensuite en disant:

De l'endroit où je suis, je ne peux voir que l'horizon et un fleuve.
C'est moi qui ai fait cette maison

Il y a un homme dans le fleuve. Il me voit. Il vient par là. Nous
parlons. Il me raconte sa vie et ensuite cet homme me demande de partir
avec lui.

--- Ce fleuve va par là, il montre du doigt.

Il dit:

--- Allons

Je sais que cette île n'existe pas, mais je ne dois pas y aller. Non je
n'irai pas.

Et quand il rame à nouveau au loin, je détourne mes yeux, j'ai peur de
voir comment l'horizon va le dévorer. C'est ainsi que la forêt
maintenant se referme sur moi aussi. Peu à peu. Bientôt la nuit viendra
sur cette île. Et il y a des animaux qui seulement alors se
réveilleront. Je sais que cette île n'existe pas.

De l'endroit où je suis je ne peux voir que l'horizon et le fleuve. Je
vois un homme là. Et cet autre qui arrive par là, lui aussi m'a vu\\

Oui, Andara est vraiment notre Afrique, de ma fenêtre, aveugle, je le
sais, dit Jacinto.

Et je veux entendre le reste.

J'écoute.

Le reste.\\

J'écoute.

En sortant de la forêt, j'ai vu la maison.

Ça, la maison.

Un animal astucieux qui va ouvrir la porte et regarde à l'intérieur
d'une maison avec des yeux féroces comme ceux avec lesquels moi j'ai
regardé ne se rencontre pas tous les jours parmi les hommes

Chez les hommes ma loi c'est avance un pied et n'avance l'autre, resté
en arrière, que bien plus tard. Je ne prends pas de risque. Je suis prêt
à m'échapper. Je suis ainsi.\\

C'est de cette façon qu'alors j'ai ouvert la porte et que je suis entré

Ismaël n'était pas chez lui. Et j'ai fait tous les gestes qu'il fallait
pour préparer le piège. Ensuite, je suis sorti. Je suis resté dehors

Derrière un arbre, je voulais être là pour tout voir. Caché. Et
maintenant j'attends le retour d'Ismaël.

Mais il me voit en arrivant.

Il a de bons yeux Ismaël, me dis-je, mais il ne verra pas le piège. J'ai
f ait tous les gestes qu'il fallait

--- Je ne vois plus de rancune dans tes yeux , ah

Me dit-il

Entre.

Et nous entrons.

Ismaël passe devant. Cette maison est la sienne

--- C'est aussi la tienne, veut-il commencer à me dire

mais un bruit vient se mêler à sa voix et l'interrompt, un autre bruit
et alors ah voilà Ismaël dans les airs. J'ai utilisé de grosses cordes.
Grosses. Suspendu ainsi il ressemble à un oiseau sans ailes, muet.
L'effroi.

Ismaël, Ismaël, je crie. Maintenant tu es foutu.

Et je volais à travers la maison autour de lui. J'applaudissais. J'ai ri
et j'ai dansé, ainsi, oublié, je me risquais, sans Ismaël sur mon chemin
je peux laisser mes pieds aller où ils voudront, disais-je,

Et le reste du piège entra en action, le reste que j'avais préparé, une
seconde attaque au cas où il échapperait à la première. Si Ismaël avait
eu des ailes, elles saigneraient,

et le piège me souleva aussi pour me balancer dans les airs à ses côtés.
Pendus tous les deux, maintenant nous sommes deux oiseaux. Cela ne
devait pas se passer comme ça, me dis-je

Non.

Et Ismaël est maintenant en train de me dire, Il faut que tu te
réveilles. J'entends sa voix au loin

Tu vois, tout cela nous fait du mal à tous les deux

--- Oui, ai-je répondu

Et je me suis réveillé.

Mais maintenant je me frotte les yeux assis dans mon hamac et je me
rappelle mon rêve, et ce oui veut se transformer en non. Et ce que je
veux c'est rêver encore une fois

--- Et c'est pour ça qu'ils l'ont traîné cet après-midi-là pour montrer
que nous tous, tous, sommes égaux, de petits hommes, des mystères qui
doivent être révélés, coûte que coûte

C'est le vent. Une de ces nouvelles rafales est en train de passer par
la fenêtre de Jacinto encore une fois.

Après, il cesse

Et l'aveugle se dit, je reconnais cette voix.

L'histoire de Sumiro ne veut-elle pas s'évanouir dans l'air?\\

Elle arrive dans le vent. J'écoute. Aveugle\\

Quand on a peur des hommes il faut aller chez un ami

Ainsi j'ai cherché Fabiano ce matin.

Nous avons marché dans son jardin.

Il me montrait ses animaux, les ombres que leurs corps projetaient, tout
ça était là autour de nous

De retour à cette maison où je m'occulte, occulte, plus tard je ne me
rappelais plus ce que nous avons fait ensemble.

Que se cache-t-il de la lumière des journées?

Que se cache-t-il dans la lumière des journées?

La nuit, je vois des mains tachées de sang. Et je revois tout ça dans un
rêve

Fabiano qui me laisse tuer tous ces animaux-là. Et il me disait, Quand
ta peur reviendra, toi aussi, tu peux revenir. Ils renaîtront tous\\

Jacinto écoute.

C'est une voix de plus. C'est du fond de la vie:\\

--- Aujourd'hui je vais essayer encore une fois

Lentement, affreusement, cette fois-ci je m'en vais d'ici

--- Je m'en vais,

je le crie avec mon cri le plus rouge et il ne cesse ni même quand je me
trouve face à la bouche tordue par un rire sur le miroir et où je me
vois pendu pour ne pas me sentir seul lors des jours où rien n'arrive.\\

C'est les rues que je veux. Et dès maintenant, je tremble quand je me
vois, en train d'anticiper, et les choses qui vont se passer dès le
moment où j'y serai. Les mauvaises choses et les autres. Toutes. Et ne
me parlez pas de la peur

Je répète. Je m'en vais d'ici.

Et déjà je me hâte.

Même si je prends un risque, celui de sortir par l'arrière de la maison
et d'arriver juste à l'arrière-cour

Et de rester coincé entre les murs, si je rate la direction de la porte
d'entrée.

Pour cela, je dois avoir de petits yeux vifs. Et pour que ça n'arrive
pas à nouveau , je dois garder la voie. Et m'en aller. Maintenant. Les
pieds bougent déjà. Je suis en train d'y aller. Vers les rues. Les rues.
Mais la porte d'entrée semble me conduire seulement à la chambre. Voire
vers sa plus grande profondeur et sous les yeux de celui qui ne me perd
pas de vue et sourit, dans le miroir, plus que jamais, sous le lit. Où
je me mets et je me sens bien maintenant.

Demain, je vais essayer encore une fois.\\

--- Nous regardions. Nous ne faisions rien. Les témoins, dit le vent

--- Et nous chantions. Nous applaudissions, dit le vent

--- Et nous buvions un breuvage amer, dit le vent\\

en passant, à nouveau, par la fenêtre de Jacinto avec l'histoire de
Sumiro.

Et cette dernière rafale qui passe en ce moment va encore une fois vers
le lointain.

Le vent va refaire le tour complet de la terre

Le vent va raconter l'histoire de Sumiro à d'autres hommes

\clearpage
\thispagestyle{empty}
\movetoevenpage

\vspace*{4cm}

D'un pont

D'une île

On me parlait d'une île qui n'existe pas, se dit Jacinto. Combien de
temps est déjà passé depuis? Les années\\

Nous avons Jacinto, et Jacinto est l'homme à la fenêtre

\pagebreak

\vspace*{4cm}

Il y a un autre temps pour l'enfance maintenant.

Là, pendant l'enfance, ce temps d'étonnement partout, où tout a
commencé. La vie

Se dit Jacinto à la fenêtre\\

Et une voix de plus arrive dans le vent. C'est la voix de Jacinto un
garçon et dans cette voix où le vent apporte l'enfance, Jacinto va
revenir\\

Je me rappelle Andara comment c'était auparavant, dit la voix.

Ah l'enfance, se dit Jacinto.\\

C'est à Andara où cette ville, Santa Maria do Grão

Andara est un lieu qui fait peur

Andara a été la première partie de la ville à apparaitre. Puis, la ville
a augmenté de plus en plus et aujourd'hui Andara est un lieu presque
oublié, un souvenir pour les noyés,

elle est restée là sur une rive d'une rivière

il y a des eaux profondes, lentes, elles passent.

Andara est un enchevêtrement de maisons dont les portes et fenêtre
donnent sur la forêt. Vous sortez de chez vous et à peine vous êtes
sortis il y a des dents, des yeux autour de vous

On entend des souffles. Une peur arrive et elle devient immense. En
haut, la lune, toujours. Blanche, s'il ne fait pas encore nuit. Donc, on
rentre et on verrouille la porte. Ce jour-là, on ne sort plus de la
maison. Andara, c'est ça. Dans cette ville, ou dans une autre, peut-être
celle qui est occulte, où elle a commencé. Santa Maria.

C'est Andara où Santa Maria do Grão a commencé. Dans l'enchevêtrement.

Un enchevêtrement est toujours vert, me dit-on.

Mais, parfois, il devient tout noir.

Il n'a pas de fin, ses fleuves qui n'existent pas et ces arbres absents
autour de moi s'étendent à perte de vue. Cela s'étend jusqu'où l'homme
peut aller. Et il va encore plus loin. Voilà la région. Et Andara est
bien plus~encore : Andara est tout l'enchevêtrement.

Je me demande, un garçon doit-il tout savoir?

Les histoires d'Andara que j'entends

Ici des choses se passent aussi.

Santa Maria do Grão aussi devient folle.

Mais à Andara on me dit c'est pire. Là on marche dans une rue et, alors,
on voit, d'une fenêtre, on nous regarde. Qui est celui à la fenêtre qui
me regarde sans vouloir être vu et voulant savoir qui suis-je,

on s'arrête dans la rue et se demande.

C'est juste la vitre de la fenêtre, et dans la vitre un autre, après on
le sait. C'est vous-même\sout{s} qui étiez là et vous regardiez. A ce
moment-là, cependant, il n'était pas possible de le savoir. Il y un
autre là-bas, il nous regarde. Et voilà tout. Et l'autre est réel

Andara est un lieu comme ça.

Comme tous les autres lieux, tous,

me dit ma tante.

--- Ça a un nom, dit-elle.

 Mais, parfois, elle, la folie, elle aussi peut venir se montrer d'un air
plutôt de grimaces, rires indéfinis et l'on est pris de désespoir de se
rouler par terre avec des cris qu'on peut entendre de loin

Mais à Andara. Voilà cette tante en train de me raconter des choses.

--- Andara, dit-elle, est là. Entre la forêt d'un côté et un fleuve de
  l'autre, avec une vocation pour la mort qui n'est pas vue par ici.
  Elle est là avec cette vocation et un souhait d'aller toujours vers un
  mystère plus profond, et vers d'autres, sans fond. C'est ça qui a fait
  partir Irido, c'est pour ça qu'il est allé habiter là-bas. Il portait,
  lui-aussi, cette vocation pour la mort.

Irido est l'homme qui l'a abandonnée.

Donc elle raconte l'histoire du cimetière.

C'est à Andara que cette ville a eu son premier endroit pour garder les
morts, dit-elle. Si on n'ouvre pas tous les yeux et ceux de l'instinct
de s'enfuir de la mort on finira aussi par se perdre, entrant dans une
rue inconnue et sortant dans une autre jamais vue auparavant, et on est
troublés, et sent qu'on est entraînés, emportés, et alors, de rue en rue
on finit par se retrouver dans le cimetière d'Andara,

qui a déjà été envahi par la forêt, se mêlant l'un à l'autre, où les
morts et le végétaux sont ensemble,

et on marche parmi les petites maisons de terre des morts, dit ma tante,
et on écoute comment la terre a des anciennes voix, par là, rumeurs.
Après quelques jours, et les jours ne passent pas, parmi les morts, une
personne à qui tout ça arrive ne sera plus jamais elle-même.

Irido y est allé parce qu'il ne me voulait plus, elle le répète toutes
les fois pour finir l'histoire.

Et se tait. La femme.

Irido. L'homme.

--- Mon oncle Irido veut que j'y aille avec lui, je lui ai dit un jour.

--- Non. Ne va pas.

Ça, elle me l'a dit ainsi, sèchement.

Et c'est toujours la même réponse quand je dis que l'homme veut que j'y
aille pour rester avec lui.

Cet homme-là à Andara.

Mais lui, il insiste.

Il me demande d'y aller. Il le demande souvent dernièrement.

Aujourd'hui encore, un homme est venu d'Andara, il est entré chez nous
et a dit:

--- Irido veut savoir si le garçon y va ou pas

Ces jours-là j'avais mes soeurs et leurs yeux, effrayés

Elles ont eu encore plus peur en entendant ce que l'homme disait, une
peur plus grande que celle à l'époque où notre père est mort et notre
mère est morte et nous sommes restés ainsi, vivants

L'homme a dîné chez nous et s'en est allé.

Mes soeurs ont peur qu'Andara, lui, m'emporte vers le cimetière

Pour elles, Andara n'est pas uniquement ces maisons-là vides, ce début
oublié de ville. C'est Andara et il nous attrape, nous entraîne vers la
mort. Et alors, il n'y a plus de retour pour ceux qui sont emportés par
Andara, disent-elles. Et elles me regardent. Et murmurent. Et pleurent.
Et me regardent comme si tout s'était déjà passé. C'est comme ça pour
ceux qui ont peur. On anticipe

Peut-être c'est pour cela que ma tante vient de décider.

Elle m'a laissé partir maintenant.

Elle aussi, elle a dû anticiper ce qui m'arriverait si j'y étais allé,
il fallait s'y soumettre pour s'en débarasser

Je m'en suis allé.

L'enfance. Elle est exactement ce temps d'étonnements partout. Et elle
ne finit jamais, je sais.

Il était là, cet homme.

Mon oncle, qui avait une nuit dans les yeux.

Il m'a pris dans ses bras. Et il m'a emmené manger la viande d'un animal
qu'il avait attrapé, je l'ai attrapé ce matin, m'a-t-il dit, pour
t'attendre. Cette nuit-là dans ses yeux

À table, pendant que nous mangions, il m'a dit, demain, si tu aimes
cette viande, nous allons en attraper un autre.

Manges-en plus.

Tu aimes~ça?

Il me parlait de l'animal et m'offrait plus de viande.

Mais aujourd'hui, quand je me suis réveillé, j'ai eu peur d'aller avec
lui attraper un autre animal. Et je n'y suis pas allé

Comme tout passe vite à Andara.

Je lui demande souvent, quel animal est-ce, quand nous mangeons, tous
les jours il en attrape un et l'apporte à la maison. Cette maison
n'existe pas, elle est trop vieille, ses murs sont troués et on peut y
voir à travers, dehors, les arbres. Et tout ce qui y vit. Allons voir de
près l'animal, mon oncle me répond. Mais je n'y vais pas. Dans cette
maison nous mangeons la viande de l'animal tous les jours, que mon oncle
met sur la table encore une fois. Le dîner.

--- Aujourd'hui, et aujourd'hui tu veux aller avec moi en attraper un?

Il me le demande tous les matins. Et il sort.

Il revient avec un animal. Déjà mort. Et sans la peau. Et coupé en
morceaux, c'est pour que je ne sache jamais quel animal est-ce.

C'est une bonne viande. Il n'y en a aucune pareille.

Et aujourd'hui?

Je n'irai jamais avec mon oncle je pense.

Je ne sors pas non plus de cette maison pour ne pas me perdre dans les
rues et finir au cimetière.

Et aujourd'hui? M'a-t-il demandé encore une fois ce matin, avant de
sortir attraper encore un autre animal.

C'est dans le cimetière qu'il les attrape tous.

Dans la partie du cimetière qui est envahie par la forêt, là où la forêt
est bien vivante, et avance toujours

Et aujourd'hui?

Je n'ai pas répondu. Je ne réponds plus. Il comprend que si je ne
réponds pas je dis non, je ne vais pas. Et il sort.

Et aujourd'hui? Me demande-t-il et il sort.

Il ne lève pas les yeux pour poser la question. Et il a toujours la nuit
dans ses yeux. Je sais.

Je n'y vais pas, je n'irai jamais

Et aujourd'hui. Me demande-t-il.

Ce que je ne veux pas c'est voir l'animal vivant et, après, mort. Je
n'oublierai jamais son goût pourtant

\pagebreak

\vspace*{4cm}

C'est la fin pour l'enfance maintenant.\\

Et cette voix qui dit dans le vent

--- Viens Curau. Viens emporter les hommes vers tes jardins

\pagebreak

\vspace*{4cm}

Mais, à nouveau, voilà une voix qui revient.

Qui dit:

--- A Santa Maria do Grão, celui qui passe par cette rue ne sait pas
  pourquoi il voit une cage en regardant la maison. Celui qui passe
  regarde la maison. Pour ne plus se tourmenter avec ça, il veut trouver
  une explication, et s arrête, regarde bien et ensuite il s'écarte et
  se dit que cela n'était qu'une illusion, que cela vient des grilles de
  la maison

--- Et l'impression part avec lui, elle accompagne des jours durant celui
  qui a vu la maison

Il n'arrive pas vraiment à s'en distraire, le voilà revenu devant cette
maison. Il cherche dans sa mémoire où il a bien pu voir cette maison, se
demande celui qui est passé.

L'enfance n'a pas de fin\\

J'attends le retour de l'oiseau. Et en attendant, j'entends les voix

Les voix de la terre viennent de loin. Pour les entendre il suffit de se
permettre de rester, ne jamais partir. Je reste. Le vent les apportera
toutes.\\

Il y a en d'autres, comme moi, partout? N'entendent-ils, comme moi, ces
voix?\\

ça, avait dit Jacinto, et cet homme était parti, parti, il était revenu,
parti encore une fois.\\

Ceci est encore un homme. Un insecte peut-être me regarde. Et ceci
insiste à comprendre

Je suis là. Aveugle.

Ici c'est partout\\

Dit Jacinto. Et Jacinto est l'homme à la fenêtre.

\clearpage
\thispagestyle{empty}
\movetooddpage

\vspace*{4cm}

Si tout continue ainsi indéfiniment

pendant des jours et des jours, un jour viendra où Jacinto ne sera plus
Jacinto.

Et il y aura un autre à sa fenêtre, qui attend encore

Bouh. Un fantôme.

Quelques os blancs

bouh, un son pour effrayer les enfants\\

Le matin, il apparait à sa fenêtre.

La fenêtre sera trop vieille.

Tous les jours, bouh apparaitra à sa fenêtre. Le bois de cette fênetre
n'aura plus d'âge, il aura des craquelures qui diront nous sommes des
choses mortes d'un fantôme\\

Et la vie, comment sera-t-elle si le Curau ne vient pas

\pagebreak
\pagecolor{black}

\chapter*{}
\pagecolor{black}\afterpage{\nopagecolor}


\movetoevenpage

\vspace*{4cm}

Bouh, dit le vent qui passe par Jacinto encore un homme.

Car bouh sera Jacinto, sauf si le Curau revient

\pagebreak

\vspace*{4cm}

Pendant plusieurs années Jacinto les entend encore de sa fenêtre. Les
voix.

Et les jours sont passés vite par lui. Et ils ne passaient pas

\clearpage
\thispagestyle{empty}
\movetooddpage

\vspace*{4cm}

L'homme était là, aveugle.

Quand l'autre est arrivé, il a dit:

--- Ah, tu es revenu. Appelle-moi Jacinto. Je suis là, aveugle. Ici c'est
  partout

Tu as ramené un garçon cette fois-ci. C'est ton fils, tu veux qu'il
écoute aussi l'histoire du Curau. Oui. Tu es revenu

Je la raconte, je la raconte encore une fois.

Oui. A Santa Maria do Grão ces choses arrivent.

Viens ici, garçon. Tu es une nouveauté pour moi. Le Curau est venu il y
a longtemps, tu n'étais pas encore né. Puis, il s'en est allé. Mais un
jour il reviendra. Toi aussi, tu dois savoir comment tout cela est
arrivé.

Le garçon écoutait.

L'aveugle parlait.

--- Il y a longtemps. Mais je me rappelle tout. Je n'oublierai jamais.

Le garçon écoutait.

Viens ici, dit l'aveugle.

Laisse-moi toucher tes yeux.

Il l'a laissé faire.

Les yeux. Ceux-là. Les tiens.

N'aie pas peur. Je touche. Je sais.

Tu es un enfant, les enfants ne doivent pas avoir peur du Curau. Que tu
ne sois un de plus à courir dans les rues, au moment où le Curau
reviendra. Cachant leurs yeux, cherchant un lieu pour se cacher, criant
le Curau, Curau. Il arrive. Comme ils criaient. Tellement. Les cris. Et
plus tard, ils ne fermaient plus les yeux la nuit par peur de n'avoir
plus d'yeux à ouvrir le matin.\\

Le garçon écoutait et le père a voulu regarder par la fenêtre.

L'aveugle racontait

Tout a commencé comme je te raconte maintenant. Je te le raconte encore
une fois. J'ai été averti avant les autres.

L'oiseau était tout rouge. Il était là, arrêté. Il semblait malade. Je
lui ai donné le nom que j'ai voulu, Curau. Le premier nom qui est sorti
de ma bouche. Il n'y a que comme cela, apprends, qu'on peut trouver le
nom caché d'une chose cachée. Et celui-ci n'était pas un oiseau comme
les autres\\

Le garçon écoutait.\\

La peur ne vient que chez ceux qui ont déjà de vieilles raisons d'avoir
peur, et ceux-là ont commencé donc à avoir peur du Curau. Ceux-là ont
peur de tout, disait Jacinto. Et le garçon écoutait. Quand l'oiseau est
arrivé, cette peur-là marchait dans les rues avec des pas qui ne nous
conduiront qu'à un terre non-sacrée

Regarde, sauf les enfants qui ont peur de choses réelles, disait Jacinto\\

Comprends, quand un garçon crie le Curau, le Curau, il ne fait qu'imiter
les autres. Si les adultes n'ont pas peur, les enfants n'auront pas peur
de l'oiseau. Ils vont rester dans leurs hamacs, calmes, et les soirs ne
se seront pas aux aguets.\\

Le Curau n'a percé que les yeux des adultes, quand il venu pour la
première fois, disait Jacinto\\

Tout le monde doit demander la venue du Curau. Son retour. On doit lui
demander tous les soirs, et dire, comme dans une prière, viens Curau et
aveugle-moi. Délivre-moi de ces yeux ne voulant plus voir les choses
ainsi, de la même manière

Il n'y a que toi qui n'ait pas besoin de demander, mon garçon, disait
Jacinto.

Demander que le Curau vienne t'aveugler.

Mais demande-le pour ton père.

Demande le soir avant de dormir.

Le Curau a été un bien qui nous est arrivé, mon garçon, et un jour il va
revenir, disait Jacinto\\

Repoussé par l'aveugle, ensuite.

Le garçon s'est écarté.

Maintenant, va t'en. J'ai déjà tout dit. Maintenant je veux rester seul
ici, assis à côté de la fenêtre. Aveugle. A dit Jacinto.\\

Dans la rue l'homme disait au garçon

--- Non

Le garçon a voulu regarder encore une fois vers le ciel.

Dans le ciel il y avait des nuages, il y avait une tâche. Elle était
grande. Rouge.\\

Peut-être maintenant un insecte me regarde pour comprendre, dit Jacinto
à sa fenêtre.

A sa fenêtre, il est en train de dire, viens Curau

\pagebreak

\vspace*{4cm}

Et cette voix qui dit dans le vent\\

Nous avons entendu, encore une fois

--- Viens Curau. Viens emporter les hommes vers tes jardins

\vfill
Fin de «Les jardins et la nuit»\\

Le voyage à Andara n'a pas de fin.


\clearpage{\pagestyle{empty}\cleardoublepage}
\addcontentsline{toc}{chapter}{Os jardins e a noite, \emph{de Vicente Franz Cecim}}
\movetooddpage
\part*{Viagem a Andara oO livro invisível\\
\bigskip
\bigskip
\bigskip
\bigskip
\HUGE{Os jardins e a noite}\\
\Large{Vicente Franz Cecim}\\
\bigskip
\bigskip
\normalsize{A liberdade é uma noite escura?}}



\chapter*{}
%\addcontentsline{toc}{chapter}{Original em português, de Vicente Franz Cecim}

\selectlanguage{brazilian}

\forceindent{}O que contém os animais?

O que os animais contém?\\

A estranha ave que cega os homens. Não proteja os seus olhos. A liberdade é uma noite escura?

Mais uma vez, bichos e homens

\pagebreak

\vspace*{4cm}

O labirantro.

Ele é Andara. Andara é onde Santa Maria do Grão começou, como se verá
agora nestes jardins, fazendo a floresta se abrir, recuar, e é por
Andara que a floresta está voltando.

Toda essa criança aí

Andara é o lugar de um mito entrevisto?

Talvez, mas não será só isso talvez

De qualquer modo, é aí que se enlaçam, se negam, natureza e essa outra
coisa inquietante que tem um nome. Civilização.

Toda essa criança escuta

A voz. Do labiantro

\pagebreak

\vspace*{4cm}

Escutará também o que não é Andara?

Andara é a viagem fora de si e deverá continuar sendo isso, um gesto sem
gesto, estará em outra parte

Isso ficará em branco. A vertigem. Tira a terra de sob os nossos pés e
no entanto não a perderemos de vista.

Os olhos que antes leram a história do Nazareno e o livro do cego Dias,
assim eles também liam um livro que não estavam lendo.

Talvez agora não sem surpresa saibam ou já sabiam? Que sob livros há não
livros.\\

Viagem a Andara, o livro invisível

\pagebreak

\vspace*{4cm}

Bem.

No labirinto se deve dizer aos outros: sem um texto, não há tempo

E assim, há a maneira infinita de ler Andara.

Um jogo de deslocamentos e, às vezes, reuniões arbitrárias, segundo cada
um e, em cada um, segundo o tempo de uma emoção

Dança para as intuições.

\pagebreak

\vspace*{4cm}

Nessa dança, a seqüência dos livros de Andara fica abolida. E se pode
iniciar a viagem por onde se abrir primeiro.

--- Ó flor Andara, de sonhos

Isso não tem um limite.

Isso se desregra. Mais tarde uma leitura de Andara na memória encontrará
novas combinações. Eis um ideal: a imaginação na penumbra, libertada
pela ausência física de um texto

Os desastres curiosos que então ocorrerão: quem voa ao lado de Caminá, é
o pássaro Curau? E: existiu um livro chamado Os dias cegos, onde se
conta a história do cego Dias?

Mas é mais: tendo arrancado as páginas de todos os livros,

e isso seria luminoso se se desse num momento de revolta,

o leitor, com a ajuda de um vento inesperado que primeiro espalhe e
depois reorganize a viagem a Andara. Paro. As janelas devem ficar sempre
abertas para que os ventos sempre entrem e nunca uma ordem final se
instale. Rigorosa.

\pagebreak

\clearpage
\thispagestyle{empty}

\movetooddpage

\vspace*{4cm}

Esta viagem à coisa humana à noite\\

E abertamente votei meu coração à terra grave e sofredora e, muitas
vezes, na noite sagrada, prometi amá-la fielmente até a morte, sem
receio, com seu pesado fardo de fatalidade, e não desprezar nenhum dos
seus enigmas. Diz Hoelderlin

Esta noite a ser negada.

Haverá outra

E abertamente votei meu coração à terra e muitas vezes, na noite
sagrada, prometi amá-la fielmente e não desprezar nenhum dos seus
enigmas\\

O fardo de uma história com uma história.

Como seria uma história sem uma só história, o fardo feito em pedaços?

O não.

O labirinto.

O não. Não à uma história única. Ainda que seja a história de um homem.
Fazer com que por ela passem tantas histórias de outros homens quantas o
ouvido ouvir e o vento contar

O labirinto. Neste que começa agora, há vozes que esse homem em sua
janela escuta, reais,

vêm de documentos antigos sobre a dor e se misturam à encenação, feita
pela voz de Andara, da dor imaginada\\

Ei-lo, então.

O não labirinto

\pagebreak

\vspace*{4cm}

Uma espécie de murmurador na noite.

É isso a imaginação

Ela vem. Não se sabe bem de onde.

Eis o que temos. Um homem numa janela. Não há como evitá-lo. E há vozes.
Então é só entregar-se,

esta viagem fala da vida e não vai parar antes do fim.\\

Outra vez.

Como será?

\pagebreak

\vspace*{4cm}

O homem estava lá. Cego.

Quando o outro veio, ele disse:

--- Me chame Jacinto. Eu estou aqui cego. Aqui é em toda parte.

Sim. Em Santa Maria acontecem essas coisas.

Você veio para ouvir a história. Lhe disseram por aí que eu sei como
tudo se deu, então você veio. Disseram onde eu moro. Trouxeram você até
a rua e apontaram a porta desta casa. Então entre.

O outro homem entrou.

Ficou escutando.

Durante uma parte da tarde, o cego falou. E o outro ouvia.

Ele, o cego, disse:

--- Não faz tanto tempo.

E eu lembro de tudo. Eu nunca esquecerei.

Ouça.

Foi como agora estou dizendo, com estas palavras. Você pode não
acreditar que fui avisado antes dos outros, mas fui. Não há pelo que
jurar. Há apenas esta voz como ela é, e não dê seu ouvido ao vento da
noite, desta noite sem clarões, para iluminar enigmas, isto se
contorcerá com lâminas por toda parte, dê seu ouvido só a esta voz e
deixe que o vento leve, também, a parte que toca a ele nisto que vai
ouvir. Outros também precisam ficar sabendo como tudo se deu\\

Ouça.

A ave era toda vermelha. Estava lá, parada. Parecia doente.

Foi assim que tudo começou.

Eu dei a ela o nome que quis, já que as coisas que se conhece têm todas
um nome de uso comum, mas as outras, essas que vêm não se sabe de onde,
não se sabe nunca, a essas se pode dar o nome que se quiser, é só abrir
a boca e ficar com a primeira palavra que sair, é só nisso que há uma
possibilidade de achar o nome oculto da coisa oculta. E aquela não era
uma ave como as outras, isso eu vi logo.

--- Curau.

Foi a primeira palavra que saiu da minha boca. Aquela boca ficou assim
viva. Curau. E ficou sendo esse o nome da ave desde o dia de rumores e
de uma mudez, a minha, quando abri a porta e ela estava lá. Estava mesmo
doente, não fugiu quando me aproximei. Quis que entrasse em minha casa.
Cuidei dela. Logo ficou boa. Era forte. Grande. A ave.

É preciso que eu lhe diga: antes, eu estava de olhos fechados, e vi
tudo.

Foi antes de achar a ave. Eu tive um sonho.

Quando acordei, não entendi o que havia visto sem meus olhos. No sonho.

Agora sei que tinha sido uma anunciação. Havia visto os dias de medo que
mais tarde vieram se instalar entre nós

Quando acordei, fiquei revendo aquelas asas imensas,

uns olhos de fogo me espiavam do fundo da noite,

e no sonho que tive havia uma noite onde homens corriam, fugindo também
como as mulheres por ruas vagas, ruas onde não pesavam as pedras,

fugindo como passaram a fugir também nas ruas reais desta cidade.
Gritando. Querendo se esconder.

Acordado do sonho, eu ainda não podia entender, vivendo como vivia na
ilusão de que só o que se vê de olhos abertos é real, e espelho falso
tudo o que nos aparece quando se fecha os olhos. Não podia. Havia aquela
ave, ela voava por cima da cidade, ela mergulhava sobre uma rua e se
ouviam gritos. Os gritos.

Esse foi o início. E eu o vi.

Foi assim que a vida me avisou que o Curau ia chegar.

E depois ela, a vida, veja, veio pôr o Curau direitinho na porta desta
casa

Mais tarde entendi

Nós temos uns olhos, como eu tive, para tropeçar por toda parte. Um
espelho nos cega apesar do sol

Ele ainda está lá, no alto?

Ele ainda existe?

A noite hesita e nos recusa com esses olhos que temos para nada

Foi tudo como eu disse.

Antes de voar entre os homens, o Curau voou em Mim, entre os olhos que
eu tinha, fechados, e a vida lá fora. Voou primeiro num céu humano.

E então achei a ave e a levei para casa.\\

O medo veio para os outros só mais tarde.

Foi mais tarde que se começam a ouvir seus gritos. Então já era o Curau,
voando por cima da cidade. Voava e eles gritavam o Curau. Curau. Lá vem
ele outra vez. E também as crianças gritavam.

Elas não deviam ter medo, porém.

O medo só veio para aqueles que já tinham as suas velhas razões para ter
medo,

esses já tinham o medo neles e passaram a ter medo, então, do Curau.
Esses têm medo de tudo. E a ave era só a coisa agora visível que fez o
medo deles surgir, aquele medo andava pelas ruas com passos que nos
levarão, que levam sempre a uma terra não-sagrada. Andavam assim na
região de um deus sem Rosto, e no olho esquerdo dele haviam enterrado um
Espinho

Veja, tirando as crianças que têm medo reais

Não. Isso eu quero dizer de outro modo. Como entendi

Quando um menino gritava o Curau, o Curau, ele só estava fazendo como os
outros faziam, imitava

Se os adultos não tivessem medo, as crianças também não teriam medo da
ave

Ficariam nas suas redes, calmas. E a noite seriam sem espreitas, para
voar sem sustos os vôos infantis

Um Curau nunca ataca, nunca faz mal a uma criança. Nunca se falou do
ataque da ave a um menino. Ela fura somente os olhos dos adultos. Você
nunca vai ouvir dizer que ela cegou uma criança\\

Outros, porém,

esses nem dormiam mais depois que a ave veio. Os adultos. Tremiam à
noite. Tinham pulos e espantos. Tinham uma certeza: a de que a ave iam
entrar pelas suas janelas a todo instante e cegá-los, nessas casas onde
depois ficariam vagando, sem rumo, se dando contra as coisas duras

E os dias passavam\\

Havia o medo. Estava dentro. E ao redor.

Ouvi desmoronamentos.\\

É que eles não entendiam.

Um Curau não faz mal.

Se era até preciso pedir a vinda dele.

--- É preciso pedir que ele venha, isso eu dizia a todos.

Dizia a eles:

--- Todos devem pedir a vinda do Curau.

Devem pedir todas as noites e dizer como quem reza:

Vem, Curau, e me cega

Me livra das coisas que não mudam,

estes meus olhos não querem mais ver os dias assim iguais

Me faz, Curau, cair

para dentro da noite que eu sei que sou,

para que tudo mude

para que eu volte a ter gosto pela vida

Curau,

isso todos deviam dizer, de olhos fechados,

eu não vejo mais nada

Não vejo os outros homens, há uma máscara em cada rosto

Faz também esses outros se perderem na tua noite, perderem os olhos com
eu, faz esse bem a eles

Cega esta mulher que dorme ao meu lado, cega estes homens ao redor de
mim,

eles também só vêem a máscara no meu rosto e não podem ver, como eu não
posso ver neles, sob a máscara, o rosto que tive. E ainda tenho. E está
oculto. Soterrado. Um ambiente sem luz.

Vem, e te peço, não vejas a máscara no rosto de uma criança. Isso eu te
peço. Pois nelas os dias vivem

Por isso deixa em paz os olhos delas

e fura só meus olhos, Curau, entrando por esta janela com a tua luz
negra.

Eu não sei mais ver\\

Isso era o que todos deviam pedir.

Isso eu dizia a eles.

Deviam se reunir nas igrejas e nas praças para pedir. E também pedir
sozinhos, como fazem aos seus santos em voz baixa

--- Peçam a vinda da ave e esperem que ela venha, eu dizia.

E que atenda logo o pedido.\\

Veja. Eu.

Você está escutando um homem que já teve os olhos furados pelo Curau,
disse o cego ao homem.

O outro ouvia.

Agora o que eu sei é o que eu sei. Nada.

O outro ouvia e o cego disse a primeira coisa que a ave fez foi me
cegar. Acabar com os olhos que eu tinha. Gastos.

O homem em frente ao cego olhou para janela.

Mal teve forças de novo, continuava o cego, depois de comer a minha
comida, ela pulou em mim e veio esta noite em que agora estou, aqui.
Aqui é em toda parte. Nesta noite eu espero pelo que virá, virão os dias
sem nome, eu sei

Espero e sei

Espero e escuto. As vozes vêm no vento

Espero e vou tocando as coisas. E entendendo.

Passo esta mão na cara de alguém e entendo que está triste e de onde vêm
as tristezas, passo esta mesma mão na cara de um outro e entendo o seu
medo e de onde vem o Medo e para onde o Medo vai. E de que é feito o
Medo.

Então digo com voz lenta para que não se assuste

tenha paciência espere um dia a ave vai voltar\\

O homem diante do cego quis outra vez olhar para a janela.

Ela se foi, ouviu que o cego dizia.

Ele então quis saber o resto, o que tinha acontecido depois.

Um dia ela foi embora, estava dizendo o cego.

Não se sabe para onde. Deve ter voltado para o lugar de onde veio.

Estará agora num ninho cruel, num lugar oculto em alguma parte da vida,
longe ou perto de nós. Não se sabe

Agora, nesta janela espero o Dia.

Esse Dia em que retornará, diz o cego. Será numa tarde como esta talvez.
Abrindo as nuvens. Quando a ave voltar, estava dizendo. E então.
Recomeçarão os gritos. As fugas. Eu

espero aqui.\\

O homem ouviu que o cego lhe dizia, Fique, você veio de longe, e espere
a volta dela. Da ave.

O cego estava pedindo para tocar nos olhos dele. Deixou.

Venha cá, dizia o cego.

Deixe eu tocar nos seus olhos.

Os olhos.

Estes também. Os seus.

Todos assim.

Os seus são como os olhos que eu tinha. Olham para fora da vida.

Livre-se deles também.

Ah mas você tem medo. Toco. Sei.

Vai ser mais um para correr pelas ruas quando a ave voltar. Tapando os
olhos, procurando um lugar para se esconder gritando o Curau, Curau. Lá
vem ele. Como eles gritavam. Tanto. Os gritos. E mais tarde também nem
fechará mais seus olhos à noite, com medo de não ter mais olhos para
abrir de manhã. Para não ver nada.\\

Empurrado pelo cego.

O homem se afastou.

O cego estava dizendo vá embora agora. Eu já disse tudo. Agora eu quero
ficar só. Ficar sentado junto à janela. Cego.

\pagebreak

\vspace*{4cm}

Escuto.

Estou aqui. Cego.

\pagebreak

\vspace*{4cm}

Depois que o homem fora embora, o cego havia ficado só outra vez

\pagebreak

\vspace*{4cm}

E essa voz que diz no vento\\

--- Vem Curau. Vem levar os homens para os teus jardins

\pagebreak

\vspace*{4cm}

Escuto.

Estou aqui. Cego.

Aqui é em toda parte.\\

É a voz do cego, falando. Mas haverão outras, foi dito\\

Espero a volta da ave. E escuto as vozes, ele está dizendo. As vozes da
terra vêm de longe. Para ouvir as vozes da terra basta se deixar ficar,
o cego diz junto à janela.

Isto é um homem, ele diz. Um inseto me olhará sem entender.

Há todos esses rumores\\

Agora o cego está escutando.

E diz: É do fundo da cabeça que me contam estas histórias. Tudo vem no
vento também

para aqueles que insistirem em avançar nesta noite, e ela, a vida, por
sua vez também avançará. E às vezes grita. Outras vezes murmura

\pagebreak

\vspace*{4cm}

Agora

é a hora das sombras se aproximando das coisas.

Mais tarde, já ninguém poderá ver seus pés, que ficarão tropeçando nas
coisas, hesitando dentro das casas sem saber para onde levar o corpo. Os
braços também irão sumir, e se poderá estender eles para frente sem ver
o que as mãos irão tocar, e todos assim, mutilados, irão vendo com
espanto embora isso aconteça todos os fins de tarde, que desaparecem. E
também irá escurecer em todos os olhos até que não reste mais nada para
entregar à noite. Mas não é o Curau voltando, não é. É só o Curau da
vida fazendo a vida desaparecer por algum tempo para nós

A noite. Agora ela vem vindo\\

Um grito. É uma ave.

Ele diz a Jacinto que tudo em Andara está deixando de ser humano. É de
Andara que veio esse grito até a sua janela em Santa Maria.

Santa Maria do Grão à noite.

Agora o medo, o Medo vai sair aqui para fora e correrá por toda parte
levado pelo vento.\\

Na janela Jacinto espera

Vem no vento uma queda.

Está começando, diz Jacinto na janela. E espera

Há um homem na janela que escuta

e a vida fala para ele

vinda no vento, ela, a vida, para que eu que levo o fardo dessas
histórias a escute também\\

Mais tarde, ele ouvirá um ai

É a infelicidade.

Ela está se instalando.

Então Jacinto sente que tudo está começando outra vez, e que esta vai
ser mais uma noite para não dormir, e que só ele, o homem na janela, vai
ficar quieto, mudo, enquanto tudo fala e enquanto os outros

Vem no vento:

--- Lhe digo que arrancou os olhos com as mãos. Caiu, rolou no chão e
chamou por alguém. Veio a mulher. E ela então, para que não sofresse
mais, usou a corda para matá-lo

Isso diz uma voz.

E na janela o cego sabe que há uns que não resistirão até o amanhecer.

Amanhã, quando a luz voltar, será mais um dia para enterros. Mas um dia
as Asas vão voltar e isso não vai mais acontecer, diz o homem na janela.

Ele espera.

E sabe que à medida que a noite avançar,

o vento aumentará sua força até que comece a arrancar janelas, derrubará
os homens das redes, baterá na porta como aquele que quer entrar à força
enquanto do outro lado da madeira se amontoam coisas, mesas, mortos,
cadeiras, tudo para resistir ao vento. E o vento também irá para a
margem do rio e afundará os barcos amarrados na ponte e virará
candeeiros e apagará velas, e não deixa nenhum refúgio de luz para onde
se possa correr, sendo perseguidos, todos, já agora também por uma coisa
sem nome que vem de dentro de cada um, e está fora também, e está também
no sono dos que dormem e tentam escapar assim fingindo que não sabem de
nada, os adormecidos, que nem estão vivos, e entra pelos seus ouvidos, e
mesmo no fundo do sono ninguém está a salvo pois agora o vento também
fará com que sonhem que está virando tudo dentro deles, em suas cabeças,
revolvendo, fora, os cabelos, e, pálidos, eles vão acordar querendo
fugir aqui para fora para a vida, mas é aqui fora que nela um verdadeiro
inferno espera e o vento agarra as mulheres e quer porque quer levantar
as roupas delas e arrancar os meninos dos seus ventres mal eles põem a
cabeça e espiam aqui para fora. Os que tentarem nascer esta noite.
Espiam a vida que se atira, se atira e para onde,

se pergunta Jacinto em sua janela.

Ele já viveu noites como esta.

A vida poderá ser outra coisa, diz o cego. Mas só para aqueles que
souberem esperar que amanheça

Logo amanhece, pensa Jacinto em sua janela.

E espera.\\

Esta noite, outras vozes outra vez virão no vento\\

Ouve.

Elas dirão a ele:

Que agora as águas do rio enlouqueceram, estão correndo ao contrário.
Sobem para lá, apontam as vozes

E dizem, Senhora ilumina com tua faca de luz esta noite

E dizem também, O encanto O encanto. As alucinações estão chegando no
porto, trazem com elas uma esperança, venham todos

Isso dizem as vozes ao homem na janela.

E dizem mais, dizem, Um filho de deus foi derrubado pelo vento e
esfregado nas lajes da igreja quando fazia suas orações. A igrejinha
está rachando, não resistirá à esta noite

Não suporto mais esperar que amanheça,

dizem as vozes.

E o homem na janela escuta. Espera.

Talvez um deles peça agora que se sacrifique um inocente, se diz
Jacinto.

Dirá, É para nos salvar.

Eles beberão seu sangue se for derramado.

E mais tarde acenderão fogueiras na noite. Haverão de fazer uma festa. E
depois tentarão adormecer, pesados de álcool.\\

Logo amanhece.\\

Esta noite já vai acabar, murmura Jacinto.

E então vêm no vento vozes mais antigas,

falam de outro tempo de torturas.

E o vento diz: Há uns, condenados a serem puxados por quatro cavalos em
quatro direções.

Sim, diz Jacinto.

E pensa, já vai amanhecer.

E vem no vento, de outras terras, esta história: Lá, numa terra que
nunca viste, uma mulher matou outra. É por isso que agora está sendo
levada para morrer, agora é a sua vez. Vai morrer em frente à cadeira
onde estava sentada a outra mulher quando a matou.

Jacinto pode ver isso da sua janela

É a vingança. Diz o vento.

E está dizendo e sendo levada ao lugar onde matou a outra, sua mão será
cortada e jogada no fogo, para que veja. É a mão direita com que matou e
é preciso que ela a veja se transformando em cinzas

Para que veja, repete o cego na janela.

Será morta com a mesma faca, diz o vento.

O homem na janela repete é a vingança

A mesma faca, diz o vento.

E pára.\\

As vozes que vinham no vento de Santa Maria do Grão também param. E não
chega mais voz alguma até o homem, nem de Andara.

Em sua janela agora ele não ouve mais nada.

Depois, o vento está vindo outra vez. E um pássaro todo negro se quebra
na cara do homem. Ele sabe que quem jogou o pássaro foi o vento. E não
se importa.

--- Mataram Mariana, mataram Mariana, grita alguém debaixo da janela.

Na janela, o homem não acredita nessa voz.

Ele sabe que foi um pássaro que o vento jogou na sua cara. E que Mariana
não é negra. Também sabe que muitas noites ainda virão para se ter esse
medo de estar vivo sem remédio. E espera que amanheça\\

A manhã agora vem vindo.

Eu a sinto chegar.

Vem por aquele lado, entra pela janela, vai iluminar primeiro as mãos
dos homens para que eles possam achar um copo de água e afastar essa
noite. Depois, iluminará seus pés para que vivem mais um dia antes que
as sombras tornem a tocar as coisas brancas

Eu estou aqui.

Aqui é em toda parte, diz o cego na janela

\pagebreak

\clearpage
\thispagestyle{empty}

\movetoevenpage

\vspace*{4cm}

Amanhece.\\

E se a infância vier até ele agora?

A infância então vem

E uma voz vem dizer no vento:

--- A constelação do cão está latindo outra vez

\pagebreak

\vspace*{4cm}

--- Fim para a infância,

grita, mudo, o cego junto à janela.

Ele está lá. Cego.

E espera a volta da sua ave. A primeira palavra que saiu da sua boca.
Curau.

--- Temos os mortos para velar, diz Jacinto. Eu antes estive com eles.
Agora não quero mais deixar este lugar, avançar no escuro. Os mortos são
peixes e estão indo para outras águas, não se sabe para onde\\

E no vento outra voz vem dizer a ele:

\bigskip
\bigskip
\bigskip
\bigskip

aquela voz vinha do outro lado da sala.

Era como um trabalho de inseto.

Depois, parou de falar

Uns grandes outros pequenos iam se aproximando do homem deitado os
vultos para mais um último beijo. Ouviu as vozes infantis. Lá fora uma
ave da noite cantou. Uma brisa agora está entrando pela janela e os seus
cabelos fingiam ainda um resto de vida para os outros esses que estão ao
seu redor

Na sala alguém murmura

--- Então passamos a fronteira, e não se escutou o fim do que dizia

Durante toda a noite estivemos velando na casa.

E depois o morto não estava mais lá.

Mas ainda o beijavam sem saber que já havia partido com os lábios frios
da madrugada. Os parentes

--- Nunca fomos tão lentos como nesse instante, voltava a voz do outro
lado da sala. Na minha terra, há pescadores que amarram os mortos pelos
pés para que não se levantem e voltem para o mar, diz aquela voz

\pagebreak

\vspace*{4cm}

Eu estou aqui.

É o cego.

Isto é um homem, ele diz. E isto também quer entender

\pagebreak

\vspace*{4cm}

--- A constelação do cão está latindo\\

--- Nunca fomos tão lentos como nesse instante\\

As vozes ainda peregrinam pelo ar dessa manhã que nasce

Enquanto em suas casas os homens dormem, curando em sonhos as feridas da
noite que passou\\

E até Jacinto na janela uma nova voz está vindo no vento para dizer:

--- Andara. Lá as ruas estão sempre vazias, e Aquilo, a floresta, avança,
vem cada vez mais para perto de nós\\

Jacinto tem fome.

Isto ainda é um homem, ele diz.\\

Andara é a África que temos dentro

\pagebreak

\vspace*{4cm}

E se viesse agora uma outra voz?

E essa voz está vindo. E diz:

--- A velha. Ela mastiga lentamente. Ela também é um animal à noite,

como me disse que era aquele tio Irido que foi para Andara e nunca mais
voltou. Não vejo os seus olhos, ela não tem olhos para se ver

Jacinto está ouvindo e pensa, Agora é uma irmã minha quem me fala\\

E a voz:

--- Agora ela anda no jardim. A velha. Um inseto a vê. Se olham. Não há o
que entender

\pagebreak

\vspace*{4cm}

Ventos falantes, ventos falantes\\

E essa voz que agora vem contar ao cego na janela

isso:

--- Um homem está ali e olha o rosto de uma mulher que dorme e a
adormecida vai desaparecendo em sua noite. Mas no rosto dela as coisas
que hoje viveu ainda vivem,

e nessa máscara que foi seu rosto as coisas dizem

vivi mais um dia com suas dores e a alegria que cabe a cada um nesta
terra que os homens chamam minha casa.

Depois, no rosto da adormecida até a máscara das coisas vai
desaparecendo. E então para olhar já não há mais nada. O vazio.

--- Por onde ela andará, pergunta o homem olhando esse vazio. Que começa a
se tornar uma paisagem,

pois outras coisas agora, no rosto dela, estão começando a vir à tona.

A mulher se agita. Afasta algo. Não. Diz sem voz. Não.

E o homem sabe que ela sonha com alguma coisa que não quer, que recusa.

E o que ela não quer é essa paisagem que surge, onde o homem vê um
menino e o menino está vendo as figuras que gostava de achar nos muros,
as manchas,

para nelas ver bichos, armas, um pouco de sangue correndo no início,
depois mais sangue ainda, correndo mais forte. É o homem e sua fome,
quer um animal para a sua mesa, dizia o menino olhando o muro. A mancha.
Era só deixar os olhos irem e se deixar ir, junto, para ver

homens atrás, perseguindo, e o animal fugindo sem ter para onde, preso
na mancha, no muro. A armadilha

\pagebreak

\vspace*{4cm}

--- A fome vem na hora inicial da vida, quando é manhã e os olhos se abrem\\

diz uma voz e a manhã ainda nasce.

Esta manhã, o tempo está estagnado

E nas suas casas os homem dormem\\

Então, mais nada irá acontecer além dessas vozes contando histórias para
o cego na janela?

Ainda não sabemos. Inquietos,

os viciados da continuidade.

\pagebreak

\vspace*{4cm}

--- Vendo a armadilha o homem foi indo na memória,

conta a voz que fala da adormecida.

Mas agora ele se agita.

É que para o rosto dela um outro homem se inclina.

--- Quem será, se pergunta o homem. E como tudo está escuro e não pode ver
o rosto do outro com quem ela sonha, pensa que o outro bem pode ser ele,
quem sabe

Uma noite agora está vindo sobre a região onde a mulher sonha, num fim
de tarde, e lá ela já não se recusa, não afasta a sombra.

--- Lá, onde ela está, onde eles estão, ninguém mais está, diz o homem ao
lado da cama.

Olha ainda e está vendo a mulher e o outro, ele talvez, se afastando por
entre árvores, num jardim. Ela tem a cabeça apoiada num ombro escuro, a
última coisa que ele vê

\pagebreak

\vspace*{4cm}

Nova voz está chegando, entra pela janela. O cego ouve

Isto não acaba. Não acaba?

Mas uma vez.

A infância não tem fim\\

--- Em Santa Maria do Grão, conta a voz

quem passa por aquela rua não sabe porque olhando a casa vê uma gaiola.
Quem passa olha a casa. Para não se atormentar mais com isso, quer
encontrar uma explicação, e pára, olha bem e depois se afasta e diz
aquilo é só uma ilusão, vem das grades que a casa tem

E a impressão vai junto, acompanha quem viu a casa durante dias

Mal se distrai, lá está de volta a casa. Vem na memória, e onde foi que
eu vi essa casa, quem passou se pergunta.

É assim.\\

A infância não tem fim?\\

Na casa mora uma mulher. Já não há homem ali, mas ficou uma menina, a
filha, e come tudo o que lhe cai do alto.

Por toda a casa se move a mãe grasnadora, arrumando coisas, varre o chão
e tem aquela pena negra que sai da abertura do vestido, atrás. Foi com
ela que agora a mulher acabou de varrer o chão.

No mais, é uma mulher como as outras. Como as outras, pode ser vista por
toda parte.

Foi assim que o homem um dia a viu.

O homem ria, se ficava por baixo da mulher sentindo aquela pena no seu
corpo agora morto.

Antes, a filha também ria. Mas a mulher sabe usar a pena com violência e
a menina nunca mais riu.

Com a pena a mulher se abana nos dias quentes, com ela abana o fogo que
quer morrer sob uma panela. Para muitas coisas a mulher usa a pena. E
agora ela está de costas fingindo que não viu mas de repente se volta e
há uma mosca a menos nesta vida

Com a pena a mulher também persegue insetos. E sabe achá-los nos buracos
e frestas e não há nenhuma frestinha para onde possam fugir

Os insetos.

Bela, a pena brilha.

O sol vem iluminá-la. Negra

À noite, porém, não se sabe para onde a mulher vai. Todas as noites sai.
Já está amanhecendo quando volta, traz comida, a filha come o que lhe
cai do alto

É uma ave. E caça ao luar.

É isso, a menina se diz, só, em casa

enquanto a mãe voa pelas ruas à noite e ela fica, fecha os olhos e assim
está vendo a mulher voar, depois vem um sono calmo e ela vai adormecendo
e dizendo, Tenho esta mãe, tenho esta mãe. Devo acordar amanhã

O rosto da mulher tem começado a anunciar a morte, porém.

--- Ela vai se cansando de estar viva, diz a menina no quarto para o
espelho. Ela vai se cansando de estar viva, disse a menina um dia
olhando o rosto da mãe adormecida, uma tarde

Ela viu que a pena ia se tornando sem luz.

E teve medo. Se ela acaba, foi dizer outra vez ao espelho.

Havia vindo o medo.

O medo.

Mas agora a menina não tem mais medo.

O medo foi embora.

Saiu daquela casa.

Foi na hora do jantar, hoje

Quando ela se sentou na mesa, havia alguma coisa embaixo dela, entre ela
e a cadeira.

Levantou, foi olhar no espelho e retorcendo-se toda para trás pode ver.
Era um início. Nela também nascia

Voltou para a mesa. E comeu com uma fome que também voltava. Olhava a
mãe sentada como sempre de lado e sentou também de lado. A herdeira. E
disse baixinho eu e ria.\\

Agora, passa alguém em frente à casa, quem passa olha a casa

\pagebreak

\vspace*{4cm}

--- Se vocês querem, falem mais da dor\\

É o cego quem diz isso. Fala com as vozes que tem escutado. As vozes.

Elas nunca deixando de vir contar a ele as suas histórias, no vento,

os dias não passavam\\

As vozes. Estas vozes, estas histórias

Por muito tempo ainda elas vindo, para dizer a ele que a vida, lá fora,
ainda é um lugar de rumores e um Não submete tudo.

Nele, homens se buscam sem se ver, diz o cego. Mas um dia a ave vai
voltar

Ele ri.

Na sala onde ele está há um espelho, mas o espelho não mostra esse riso

\pagebreak

\clearpage
\thispagestyle{empty}

\movetooddpage

\vspace*{4cm}

Ah você bateu na minha porta. Você voltou.

É o homem quem entra agora.

Ele voltou.

Veio ver como vou passando.

Eu não passo. Os dias não passam. Nada disso muda enquanto o Curau não
volta, diz o cego ao homem.

Eu tenho esta janela, e nela o tempo está parado. Nela espero a ave.
Você lembra, eu lhe contei tudo, como começou,

diz o cego ao homem e o homem estava de volta.

Havia entrado como da primeira vez.

E agora estava outra vez em frente ao cego.\\

--- Sim, tenho ouvido o vento, lhe diz Jacinto. As histórias que ele vem
me contar. Eu escuto o vento sempre. Não, você não poderá ouvir as
histórias também. Não. Você ainda tem esses olhos. Quer tê-los.

Se quiser ouvir outra voz. A minha.\\

Agora o vento parou. E o cego tem aquele homem de volta. E espera a
volta da ave. Quer ouvir asas no céu. As Asas.\\

--- Quando o vento pára, eu lembro outras histórias, ele diz. Quer ouvir
uma delas?

Quer. Já contei a história do Curau, agora você quer ouvir mais, sempre
mais? Deixe que eu fale então, enquanto espero. E não pense que a
história do Curau é só uma história. Eu estou aqui, só. Aqui é em toda
parte.\\

O homem escuta.

--- Olhe, diz o cego, o tempo é isso e então eu já não lembro como acabou.

Agora é a história de Sumiro que ele irá contar ao homem. Não pense que
a história do Curau é só uma história, ele repete.

Talvez invente um final para essa história agora, diz. E o homem ouve.
Talvez recorde à medida que for inventando. Não se sabe nunca. Há a
memória, esta coisa à noite. Não lembro mais também os nomes. De
qualquer maneira, nela, Imaginação, há coisas que crescem, fogos
enormes, e há o que se apaga. Ou vem mudado de volta, na volta, quando
se quer lembrar. Na Memória.

Por esses caminhos eu vou. Imagino. Lembro.

Eles são dois. Se misturam.

Disse que não lembro mais os nomes?

Não minto. Só o nome dele eu não esqueci. Sumiro. Era Sumiro. O
inesquecível.\\

Foi num fim de tarde aquilo.

Eles vieram e o levaram, arrastado.

Naquele tempo eu ainda tinha olhos para ver, diz o cego. Foi antes do
Curau e com eles eu não via nada, entenda.

Veja, a terra por onde o arrastavam já ia escrevendo no corpo dele esta
história que agora eu lhe conto

escrevendo ela também no seu ouvido,

e isso se dando enquanto o levavam, era uma antecipação de toda a dor
que ainda viria mais tarde para ele,

como ainda virá mais tarde para você que agora ouve a história a sua
parte de dor

Eles, porém, não iam levando o homem arrastado porque resistisse, não. É
que queriam levar ele assim. Veja.

No lugar que escolheram para fazer o que fizeram, havia um que esperava,
vigiando, um par de olhos vivos, as armas prontas para usar contra ai de
quem tentasse impedir o que eles iam fazer. Veja agora isso, assim
aquele era um Lugar fora da terra, ficava por cima do inferno. Foi lá
que aquilo se deu. Naquela hora aves começaram a voar baixo, anunciando
vem morte vem a morte, voavam e a gente entendia aquelas asas.

--- Tome um café, disse o cego ao homem.

Ah este frio, disse o homem.

É a noite, disse o cego.

Não. Eu não quero. Esta noite, mais tarde eu quero sonhar depois de ter
lhe contado tudo. Se as vozes deixarem

Ouça.

Então, eu ia dizendo: eles arrastaram aquilo como se não fosse um homem.
Sumiro. Mas era um homem.

Ainda que vivesse daquele jeito sempre, sempre tinha uns olhos baixos
procurando no chão e não se sabia o que procurava. Nunca se viu ele
levantar os olhos. Eles, os olhos daquele homem, se davam à terra. Só a
ela. Dias e dias, ele vivia assim. Era por dentro que vivia. Havia quem
dissesse que nem era mais um homem e que não estava mais entre nós, os
outros. Para mim, era.

Não sei que outra coisa podia ser aquilo, que às vezes até falava embora
não se entendesse o que dizia

Se havia perdido alguma coisa e vivia procurando, esse era um segredo
dele

Cada um terá o seu.

Não teria sido isso, nele, um segredo que não se podia medir, me
pergunto, o que eles não perdoaram,

e por isso o arrastaram naquela tarde para mostrar que todos, todos
somos iguais, homenzinhos, mistérios que têm que ser revelados custe o
que custar para que tudo, o humano, fique sob luz que faca possa contar,
que se possa negar quando for preciso fazer mais um morto?

Então.

Levado arrastado, o que era para rir era isto. Ele ia aproveitando para
continuar procurando, quieto, olhava o chão passando rente aos seus
olhos, queria usar até o último momento aqueles olhos, tinha ainda
esperança de achar o que havia perdido antes do fim. E não dizia nada.\\

Havia uma árvore no lugar para onde o levaram.

Veja isso agora: eles também tinham um segredo.

Pois fizeram aparecer lá uma mulher, taparam os olhos dela e a mandaram
amaldiçoar a árvore para que não desse mais frutos. Os frutos.

E depois, pregaram o homem na árvore.

Sumiro, o homem, ficou pregado lá

Então veio a hora da facas,

a primeira hora das facas

Haveria outras mais.

Foi nessa primeira hora que eles então tiraram o sexo dele.

Um homem pode deixar de gritar sua dor?

Então ele gritou. Ainda se ouve esse grito quando se passa por aquele
lugar à mesma hora, nos fins de tarde. Dizem. Eu não voltei mais lá, não
sei. Ele ficou pregado na árvore. E depois disso não veio mais nenhum
som dele.

Nada mudava ali.

Haviam levado Sumiro arrastado e agora ele estava lá, na árvore. Espere.
Tento lembrar

Lembro. O céu mostrava nuvens paradas. Não eram nuvens leves. Nelas,
havia um vermelho de sangue que não anda. Pesado. A gente ali sabia que
uma chuva ia cair, mas que chuva era aquela, parada, no alto, isso não
se sabia

Tínhamos o céu por cima e embaixo éramos só homens, mulheres e havia
também umas crianças ao redor da árvore.

Foi outra vez a hora das facas

Dos lados do horizonte veio vindo um barulhão.

Parecia que a chuva ia cair. Mas não caiu. Não caía. E essa foi a hora
em que eles tiraram os olhos do homem.

É que ele não havia parado de olhar o chão, de procurar, e era preciso
que isso acabasse. Não teria adiantado nada fazer aquilo com ele se não
parava de procurar.

Nós olhávamos. Não fazíamos nada.

As testemunhas.

Eles queriam que nem os olhos do homem, sozinhos, ainda procurassem,
rolando na terra, e, vimos, um deles tirou uma bolsinha, juntou os olhos
de Sumiro e guardou nela, fechados,

no escuro.

Como é que ele ia procurar agora?

Eu não sabia.

É que naquele tempo eu ainda olhava, tinha meus olhos e nada via. Via o
homem sem olhos na árvore e me perguntava como é que agora ele vai
procurar, sem entender nada. Mas ele ainda procurava, agora eu sei,
mesmo que estivesse tão no escuro quanto os seus olhos guardados na
bolsa do outro.\\

--- Não enjoe. Parece que é com os olhos da gente, não é?

Mas eram os olhos dele, não os seus, lembre-se

O homem olhou pela janela e disse:

--- Só pode ser artista quem tiver uma visão original do infinito.

Foi Schlegel quem disse

--- Não sei, disse o cego. Não sei nada. Eu estou aqui. Cego.

Só posso lhe dizer isso que lhe digo, estamos todos juntos nesta vida,

homens e deuses,

esses que nos fizeram de terra

Os da Água, os do Ar e os do Fogo

E os da Ausência.

Agora eu sei isso\\

No vento estava passando outra vez a voz que dizia\\

--- Vem Curau. Vem levar os homens para os teus jardins\\

E o cego disse:

Continuo.

Mais tarde já tinham ele nu, haviam arrancado as roupas, queimado. As
mulheres não viraram as caras, aquilo era uma outra vergonha

Lhe cortaram as mãos, na terceira hora das facas. E isso aliviou nele
uma dor, uma dor apagando outra dor, veja: a dor que ele sentia antes,

a dos seus olhos arrancados

Mas mal eles viram a nova dor nascendo e a dor mais antiga sumindo no
rosto de Sumiro, fizeram uma outra dor. Meteram um prego grande, Negro,
na sua boca. E assim ele não podia mais falar, mesmo não tendo dito nada
durante tudo isso e não fosse mesmo dizer nada enquanto aquilo durasse.
Isso eu sabia\\

E durou.

Olhando, mais tarde, não se podia saber onde ele estava acabando e onde
começava a árvore.

Isso foi assim.

É que eles tinham aberto o tronco, na quarta hora das facas, e metido o
homem dentro da árvore,

uma parte aparecendo, a outra oculta.

Queriam desse modo fazer dele cada vez menos um homem. Tirar dele todo
mistério. Fazer dele árvore.

Cada um entenderá como quiser. Sim. É. Cada um é um outro.

Mas veja, era o que queriam.

É como eu entendo.

Eu estou aqui, cego. Aqui é em toda parte

As vozes da terra vêm de longe

Um inseto me olhara

O que há para ver?\\

Ficamos lá. Eu olhava. Os outros também olhavam.

E lá se iam os dias não sei para onde

Estou lá. Olho. Não fazíamos nada.

Tinham o homem na árvore.

Decidiram dar comida a ele para que durasse mais. E aquele homem então,
ex-nosso vizinho, Sumiro, aceitou a comida deles.

As nuvens continuavam no alto, a chuva não caía, aquela chuva, e ele não
morria, aquele homem na árvore, nem deixava de ser homem nem virava
árvore de uma vez. Nem tinha nada para nos dizer. Olhávamos\\

No dia em que quis pôr para fora a comida, não pôde. Tinham fechado a
saída.

Teve que sofrer também as dores comuns.

Então parecia ainda mais um homem.

Isso eles notaram. Viram que não deviam dar comida a ele, e não deram
mais

Estava ali, perto. O homem na árvore. E no entanto bem longe daquele
lugar alguma coisa mudava nele, ficava nos seus sonhos onde nada daquilo
estava acontecendo

Nele, nos sonhos que tinha, às vezes deviam entrar uma ou outra as vozes
deles, de guarda junto à árvore

Mas isso acabou quando veio outra hora das facas e furaram os ouvidos. E
o deixaram inteiramente só, num branco,

e se fez um silêncio como não se sabia que existisse.

E assim ele já não podia mais ouvir quando novas horas vieram para as
facas, em ondas, umas após outras, soando,

e arrancaram suas pernas para que não andasse humano nunca mais,

e isso, acho, já lhe acontecia como se fosse a um outro, numa outra
história. E nessa história eles queriam talvez arrancar dele a sua alma,
a sua ave. Para trancá-la numa caixa de madeira. Madeira da árvore.
Árvore em que seu corpo desaparecia como coisa humana. Mas na caixa
aquela alma não ia parar de voar, voaria como antes no corpo do homem. É
que ele e a árvore já eram então uma só coisa.

No alto, o céu se mexeu. Era o sangue dele que finalmente ia cair?

Era aquela chuva. Que ameaçava cair e não caía. Não caía. Não caiu nem
cairá.

As horas das facas iam e vinham\\

Tiraram a cabeça dele para jogar com ela. E ele entendeu o vazio

Já havia entendido a escuridão, não era?

Já havia entendido o silêncio

E agora entendia o vazio.

Os que faziam guarda ao redor da árvore se divertiram, riram,

é que o homem é sempre uma criança? Me pergunto. Riam e atiravam a
cabeça uns para os outros. Era para passar o tempo, ali. Jogavam ela de
um lado para outro, a bola mágica, voava entre aqueles homens,

e isso sem que ele se sentisse humilhado com nada.

Mas dias nasceram. Morreram\\

Tome mais um café. Tome. Já vai acabar. É preciso que acabe. Agora
estamos indo para a parte onde ela é uma outra, a vida\\

Ouça.

Eles trouxeram a mulher que havia feito Sumiro. Puseram ela na frente da
árvore. É que então tinha descido sobre aqueles homens um desespero. O
de ter perdido Sumiro nesta vida. Queriam trazê-lo de volta. Onde
estaria ele agora?

Se reuniram então e falaram. Embaixo de uma outra árvore. Afastados.

Depois voltaram.

Vamos fazer assim, se disseram. E fizeram. Para que Sumiro voltasse,
foram buscar onde achavam que ainda estava, e arrancaram a roupa da
mulher, e procuraram no corpo dela

E tudo diante da árvore para que ele visse sem os olhos.

Trouxeram também um menino e diziam, É filho.

Procuraram sinais de Sumiro nele.

E trouxeram uns outros, assustados, e diziam, São irmãos.

Fizeram com todos a mesma coisa. E outras. As sem nome.

Sem nome foi também o que acabaram fazendo com ele.

Como não queriam ficar sem Sumiro,

se transformaram, ali diante de nós, em homens-areias, secos, e beberam
o sangue dele,

se transformaram em homens-cachorros, tinham raiva, e roeram os ossos,

e na volta aos homens que eram outra vez, trabalharam, trabalharam,
organizados, exatos. E vimos que haviam feito com a pele de Sumiro uma
tecida rede, e secaram ela sob um sol violento que surgiu no céu por
entre as nuvens daquela chuva que não caía não caía não caiu nem cairá e
se mostrou para os nossos olhos. As testemunhas.\\

Foi assim.

E uma roda girou. Lentamente

Primeiro para a esquerda.

Depois para a direita.

Depois ficou indo uma vez para a esquerda outra vez para a direita, como
uma dúvida. E a cada volta que ela não completava, víamos

Nasciam as frutas na árvore.

Então dançamos. As testemunhas.

E cantávamos. Batíamos palmas e bebíamos bebida amarga, rodando em torno
dele enfim inquieto, ou sereno, ou desesperado e sem saber por quê

\pagebreak

\vspace*{4cm}

Agora vá embora,

disse o cego ao homem depois. E o homem foi

Eu quero ficar só outra vez.

Ficar aqui. Ele disse cego.

\pagebreak

\vspace*{4cm}

A memória.

É ela outra vez. Mas é a memória de um outro nesta voz que vem falar a
Jacinto,\\

e vem no vento\\

--- Quando nosso irmão nasceu o relógio da sala parou. Eu lembro

Aquela ficou sendo a hora marcada.

Eu lembro

Nosso pai disse, Dele, deste filho que nasce de mim um dia virá a
alegria. Disse isso e empalidecendo, de repente mais velho, se afastou,
se arrastou para o seu quarto e tem se mantido mudo desde então,

uma outra máquina quebrada nesta casa.

Aquele pai.

Era um menino. Não chorava. Não ria.

Só olhava para nós, distante. O Distante. Estava lá entre nós e não
estava. Era se como continuasse no lugar de onde veio

Ele veio da nossa mãe, diz um dos irmãos que eu tenho.

Mas eu sei que veio de mais longe quando vou vê-lo.

Ainda é o nosso recém-nascido. O Eterno. Nunca cresce. E os anos passam.
Esperamos

Nosso pai disse aquilo. E agora estamos à espera da alegria.

E esta espera vai nos tornando assustados. Tudo é um espanto. É essa
espera

Um rumor, um vidro se quebra

Os desastres, os dias. Não é isso a alegria.

Fomos ver o copo quebrado na cozinha e não era ela.

E então, se nada aconteceu, temos procurado por toda a casa um indício
da vinda dela.

Mas nunca estamos de acordo.

Virá num dia dos mortos, diz minha mãe. A alegria.

Uma irmã quer que venha num dia de festa.

Temos esse pai e ele procura nos bolsos das roupas, revira tudo, quer
achar o outro lado das coisas, uma carta perdida, uma nome anotado que
não lembra onde guardou. E seus dedos tremem. Tem medo de tocar de
repente a coisa e isso não se dará num dia como os outros se ela vier,
quando vier e um de nós a achar.

Eu sei.

Nossa alegria virá quando ele começar a crescer

Ele fica lá no quarto. E não cresce. Esperamos.

Rondando o lugar onde ele está, deitado, está no fundo de um poço de
onde queremos tirar alguma coisa

Um de nós tenta adivinhar. Faz jogos. Nada. E se impacienta.

Às vezes um outro chora.

Minha mãe grita.

Na sala, o relógio ainda está parado. É o primeiro movimento dele que eu
espero. Será quando os ponteiros voltarem a andar que terei o aviso. Ela
vem. A alegria

o menino continua no lugar de onde veio. E por isso eu fico aqui também
parada, na sala. E olho o relógio.

No quarto o nosso recém-nascido, eterno, ainda espera.

Ele é quem sabe quando virá. A Alegria.

Só espero que ela venha antes da nossa morte

\pagebreak

\vspace*{4cm}

Lá fora agora é noite.

--- Esta noite não passa, diz o cego na janela.

Quando é noite, lá fora Jacinto acha que está ouvindo asas vermelhas
vindo no céu. Ainda agora ele ouviu de novo. As asas

Elas vêm

Elas estão vindo

Não era o Curau de volta, porém.

E ele espera. Essa espera. E os anos passam

\pagebreak

\clearpage
\thispagestyle{empty}

\movetooddpage

\vspace*{4cm}

Escuto.

É do fundo da cabeça. E vem no vento:\\

Ontem voltamos à ponte.

Nela a madeira está partindo da sua forma de tábuas, estou partindo
desta forma humana que me deram, a madeira diz\\

Fica cada vez mais velha esta ponte. Ela diz estou caindo aos pedaços,

e quando passamos sobre ela um de nós também cai na água, lá embaixo.
Nem todos voltam à tona.

--- Há um mal com olhos de criança e dentes lá embaixo, dizemos uns aos
outros e rimos. Mas ficamos na ponte.

Há um mal lá embaixo, repetimos. E rimos.

E fazemos a ponte tremer pulando sobre ela.

Quando chegamos à ponte já está anoitecendo.\\

--- Eu olho esta ponte e isso me deixa triste, diz o velho.

Ele sempre vem conosco.

Eu vejo um de vocês cair lá embaixo e não voltar,

ele diz. E isso me deixa triste.

E se às vezes vejo um outro, os pulos de alegria que dá por toda parte,
fico alegre também.

Vejam, não somos mais que isso, e isso vai sendo mudado pela vida.

Tudo vem de fora e entra.

E já não se é o mesmo, mais tarde.

E há também as coisas que saem de mim e entram na vida.

E seguimos em frente.

É assim.

Vivemos. É talvez o pior.\\

Quando o velho fala, sem sempre se entenderá o que ele diz.

Escutamos. Ele falava. Depois se afastou de nós.

Fica lá sentado na margem, só, e olha a água passar sob a ponte.\\

É quando a tarde está acabando que vamos à ponte. Reunidos numa rua da
cidade, de repente andamos, para a ponte, para a ponte

E uns já desaparecem pelo caminho, talvez tenham ido para as suas casas
talvez não não se sabe\\

outros esperam para cair da ponte quando chegarmos, que é uma outra
maneira de sumir.

Nem tudo porém são esses desaparecimentos.

Acontece também de alguns desaparecerem só em parte. Um acidente. E
temos um mutilado para rir, terá de andar agora numa perna só pela vida.
Isto pula. Já não é um homem inteiro. Nós somos assim. Nos fazemos aos
pedaços. Ficamos debruçados na ponte olhando a água escurecer\\

Hoje, até agora ainda nada aconteceu. Esperamos que a noite esteja aqui
plenamente

Então aí vai um de nós.

Entra na água agora.

Vemos ele entrar lembrando como outros antes também entraram na água
escura pela última vez

E eu digo:

--- Pedras. As pedras.

E vamos pegar as pedras. Elas estão por aí, pesam pela vida toda. Mas
ficarão leves no ar

E dou o exemplo, no que os outros já me imitam e o que entrou na água
sai correndo, foge, tem sangue na cabeça, vai ficar sentado também lá na
margem ao lado do velho. Há um outro lá. Olhamos. Agora na margem estão
aqueles dois.

Estamos nesta ponte.

Esperamos. E vemos lá na margem aqueles outros.

Estão sós. Lado a lado. Olham agora a água passar, escura

Na margem o atingido pelas pedras chora. Um apedrejado

Nós somos assim, não há nada a fazer\\

Então: Está para acontecer, diz um de nós agora.

Sim

Isso eu senti neste instante também.

Lá na margem, aquele que apedrejamos pára de chorar. Levantou os olhos.
Vê ao seu redor as árvores paradas, não há vento. E ele corre, vem para
estar novamente junto de nós, em cima da ponte, nos apertamos uns contra
os outros e esperamos o que virá. Mas não saímos da ponte. E só o velho
fica lá na margem olhando a água que passa

Nunca sabemos com qual de nós será.

Depois estaremos prontos para voltar amanhã, e nos outros dias da nossa
vida também.

E para ir embora daqui.

E já está acontecendo

Um de nós não tem mais onde pisar, onde manter a vida à tona, a madeira
se quebra embaixo dele, nada faremos para agarrá-lo. E as águas passam
sob a ponte. Somos assim. Não há nada a fazer. Desta vez era ele então

Já se foi. Um de nós.

Olhamos. E a água está calma.

E voltamos a nos afastar uns dos outros. O amontoado se desfaz. E já
somos cada um por si nesta vida que alguns chamam a ponte. Outros, a
água. Outros, um afundamento.

Amanhã é dia de vir outra vez, no fim da tarde.

Na margem o velho se levanta.

Voltamos.

Ninguém diz nada. Voltamos sem falar, dentro da noite, e nos despedimos
nas esquinas, para casa. Para casa. E sumimos nas ruas. Não há nada para
lamentar

\pagebreak

\clearpage
\thispagestyle{empty}

\movetooddpage

\vspace*{4cm}

O vento com a história de Sumiro que o cego contou já deu a volta a toda
a terra

e agora está passando outra vez pela janela de Jacinto\\

--- Eles vieram e o levaram, arrastado. Veja, a terra por onde o
arrastavam já ia escrevendo no corpo dele esta história que agora eu lhe
conto escrevendo ela também no seu ouvido

Diz o vento, só uma rajada dele que vem e passa pela janela de Jacinto
trazendo um pedaço da história de Sumiro, que permanece no ar depois que
o cego a contou.

E já passou. E daquele pedaço de voz de vento não fica mais nada\\

Uma outra voz, depois, está vindo dizer:

Do lugar onde estou, só posso ver o horizonte e um rio. Esta casa eu
mesmo fiz

Há um homem no rio. Ele me vê. Está vindo para cá. Conversamos. Me conta
a sua vida, e depois este homem está me dizendo para ir com ele.

--- Este rio vai para lá, aponta.

Diz:

--- Vamos.

Eu sei que esta ilha não existe, mas não devo ir. Não vou.

E quando ele rema outra vez para longe, desvio os olhos, tenho medo de
ver como o horizonte vai devorá-lo. É assim que a floresta agora também
está se fechando sobre mim. Aos poucos. Logo a noite virá a esta ilha. E
há animais que só então irão acordar. Eu sei que esta ilha não existe.

Do lugar onde estou só posso ver o horizonte e o rio. Vejo um homem lá.
Está vindo para cá este outro, ele também já me viu\\

Sim, Andara é mesmo a nossa África, da minha janela, cego, eu sei, diz
Jacinto.

E quero ouvir o resto.

Escuto.

O resto\\

Escuto.\\

Saindo da floresta, vi a casa.

Aquilo, a casa.

Um animal astucioso que vai abrir uma porta e olha para dentro de uma
casa com olhos selvagens como eu olhei não se encontra entre os homens
todos os dias

Entre os homens a minha lei é um pé na frente e só bem mais tarde o
outro guardado atrás avança. Não me arrisco. Fico pronto para escapar.
Eu sou assim.

Foi desse modo que então abri a porta e entrei

Ismael não estava em casa. E tive todos os gestos que quis para armar a
armadilha. Depois, saí. Fiquei do lado de fora

Atrás de uma árvore, eu queria estar lá para ver tudo. Oculto. E agora
espero a volta de Ismael.

Mas ele me vê quando chega.

Tem bons olhos esse Ismael, digo, mas não verá a armadilha. Tive todos
os gestos que quis

--- Não vejo mais rancor nos teus olhos, ah

me diz ele

Entra.

E entramos.

Ismael vai na frente. Esta casa é sua

--- É tua também, quer começar a me dizer

mas logo um estalar vem se misturar à sua voz e a interrompe, estala
outra vez e então ah tenho Ismael no ar. Usei cordas fortes. Grossas.
Suspenso assim ele é uma ave sem asas, muda. O espanto.

Ismael, Ismael, grito. Agora tu estás acabado.

E voava pela casa ao redor dele. Batia palmas. Ri e dancei, assim,
esquecido, me arriscava, sem Ismael no meu caminho posso deixar meus pés
irem para onde quiserem, dizia,

e também o resto da armadilha entra em ação, o resto que eu armara para
ele escapar talvez do primeiro mas receber um segundo ataque. Se Ismael
tivesse asas suas asas sangrariam,

e a armadilha me levou também para balançar no ar ao seu lado. Pendurado
junto com Ismael, agora somos duas aves. Isso não devia ser assim, digo

Não devia.

E Ismael agora está dizendo, É preciso que tu acordes. Escuto a sua voz,
longe

Tudo isso, vê, faz mal a nós dois, ele diz

--- Sim, respondi.

E acordei.

Mas agora eu esfrego os olhos sentado na rede e lembro o sonho, e este
sim quer se transformar em não. E eu quero é sonhar outra vez\\

--- E por isso o arrastaram naquela tarde para mostrar que todos, todos
somos iguais, homenzinhos, mistérios que têm que ser revelados custe o
que custar

É o vento. Nova rajada dele está passando pela janela de Jacinto outra
vez.

Depois, cessa

E cego se diz reconheço essa voz.

A história de Sumiro não quer se extinguir no ar?\\

Vem no vento. Escuto. Cego\\

Quando se está com medo dos homens é preciso ir à casa de um amigo

Por isso, esta manhã procurei Fabiano.

Andamos no seu jardim.

Ele me mostrava seus animais, as sombras que seus corpos faziam, tudo
isso esteve lá em torno de nós

Na volta para esta casa onde me oculto, oculto, mais tarde eu já não
lembrava o que fizemos juntos.

O que se esconde da luz dos dias?

E o que se esconde na luz dos dias.

À noite, vejo mãos sujas de sangue. E revejo tudo num sonho

Fabiano me deixando matar todos aqueles animais. E ele me dizia, Quando
o teu medo voltar, também podes voltar. Eles renascerão\\

Jacinto escuta.

É mais uma voz. É do fundo da vida:

--- Hoje eu vou tentar outra vez

Lentamente, terrivelmente, desta vez eu vou embora daqui

--- Eu vou embora,

grito isso com o meu grito mais vermelho e ele não cessa nem quando dou
com a boca torcida por um riso no espelho onde me vejo pendurado para
não me sentir só nestes dias parados.

É as ruas o que eu quero E desde já estremeço ao me ver, antecipando, e
as coisas que irão me acontecer quando estiver nelas. As ruins e as
outras. Todas elas. E não me falem do medo

Repito. Eu vou embora daqui.

E já me apresso.

Mesmo que haja um risco, o de sair pelo fundo da casa e só chegar ao
quintal

E ficar preso entre os muros, se perder a direção da porta da rua.

Para isso, devo ter uns olhinhos vivos. E para que não aconteça outra
vez, manter a direção. E ir daqui. Agora. Os pés já se movem. Estou
indo. Para as ruas. As ruas. Mas a porta da rua só parece me levar de
volta para o quarto. E até mesmo para o mais fundo dele e sob os olhos
daquele que não me perde de vista e ri, no espelho, mais ainda, para
baixo da cama. Onde me meto e estou bem agora.

Amanhã vou tentar outra vez\\

--- Nós olhávamos. Não fazíamos nada. As testemunhas, diz o vento

--- E cantávamos. Batíamos palmas, diz o vento

--- E bebíamos bebida amarga, diz o vento\\

passando novamente pela janela de Jacinto com a história de Sumiro.

E esta última rajada que agora passa está indo outra vez para longe.

O vento vai dar de novo a volta à terra inteira

O vento está indo contar a história de Sumiro a outros homens

\pagebreak

\clearpage
\thispagestyle{empty}

\movetooddpage

\vspace*{4cm}

De uma ponte

De uma ilha

Me falavam de uma ilha que não existe, se diz Jacinto. Quanto tempo já
se passou desde então? Os anos\\

Temos Jacinto, e Jacinto é o homem na janela

\pagebreak

\vspace*{4cm}

Há um outro tempo para a infância agora.

Foi por aí, a infância, esse tempo de espantos por toda parte, que isto
começou. A vida

Se diz Jacinto na janela\\

E uma nova voz está vindo no vento. É a voz de Jacinto um menino e nessa
voz em que o vento traz a infância, Jacinto vai voltar\\

Lembro Andara como era antes, diz a voz.

Ah a infância, se diz Jacinto.\\

Foi em Andara que esta cidade, Santa Maria do Grão

Andara é um lugar que mete medo

Andara foi a primeira parte da cidade que apareceu. Depois a cidade foi
aumentando e hoje Andara é um lugar quase esquecido, uma lembrança para
afogados,

ficou lá na margem de um rio,

há águas fundas, lentas, elas passam.

Andara é um amontoado de casas que dão portas e janelas para a floresta.
Você sai de casa e mal saiu já está sendo rondado por dentes, olhos

Ouve-se respirações. Um medo vem e vai ficando imenso. No alto, está a
lua, sempre. Branca, se ainda não é noite. Então se volta e se põe
trancas. Nesse dia não se sai mais de casa Andara é isso. Foi onde esta
cidade, ou uma outra, talvez a oculta, começou. Santa Maria.

Andara foi onde Santa Maria do Grão começou. No emaranhado.

Um emaranhado é sempre verde, me dizem.

Às vezes porém ele fica todo escuro.

Este não tem fim, seus rios que não existem e estas árvores ausentes ao
meu redor se estendem a perder de vista. Isto vai até onde um homem
puder ir. E vai mais longe ainda. Esta é a região. E Andara é mais:
Andara é o emaranhado inteiro.

Eu me pergunto, um menino deve saber tudo?

As histórias de Andara que tenho ouvido

Aqui também acontecem coisas.

Santa Maria do Grão também enlouquece.

Mas em Andara me dizem é pior. Lá se vai andando por uma rua e, então,
se vê, de uma janela estão nos olhando. Quem é aquele na janela que me
olha querendo não ser visto e querendo saber quem sou,

a pessoa pára na rua e se pergunta.

É só o vidro da janela, e no vidro um outro, depois se sabe. Você mesmo
é que estava lá e se olhava. Na hora, porém, não há como saber. Há um
outro ali, nos olha. E isso é tudo. E o outro é real\\

Andara é um lugar assim.

Como todos os outros lugares, todos,

me diz minha tia.

--- Isso tem um nome, ela diz.

Às vezes porém ela, a loucura, também pode vir se mostrar com uma cara
mais de caretas, risos indefinidos e também dá um desespero de rolar no
chão com gritos de se ouvir longe

Mas em Andara. Aí está essa tia me contando coisas.

--- Andara, diz ela, está lá. Entre a floresta de um lado e um rio do
outro lado, com uma vocação para a morte que não se vê por aqui. Está lá
com essa vocação e um querer ir sempre para um mistério mais fundo, e
outros, os sem fundo. Foi isso que fez Irido ir embora, por isso ele foi
morar lá. Nele também havia essa vocação para a morte.

Irido é o homem que a deixou.

Então ela conta a história do cemitério.

Em Andara foi que esta cidade teve o seu primeiro lugar de guardar os
mortos, diz. Se a pessoa não abrir todos os olhos e os do instinto de
fugir da morte também vai acabar se perdendo, entra numa rua que não
conhece e sai numa outra que nunca viu antes, e se atormenta, e sente
que está sendo puxada, levada, e então de rua em rua ela vai parar no
cemitério de Andara,

que já foi invadido pela floresta, que se mistura com ela, onde os
mortos e os vegetais estão juntos,

e a pessoa fica andando entre as casinhas de terra dos mortos, diz minha
tia, e ouvindo como a terra tem vozes antigas por ali, rumores. Depois
de dias e os dias não passam, entre os mortos, uma pessoa a quem isso
acontecer nunca mais será ela mesma.

Irido foi para lá porque não me queria mais, ela repete todas as vezes
para acabar a história. E se cala. A mulher.

Irido. O homem.

--- Tio Irido quer que eu vá para lá com ele, eu disse a ela um dia.

--- Não. Não vai.

Isso ela me diz assim, seca.

E essa tem sido a resposta quando falo que o homem quer que eu vá para
ficar lá com ele. Aquele homem lá em Andara.

Ele, porém, insiste.

Manda pedir que eu vá. Tem pedido muito.

Ainda hoje veio um homem de Andara, entrou na nossa casa e disse:

--- Irido quer saber se o menino vai ou não

Naqueles dias eu tinha as minhas irmãs e os olhos delas, de medo

Ficaram com mais medo ouvindo o homem falar, mais do que já tinham desde
que o nosso pai morreu nossa mãe morreu e ficamos assim, vivos

O homem jantou em casa e foi embora.

Minhas irmãs têm medo que o Andara me leve para o cemitério

Para elas, Andara não é só aquelas casas vazias lá, aquele começo de
cidade esquecido. É o Andara, e ele pega a gente, leva para a morte. E
aí não há mais volta para quem o Andara leva, elas dizem. E me olham. E
murmuram. E choram. E me olham como se já tivesse acontecido. É assim
que se dá com quem tem medo. A gente antecipa

Deve ter sido por isso que a minha tia acaba de decidir.

Agora ela deixou eu ir.

Ela também deve ter estado antecipando o que iria me acontecer se eu
fosse, só deixando poderá se livrar disso

Fui.

A infância. Ela é bem esse tempo de espantos por toda parte. E ela não
acaba nunca, eu sei.

Lá estava aquele homem.

Meu tio, que tinha uma noite nos olhos.

Me abraçou. E me levou para comer a carne de um animal que havia
apanhado, apanhei esta manhã, me disse, para te esperar. Aquela noite
nos olhos dele

Na mesa, enquanto comíamos, me disse amanhã, se tu gostas da carne,
iremos apanhar um outro.

Come mais.

Tu gostas?

Falava do animal e me dava mais para comer

Hoje, porém, quando acordei, tive medo de ir com ele apanhar outro
animal. E não fui

Como tudo passa rápido em Andara.

Tenho pedido tio me diga que animal é esse, quando comemos, todo dia ele
apanha um e traz para casa. Esta casa não existe, é muito velha, tem
paredes furadas e através delas podemos ver, lá fora, as árvores. E tudo
que vive nelas. Vamos ver o animal de perto, responde meu tio. Mas não
vou. Nesta casa comemos todo dia a carne do animal, que agora meu tio
põe outra vez na mesa. O jantar.

--- Hoje, e hoje tu queres ir comigo apanhar um?

Todas as manhãs me pergunta. E sai.

E volta com um animal. Já morto. E sem pele. E cortado em pedaços, é
para que eu nunca saiba que animal é.

É uma carne boa. Não há outra igual

E hoje?

Nunca irei com meu tio acho.

E também não saio desta casa para não me perder nas ruas e ir parar no
cemitério.

E hoje? Me perguntou esta manhã outra vez, antes de sair para apanhar
mais um animal.

É no cemitério que apanha todos eles.

Na parte do cemitério que vai sendo invadida pela floresta, ali onde a
floresta está bem viva, e avança sempre

E hoje?

Não respondi. Já não respondo mais. Ele entende que quando não respondo
estou dizendo não, não vou. E sai de casa

E hoje? Ele me pergunta sai.

Não levanta os olhos para fazer a pergunta. E tem sempre a noite
naqueles olhos. Eu sei.

Não vou, nunca irei

E hoje. Me pergunta.

O que eu não quero é ver o animal vivo e, depois, morto. Nunca
esquecerei o seu gosto porém

\pagebreak

\vspace*{4cm}

Fim para a infância agora.\\

E essa voz que diz no vento\\

--- Vem Curau. Vem levar os homens para os teus jardins\\

\pagebreak

\vspace*{4cm}

Mas uma outra voz está chegando.

Que diz:\\

--- Em Santa Maria do Grão quem passa por aquela rua não sabe porque
olhando a casa vê uma gaiola. Quem passa olha a casa. Para não se
atormentar mais com isso, quer encontrar uma explicação, e pára, olha
bem e depois se afasta e diz aquilo é só uma ilusão, vem das grades que
a casa tem.

E a impressão vai junto, acompanha quem viu a casa durante dias.

Mal se distrai, lá está de volta a casa. Vem na memória, e onde foi que
eu vi essa casa, quem passou se pergunta\\

A infância não tem fim

\pagebreak

\vspace*{4cm}

Espero a volta da ave. E enquanto isso escuto as vozes

As vozes da terra vêm de longe. Para ouvi-las basta se deixar ficar, não
ir embora nunca. Eu fico. O vento trará todas elas\\

Há outros por toda parte? Não ouvirão como eu estas vozes?\\

isso havia dito Jacinto e aquele homem fora embora, fora, voltara e
havia ido embora outra vez.\\

Isto ainda é um homem. Um inseto talvez me olhe. E isto insiste em
entender\\

Eu estou aqui. Cego.

Aqui é em toda parte\\

Diz Jacinto. E Jacinto é o homem na janela.

\pagebreak

\clearpage
\thispagestyle{empty}

\movetooddpage

\vspace*{4cm}

Se tudo continuar indefinidamente assim

por dias e dias e dias, um dia virá em que Jacinto já não será mais
Jacinto.

E haverá um outro na janela, ainda à espera

Bu. Um fantasma.

Uns ossos brancos

bu, um som para amedrontar crianças\\

De manhã, ele aparecerá na janela.

A janela estará muito velha.

Todos os dias, bu aparecerá na janela. Nela, a madeira não terá mais
idade, terá rachaduras que dirão somos as coisas mortas de um fantasma\\

E a vida, como será se o Curar não vier

\pagebreak
\pagecolor{black}

\chapter*{}
\pagecolor{black}\afterpage{\nopagecolor}


\movetoevenpage

\vspace*{4cm}

Bu, diz o vento passando agora por Jacinto ainda um homem.

Pois bu será Jacinto só se o Curau não voltar

\pagebreak

\vspace*{4cm}

Por muitos anos ainda Jacinto vai ouvi-las da janela. As vozes.

E os dias passaram velozes por ele. E não passavam

\pagebreak
\clearpage
\thispagestyle{empty}

\movetooddpage

\vspace*{4cm}

O homem estava lá, cego.

Quando o outro veio ele disse:

--- Ah você voltou outra vez. Me chame Jacinto. Eu estou aqui cego. Aqui é
em toda parte

Trouxe um menino desta vez. É seu filho, quer que ele escute também a
história do Curau. Sim. Você voltou

Eu conto, eu conto outra vez.

Sim. Em Santa Maria do Grão acontecem essas coisas.

Venha cá, menino. Você é novo para mim. Foi há muito tempo que o Curau
veio, você ainda não tinha nascido. E depois ele foi embora. Mas um dia
vai voltar. Você também precisa ficar sabendo como tudo se deu.

O menino ficou escutando.

O cego falava.

--- Foi há muito tempo. Mas eu lembro tudo. Eu nunca esquecerei.

O menino escutava.

Venha cá, disse o cego.

Deixe eu tocar nos seus olhos.

Deixou.

Os olhos. Estes. Os seus.

Não tenha medo. Toco. Sei.

Você é uma criança, as crianças não precisam ter medo do Curau. Que você
não seja mais um para correr pelas ruas quando a ave voltar. Tapando os
olhos, procurando um lugar para se esconder gritando o Curau, Curau. Lá
vem ele. Como eles gritavam. Tanto. Os gritos. E mais tarde nem fechavam
mais os olhos à noite com medo de não ter mais olhos para abrir de manhã\\

O menino ouvia e o pai quis olhar para a janela.

O cego contava

Tudo começou como eu conto agora. Conto para você outra vez. Eu fui
avisado antes dos outros.

A ave era toda vermelha. Estava lá, parada. Parecia doente. Dei a ela o
nome que quis, Curau. Foi o primeiro nome que saiu da minha boca. É só
assim, aprenda, que se pode achar o nome oculto de uma coisa oculta. E
aquela não era uma ave como as outras\\

O menino ouvia.

O medo só veio para aqueles que tinham as suas velhas razões para ter
medo, e esses passaram a ter medo então do Curau. Eles têm medo de tudo,
dizia Jacinto. E o menino ouvia. Quando a ave veio, aquele medo andava
pelas ruas com passos que só nos levarão a uma terra não-sagrada\\

Veja, tirando as crianças que têm medo reais, dizia Jacinto\\

Entenda, quando um menino grita o Curau, o Curau, ele só está fazendo
como os outros fazem. Imita. Se os adultos não tiverem medo, as crianças
também não terão medo da ave. Vão ficar nas suas redes, calmas, e as
noites serão sem espreitas\\

O Curau só furou os olhos dos adultos, quando veio da primeira vez,
dizia Jacinto\\

Todos devem pedir a vinda do Curau. A sua volta. Devem pedir todas as
noites, e dizer como quem reza vem Curau, e me cega. Me livra desses
olhos que não querem mais ver as coisas assim iguais

Só você não precisa pedir, menino, dizia Jacinto.

Pedir que o Curau venha cegá-lo.

Mas peça por seu pai.

Peça à noite antes de dormir.

O Curau é um bem que nos aconteceu, menino, e um dia ele vai voltar,
dizia Jacinto\\

Empurrado pelo cego, depois.

O menino se afastou.

Agora vá embora. Eu já disse tudo. Agora eu quero ficar só aqui, sentado
junto à janela. Cego. Disse Jacinto.\\

Na rua o homem dizia ao menino

--- Não

O menino quis olhar outra vez para o céu.

No céu havia nuvens, havia uma mancha. Era grande. Vermelha\\

Talvez agora um inseto me olhe para entender, diz Jacinto na sua janela.

Na sua janela, ele agora está dizendo, vem Curau

\pagebreak

\vspace*{4cm}

E essa voz que diz no vento\\

Ouvimos, ainda uma vez:

--- Vem Curau. Vem levar os homens para os teus jardins


\vfill
Fim de Os jardins e a noite\\

A viagem a Andara não tem fim