%\chapter{Prêmios recebidos pelo autor}

%\section{ORIGINAL PORTUGUÊS}

%PRÊMIOS RECEBIDOS:

%\forceindent{}


%\begin{enumerate}
%\item[1980]  Melhor Revelação de Autor (Apca -- Associação Paulista de
%Críticos de Arte) por \emph{Os animais da terra};

%\item[1981]	 Menção Especial no Prêmio Internacional Plural, México por \emph{Os jardins e a noite}, sob o titulo \emph{A Noite do Curau};

%\item[1988]	 Grande Prêmio da Crítica (Apca/Associação Paulista de Críticos de
%Arte) por \emph{Viagem a Andara};

%\item[2001]	 Segundo melhor lançamento do ano, Portugal, Melhores da Crítica
%Portuguesa, Jornal Público, com \emph{Ó Serdespanto};

%\item[2014]	 Prêmio de Romance Haroldo Maranhão (Casa das Artes, Brasil, Amazônia) com \emph{Breve é a febre da terra};

%\item[2016]	 Finalista ao Prêmio de Poesia (Apca/Associação Paulista de
%Críticos de Arte) com \emph{K O escuro da semente}.
%\end{enumerate}

\chapter{Prix reçus par l'auteur}

\begin{enumerate}
\item[1980] Révélation de l'Année -- auteur brésilien -- avec \emph{Os animais
da terra} [\emph{Les animaux de la terre}] (Apca/Associação Paulista de Críticos de Arte);

\item[1981] Mention Spéciale au Prix International Pluriel [Prêmio
Internacional Plural], au Mexique, avec \emph{Os jardins e a noite} [\emph{Les jardins et la nuit}]

\item[1988] Grand Prix National de la Critique avec \emph{Viagem a Andara} [\emph{Voyage à Andara}]
(Apca/Associação Paulista de Críticos de Arte);

\item[2001] Deuxième Meilleure Parution de l'Année, au Portugal, avec \emph{O Serdespanto} [\emph{L'Êtredétonnement}] (Jornal Público: Os Melhores da Crítica Portuguesa);

\item[2014] Prix de Meilleur Roman Haroldo Maranhão avec \emph{Breve é a
febre da terra} [Brève, c'est la fiévre de la terre];

\item[2016] Finaliste du Prix National de Poésie avec \emph{K, O escuro da semente} [\emph{K, l'Obscure de la sémence}] (Apca/Associação Paulista de Críticos de Arte) (Casa das Artes, Brasil, Amazonia).
\end{enumerate}