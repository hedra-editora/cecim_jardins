\chapter*{Prêmios recebidos pelo autor}

\addcontentsline{toc}{chapter}{Menção Especial no Prêmio Plural, México, 1981, com o título A Noite do Curau}

\section{ORIGINAL PORTUGUÊS}

PRÊMIOS RECEBIDOS:

1980 - Melhor Revelação de~Autor (Apca -- Associação Paulista de
Críticos de Arte) por \emph{Os animais da terra}

1981 - Menção Especial no Prêmio Internacional Plural, México \emph{por
'Os jardins e a noite/}sob o titulo \emph{A Noite do Curau}

1988 - Grande Prêmio da Crítica.(Apca/Associação Paulista de Críticos de
Arte) por \emph{Viagem a Andara}

2001 -- Segundo melhor lançamento do ano, Portugal, Melhores da Crítica
Portuguesa, Jornal Público, com \emph{Ó Serdespanto}

2014 - Prêmio de Romance Haroldo Maranhão

(Casa das Artes, Brasil, Amazônia) com \emph{Breve é a febre da terra}

2016 - Finalista ao Prêmio de Poesia (Apca/Associação Paulista de
Críticos de Arte) com \emph{K O escuro da semente}

\section{TRADUÇÃO PARA O FRANCÊS}

PRIX REÇUS

1980 - Révélation de l'Année - auteur brésilien - avec \emph{Os animais
da terra} {[}Les animaux de la terre{]} (Apca/Associação Paulista de
Críticos de Arte)

1981 - Mention Spéciale au Prix International Pluriel {[}Prêmio
Internacional Plural{]}, au Mexique, avec \emph{Os jardins e a noite}
{[}Les jardins et la nuit{]}

1988 - Grand Prix National de la Critique~ avec \emph{Viagem a Andara}
{[}Voyage à Andara{]}

(Apca/Associação Paulista de Críticos de Arte)

2001 - Deuxième Meilleure Parution de l'Année, au Portugal, avec \emph{O
Serdespanto} {[}L'Êtredétonnement{]}

(Jornal Público: Os Melhores da Crítica Portuguesa)

2014 - Prix de Meilleur Roman Haroldo Maranhão avec \emph{Breve é a
febre da terra}~ {[}Brève, c'est la fiévre de la terre{]}

2016 - Finaliste du Prix National de Poésie avec \emph{K, O escuro da
semente} {[}K, l'Obscure de la sémence{]}

(Apca/Associação Paulista de Críticos de Arte)

(Casa das Artes, Brasil, Amazonia)
