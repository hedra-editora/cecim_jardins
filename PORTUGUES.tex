\addcontentsline{toc}{chapter}{Os jardins e a noite, \emph{de Vicente Franz Cecim}}
\part*{Viagem a Andara oO livro invisível\\
\bigskip
\bigskip
\bigskip
\bigskip
\HUGE{Os jardins e a noite}\\
\Large{Vicente Franz Cecim}\\
\bigskip
\bigskip
\normalsize{A liberdade é uma noite escura?}}



%\chapter*{}
%\addcontentsline{toc}{chapter}{Original em português, de Vicente Franz Cecim}

\selectlanguage{brazilian}

\forceindent{}O que contém os animais?

O que os animais contém?\\

A estranha ave que cega os homens. Não proteja os seus olhos. A liberdade é uma noite escura?

Mais uma vez, bichos e homens

\breakk

\vspace*{4cm}

O labirantro.

Ele é Andara. Andara é onde Santa Maria do Grão começou, como se verá
agora nestes jardins, fazendo a floresta se abrir, recuar, e é por
Andara que a floresta está voltando.

Toda essa criança aí

Andara é o lugar de um mito entrevisto?

Talvez, mas não será só isso talvez

De qualquer modo, é aí que se enlaçam, se negam, natureza e essa outra
coisa inquietante que tem um nome. Civilização.

Toda essa criança escuta

A voz. Do labiantro

\breakk

\vspace*{4cm}

Escutará também o que não é Andara?

Andara é a viagem fora de si e deverá continuar sendo isso, um gesto sem
gesto, estará em outra parte

Isso ficará em branco. A vertigem. Tira a terra de sob os nossos pés e
no entanto não a perderemos de vista.

Os olhos que antes leram a história do Nazareno e o livro do cego Dias,
assim eles também liam um livro que não estavam lendo.

Talvez agora não sem surpresa saibam ou já sabiam? Que sob livros há não
livros.\\

Viagem a Andara, o livro invisível

\breakk

\vspace*{4cm}

Bem.

No labirinto se deve dizer aos outros: sem um texto, não há tempo

E assim, há a maneira infinita de ler Andara.

Um jogo de deslocamentos e, às vezes, reuniões arbitrárias, segundo cada
um e, em cada um, segundo o tempo de uma emoção

Dança para as intuições.

\breakk

\vspace*{4cm}

Nessa dança, a seqüência dos livros de Andara fica abolida. E se pode
iniciar a viagem por onde se abrir primeiro.

--- Ó flor Andara, de sonhos

Isso não tem um limite.

Isso se desregra. Mais tarde uma leitura de Andara na memória encontrará
novas combinações. Eis um ideal: a imaginação na penumbra, libertada
pela ausência física de um texto

Os desastres curiosos que então ocorrerão: quem voa ao lado de Caminá, é
o pássaro Curau? E: existiu um livro chamado Os dias cegos, onde se
conta a história do cego Dias?

Mas é mais: tendo arrancado as páginas de todos os livros,

e isso seria luminoso se se desse num momento de revolta,

o leitor, com a ajuda de um vento inesperado que primeiro espalhe e
depois reorganize a viagem a Andara. Paro. As janelas devem ficar sempre
abertas para que os ventos sempre entrem e nunca uma ordem final se
instale. Rigorosa.

\breakk

\clearpage
\thispagestyle{empty}

\movetooddpage

\vspace*{4cm}

Esta viagem à coisa humana à noite\\

E abertamente votei meu coração à terra grave e sofredora e, muitas
vezes, na noite sagrada, prometi amá-la fielmente até a morte, sem
receio, com seu pesado fardo de fatalidade, e não desprezar nenhum dos
seus enigmas. Diz Hoelderlin

Esta noite a ser negada.

Haverá outra

E abertamente votei meu coração à terra e muitas vezes, na noite
sagrada, prometi amá-la fielmente e não desprezar nenhum dos seus
enigmas\\

O fardo de uma história com uma história.

Como seria uma história sem uma só história, o fardo feito em pedaços?

O não.

O labirinto.

O não. Não à uma história única. Ainda que seja a história de um homem.
Fazer com que por ela passem tantas histórias de outros homens quantas o
ouvido ouvir e o vento contar

O labirinto. Neste que começa agora, há vozes que esse homem em sua
janela escuta, reais,

vêm de documentos antigos sobre a dor e se misturam à encenação, feita
pela voz de Andara, da dor imaginada\\

Ei-lo, então.

O não labirinto

\breakk

\vspace*{4cm}

Uma espécie de murmurador na noite.

É isso a imaginação

Ela vem. Não se sabe bem de onde.

Eis o que temos. Um homem numa janela. Não há como evitá-lo. E há vozes.
Então é só entregar-se,

esta viagem fala da vida e não vai parar antes do fim.\\

Outra vez.

Como será?

\breakk

\vspace*{4cm}

O homem estava lá. Cego.

Quando o outro veio, ele disse:

--- Me chame Jacinto. Eu estou aqui cego. Aqui é em toda parte.

Sim. Em Santa Maria acontecem essas coisas.

Você veio para ouvir a história. Lhe disseram por aí que eu sei como
tudo se deu, então você veio. Disseram onde eu moro. Trouxeram você até
a rua e apontaram a porta desta casa. Então entre.

O outro homem entrou.

Ficou escutando.

Durante uma parte da tarde, o cego falou. E o outro ouvia.

Ele, o cego, disse:

--- Não faz tanto tempo.

E eu lembro de tudo. Eu nunca esquecerei.

Ouça.

Foi como agora estou dizendo, com estas palavras. Você pode não
acreditar que fui avisado antes dos outros, mas fui. Não há pelo que
jurar. Há apenas esta voz como ela é, e não dê seu ouvido ao vento da
noite, desta noite sem clarões, para iluminar enigmas, isto se
contorcerá com lâminas por toda parte, dê seu ouvido só a esta voz e
deixe que o vento leve, também, a parte que toca a ele nisto que vai
ouvir. Outros também precisam ficar sabendo como tudo se deu\\

Ouça.

A ave era toda vermelha. Estava lá, parada. Parecia doente.

Foi assim que tudo começou.

Eu dei a ela o nome que quis, já que as coisas que se conhece têm todas
um nome de uso comum, mas as outras, essas que vêm não se sabe de onde,
não se sabe nunca, a essas se pode dar o nome que se quiser, é só abrir
a boca e ficar com a primeira palavra que sair, é só nisso que há uma
possibilidade de achar o nome oculto da coisa oculta. E aquela não era
uma ave como as outras, isso eu vi logo.

--- Curau.

Foi a primeira palavra que saiu da minha boca. Aquela boca ficou assim
viva. Curau. E ficou sendo esse o nome da ave desde o dia de rumores e
de uma mudez, a minha, quando abri a porta e ela estava lá. Estava mesmo
doente, não fugiu quando me aproximei. Quis que entrasse em minha casa.
Cuidei dela. Logo ficou boa. Era forte. Grande. A ave.

É preciso que eu lhe diga: antes, eu estava de olhos fechados, e vi
tudo.

Foi antes de achar a ave. Eu tive um sonho.

Quando acordei, não entendi o que havia visto sem meus olhos. No sonho.

Agora sei que tinha sido uma anunciação. Havia visto os dias de medo que
mais tarde vieram se instalar entre nós

Quando acordei, fiquei revendo aquelas asas imensas,

uns olhos de fogo me espiavam do fundo da noite,

e no sonho que tive havia uma noite onde homens corriam, fugindo também
como as mulheres por ruas vagas, ruas onde não pesavam as pedras,

fugindo como passaram a fugir também nas ruas reais desta cidade.
Gritando. Querendo se esconder.

Acordado do sonho, eu ainda não podia entender, vivendo como vivia na
ilusão de que só o que se vê de olhos abertos é real, e espelho falso
tudo o que nos aparece quando se fecha os olhos. Não podia. Havia aquela
ave, ela voava por cima da cidade, ela mergulhava sobre uma rua e se
ouviam gritos. Os gritos.

Esse foi o início. E eu o vi.

Foi assim que a vida me avisou que o Curau ia chegar.

E depois ela, a vida, veja, veio pôr o Curau direitinho na porta desta
casa

Mais tarde entendi

Nós temos uns olhos, como eu tive, para tropeçar por toda parte. Um
espelho nos cega apesar do sol

Ele ainda está lá, no alto?

Ele ainda existe?

A noite hesita e nos recusa com esses olhos que temos para nada

Foi tudo como eu disse.

Antes de voar entre os homens, o Curau voou em Mim, entre os olhos que
eu tinha, fechados, e a vida lá fora. Voou primeiro num céu humano.

E então achei a ave e a levei para casa.\\

O medo veio para os outros só mais tarde.

Foi mais tarde que se começam a ouvir seus gritos. Então já era o Curau,
voando por cima da cidade. Voava e eles gritavam o Curau. Curau. Lá vem
ele outra vez. E também as crianças gritavam.

Elas não deviam ter medo, porém.

O medo só veio para aqueles que já tinham as suas velhas razões para ter
medo,

esses já tinham o medo neles e passaram a ter medo, então, do Curau.
Esses têm medo de tudo. E a ave era só a coisa agora visível que fez o
medo deles surgir, aquele medo andava pelas ruas com passos que nos
levarão, que levam sempre a uma terra não-sagrada. Andavam assim na
região de um deus sem Rosto, e no olho esquerdo dele haviam enterrado um
Espinho

Veja, tirando as crianças que têm medo reais

Não. Isso eu quero dizer de outro modo. Como entendi

Quando um menino gritava o Curau, o Curau, ele só estava fazendo como os
outros faziam, imitava

Se os adultos não tivessem medo, as crianças também não teriam medo da
ave

Ficariam nas suas redes, calmas. E a noite seriam sem espreitas, para
voar sem sustos os vôos infantis

Um Curau nunca ataca, nunca faz mal a uma criança. Nunca se falou do
ataque da ave a um menino. Ela fura somente os olhos dos adultos. Você
nunca vai ouvir dizer que ela cegou uma criança\\

Outros, porém,

esses nem dormiam mais depois que a ave veio. Os adultos. Tremiam à
noite. Tinham pulos e espantos. Tinham uma certeza: a de que a ave iam
entrar pelas suas janelas a todo instante e cegá-los, nessas casas onde
depois ficariam vagando, sem rumo, se dando contra as coisas duras

E os dias passavam\\

Havia o medo. Estava dentro. E ao redor.

Ouvi desmoronamentos.\\

É que eles não entendiam.

Um Curau não faz mal.

Se era até preciso pedir a vinda dele.

--- É preciso pedir que ele venha, isso eu dizia a todos.

Dizia a eles:

--- Todos devem pedir a vinda do Curau.

Devem pedir todas as noites e dizer como quem reza:

Vem, Curau, e me cega

Me livra das coisas que não mudam,

estes meus olhos não querem mais ver os dias assim iguais

Me faz, Curau, cair

para dentro da noite que eu sei que sou,

para que tudo mude

para que eu volte a ter gosto pela vida

Curau,

isso todos deviam dizer, de olhos fechados,

eu não vejo mais nada

Não vejo os outros homens, há uma máscara em cada rosto

Faz também esses outros se perderem na tua noite, perderem os olhos com
eu, faz esse bem a eles

Cega esta mulher que dorme ao meu lado, cega estes homens ao redor de
mim,

eles também só vêem a máscara no meu rosto e não podem ver, como eu não
posso ver neles, sob a máscara, o rosto que tive. E ainda tenho. E está
oculto. Soterrado. Um ambiente sem luz.

Vem, e te peço, não vejas a máscara no rosto de uma criança. Isso eu te
peço. Pois nelas os dias vivem

Por isso deixa em paz os olhos delas

e fura só meus olhos, Curau, entrando por esta janela com a tua luz
negra.

Eu não sei mais ver\\

Isso era o que todos deviam pedir.

Isso eu dizia a eles.

Deviam se reunir nas igrejas e nas praças para pedir. E também pedir
sozinhos, como fazem aos seus santos em voz baixa

--- Peçam a vinda da ave e esperem que ela venha, eu dizia.

E que atenda logo o pedido.\\

Veja. Eu.

Você está escutando um homem que já teve os olhos furados pelo Curau,
disse o cego ao homem.

O outro ouvia.

Agora o que eu sei é o que eu sei. Nada.

O outro ouvia e o cego disse a primeira coisa que a ave fez foi me
cegar. Acabar com os olhos que eu tinha. Gastos.

O homem em frente ao cego olhou para janela.

Mal teve forças de novo, continuava o cego, depois de comer a minha
comida, ela pulou em mim e veio esta noite em que agora estou, aqui.
Aqui é em toda parte. Nesta noite eu espero pelo que virá, virão os dias
sem nome, eu sei

Espero e sei

Espero e escuto. As vozes vêm no vento

Espero e vou tocando as coisas. E entendendo.

Passo esta mão na cara de alguém e entendo que está triste e de onde vêm
as tristezas, passo esta mesma mão na cara de um outro e entendo o seu
medo e de onde vem o Medo e para onde o Medo vai. E de que é feito o
Medo.

Então digo com voz lenta para que não se assuste

tenha paciência espere um dia a ave vai voltar\\

O homem diante do cego quis outra vez olhar para a janela.

Ela se foi, ouviu que o cego dizia.

Ele então quis saber o resto, o que tinha acontecido depois.

Um dia ela foi embora, estava dizendo o cego.

Não se sabe para onde. Deve ter voltado para o lugar de onde veio.

Estará agora num ninho cruel, num lugar oculto em alguma parte da vida,
longe ou perto de nós. Não se sabe

Agora, nesta janela espero o Dia.

Esse Dia em que retornará, diz o cego. Será numa tarde como esta talvez.
Abrindo as nuvens. Quando a ave voltar, estava dizendo. E então.
Recomeçarão os gritos. As fugas. Eu

espero aqui.\\

O homem ouviu que o cego lhe dizia, Fique, você veio de longe, e espere
a volta dela. Da ave.

O cego estava pedindo para tocar nos olhos dele. Deixou.

Venha cá, dizia o cego.

Deixe eu tocar nos seus olhos.

Os olhos.

Estes também. Os seus.

Todos assim.

Os seus são como os olhos que eu tinha. Olham para fora da vida.

Livre-se deles também.

Ah mas você tem medo. Toco. Sei.

Vai ser mais um para correr pelas ruas quando a ave voltar. Tapando os
olhos, procurando um lugar para se esconder gritando o Curau, Curau. Lá
vem ele. Como eles gritavam. Tanto. Os gritos. E mais tarde também nem
fechará mais seus olhos à noite, com medo de não ter mais olhos para
abrir de manhã. Para não ver nada.\\

Empurrado pelo cego.

O homem se afastou.

O cego estava dizendo vá embora agora. Eu já disse tudo. Agora eu quero
ficar só. Ficar sentado junto à janela. Cego.

\breakk

\vspace*{4cm}

Escuto.

Estou aqui. Cego.

\breakk

\vspace*{4cm}

Depois que o homem fora embora, o cego havia ficado só outra vez

\breakk

\vspace*{4cm}

E essa voz que diz no vento\\

--- Vem Curau. Vem levar os homens para os teus jardins

\breakk

\vspace*{4cm}

Escuto.

Estou aqui. Cego.

Aqui é em toda parte.\\

É a voz do cego, falando. Mas haverão outras, foi dito\\

Espero a volta da ave. E escuto as vozes, ele está dizendo. As vozes da
terra vêm de longe. Para ouvir as vozes da terra basta se deixar ficar,
o cego diz junto à janela.

Isto é um homem, ele diz. Um inseto me olhará sem entender.

Há todos esses rumores\\

Agora o cego está escutando.

E diz: É do fundo da cabeça que me contam estas histórias. Tudo vem no
vento também

para aqueles que insistirem em avançar nesta noite, e ela, a vida, por
sua vez também avançará. E às vezes grita. Outras vezes murmura

\breakk

\vspace*{4cm}

Agora

é a hora das sombras se aproximando das coisas.

Mais tarde, já ninguém poderá ver seus pés, que ficarão tropeçando nas
coisas, hesitando dentro das casas sem saber para onde levar o corpo. Os
braços também irão sumir, e se poderá estender eles para frente sem ver
o que as mãos irão tocar, e todos assim, mutilados, irão vendo com
espanto embora isso aconteça todos os fins de tarde, que desaparecem. E
também irá escurecer em todos os olhos até que não reste mais nada para
entregar à noite. Mas não é o Curau voltando, não é. É só o Curau da
vida fazendo a vida desaparecer por algum tempo para nós

A noite. Agora ela vem vindo\\

Um grito. É uma ave.

Ele diz a Jacinto que tudo em Andara está deixando de ser humano. É de
Andara que veio esse grito até a sua janela em Santa Maria.

Santa Maria do Grão à noite.

Agora o medo, o Medo vai sair aqui para fora e correrá por toda parte
levado pelo vento.\\

Na janela Jacinto espera

Vem no vento uma queda.

Está começando, diz Jacinto na janela. E espera

Há um homem na janela que escuta

e a vida fala para ele

vinda no vento, ela, a vida, para que eu que levo o fardo dessas
histórias a escute também\\

Mais tarde, ele ouvirá um ai

É a infelicidade.

Ela está se instalando.

Então Jacinto sente que tudo está começando outra vez, e que esta vai
ser mais uma noite para não dormir, e que só ele, o homem na janela, vai
ficar quieto, mudo, enquanto tudo fala e enquanto os outros

Vem no vento:

--- Lhe digo que arrancou os olhos com as mãos. Caiu, rolou no chão e
chamou por alguém. Veio a mulher. E ela então, para que não sofresse
mais, usou a corda para matá-lo

Isso diz uma voz.

E na janela o cego sabe que há uns que não resistirão até o amanhecer.

Amanhã, quando a luz voltar, será mais um dia para enterros. Mas um dia
as Asas vão voltar e isso não vai mais acontecer, diz o homem na janela.

Ele espera.

E sabe que à medida que a noite avançar,

o vento aumentará sua força até que comece a arrancar janelas, derrubará
os homens das redes, baterá na porta como aquele que quer entrar à força
enquanto do outro lado da madeira se amontoam coisas, mesas, mortos,
cadeiras, tudo para resistir ao vento. E o vento também irá para a
margem do rio e afundará os barcos amarrados na ponte e virará
candeeiros e apagará velas, e não deixa nenhum refúgio de luz para onde
se possa correr, sendo perseguidos, todos, já agora também por uma coisa
sem nome que vem de dentro de cada um, e está fora também, e está também
no sono dos que dormem e tentam escapar assim fingindo que não sabem de
nada, os adormecidos, que nem estão vivos, e entra pelos seus ouvidos, e
mesmo no fundo do sono ninguém está a salvo pois agora o vento também
fará com que sonhem que está virando tudo dentro deles, em suas cabeças,
revolvendo, fora, os cabelos, e, pálidos, eles vão acordar querendo
fugir aqui para fora para a vida, mas é aqui fora que nela um verdadeiro
inferno espera e o vento agarra as mulheres e quer porque quer levantar
as roupas delas e arrancar os meninos dos seus ventres mal eles põem a
cabeça e espiam aqui para fora. Os que tentarem nascer esta noite.
Espiam a vida que se atira, se atira e para onde,

se pergunta Jacinto em sua janela.

Ele já viveu noites como esta.

A vida poderá ser outra coisa, diz o cego. Mas só para aqueles que
souberem esperar que amanheça

Logo amanhece, pensa Jacinto em sua janela.

E espera.\\

Esta noite, outras vozes outra vez virão no vento\\

Ouve.

Elas dirão a ele:

Que agora as águas do rio enlouqueceram, estão correndo ao contrário.
Sobem para lá, apontam as vozes

E dizem, Senhora ilumina com tua faca de luz esta noite

E dizem também, O encanto O encanto. As alucinações estão chegando no
porto, trazem com elas uma esperança, venham todos

Isso dizem as vozes ao homem na janela.

E dizem mais, dizem, Um filho de deus foi derrubado pelo vento e
esfregado nas lajes da igreja quando fazia suas orações. A igrejinha
está rachando, não resistirá à esta noite

Não suporto mais esperar que amanheça,

dizem as vozes.

E o homem na janela escuta. Espera.

Talvez um deles peça agora que se sacrifique um inocente, se diz
Jacinto.

Dirá, É para nos salvar.

Eles beberão seu sangue se for derramado.

E mais tarde acenderão fogueiras na noite. Haverão de fazer uma festa. E
depois tentarão adormecer, pesados de álcool.\\

Logo amanhece.\\

Esta noite já vai acabar, murmura Jacinto.

E então vêm no vento vozes mais antigas,

falam de outro tempo de torturas.

E o vento diz: Há uns, condenados a serem puxados por quatro cavalos em
quatro direções.

Sim, diz Jacinto.

E pensa, já vai amanhecer.

E vem no vento, de outras terras, esta história: Lá, numa terra que
nunca viste, uma mulher matou outra. É por isso que agora está sendo
levada para morrer, agora é a sua vez. Vai morrer em frente à cadeira
onde estava sentada a outra mulher quando a matou.

Jacinto pode ver isso da sua janela

É a vingança. Diz o vento.

E está dizendo e sendo levada ao lugar onde matou a outra, sua mão será
cortada e jogada no fogo, para que veja. É a mão direita com que matou e
é preciso que ela a veja se transformando em cinzas

Para que veja, repete o cego na janela.

Será morta com a mesma faca, diz o vento.

O homem na janela repete é a vingança

A mesma faca, diz o vento.

E pára.\\

As vozes que vinham no vento de Santa Maria do Grão também param. E não
chega mais voz alguma até o homem, nem de Andara.

Em sua janela agora ele não ouve mais nada.

Depois, o vento está vindo outra vez. E um pássaro todo negro se quebra
na cara do homem. Ele sabe que quem jogou o pássaro foi o vento. E não
se importa.

--- Mataram Mariana, mataram Mariana, grita alguém debaixo da janela.

Na janela, o homem não acredita nessa voz.

Ele sabe que foi um pássaro que o vento jogou na sua cara. E que Mariana
não é negra. Também sabe que muitas noites ainda virão para se ter esse
medo de estar vivo sem remédio. E espera que amanheça\\

A manhã agora vem vindo.

Eu a sinto chegar.

Vem por aquele lado, entra pela janela, vai iluminar primeiro as mãos
dos homens para que eles possam achar um copo de água e afastar essa
noite. Depois, iluminará seus pés para que vivem mais um dia antes que
as sombras tornem a tocar as coisas brancas

Eu estou aqui.

Aqui é em toda parte, diz o cego na janela

\breakk

\clearpage
\thispagestyle{empty}

\movetoevenpage

\vspace*{4cm}

Amanhece.\\

E se a infância vier até ele agora?

A infância então vem

E uma voz vem dizer no vento:

--- A constelação do cão está latindo outra vez

\breakk

\vspace*{4cm}

--- Fim para a infância,

grita, mudo, o cego junto à janela.

Ele está lá. Cego.

E espera a volta da sua ave. A primeira palavra que saiu da sua boca.
Curau.

--- Temos os mortos para velar, diz Jacinto. Eu antes estive com eles.
Agora não quero mais deixar este lugar, avançar no escuro. Os mortos são
peixes e estão indo para outras águas, não se sabe para onde\\

E no vento outra voz vem dizer a ele:

\bigskip
\bigskip
\bigskip
\bigskip

aquela voz vinha do outro lado da sala.

Era como um trabalho de inseto.

Depois, parou de falar

Uns grandes outros pequenos iam se aproximando do homem deitado os
vultos para mais um último beijo. Ouviu as vozes infantis. Lá fora uma
ave da noite cantou. Uma brisa agora está entrando pela janela e os seus
cabelos fingiam ainda um resto de vida para os outros esses que estão ao
seu redor

Na sala alguém murmura

--- Então passamos a fronteira, e não se escutou o fim do que dizia

Durante toda a noite estivemos velando na casa.

E depois o morto não estava mais lá.

Mas ainda o beijavam sem saber que já havia partido com os lábios frios
da madrugada. Os parentes

--- Nunca fomos tão lentos como nesse instante, voltava a voz do outro
lado da sala. Na minha terra, há pescadores que amarram os mortos pelos
pés para que não se levantem e voltem para o mar, diz aquela voz

\breakk

\vspace*{4cm}

Eu estou aqui.

É o cego.

Isto é um homem, ele diz. E isto também quer entender

\breakk

\vspace*{4cm}

--- A constelação do cão está latindo\\

--- Nunca fomos tão lentos como nesse instante\\

As vozes ainda peregrinam pelo ar dessa manhã que nasce

Enquanto em suas casas os homens dormem, curando em sonhos as feridas da
noite que passou\\

E até Jacinto na janela uma nova voz está vindo no vento para dizer:

--- Andara. Lá as ruas estão sempre vazias, e Aquilo, a floresta, avança,
vem cada vez mais para perto de nós\\

Jacinto tem fome.

Isto ainda é um homem, ele diz.\\

Andara é a África que temos dentro

\breakk

\vspace*{4cm}

E se viesse agora uma outra voz?

E essa voz está vindo. E diz:

--- A velha. Ela mastiga lentamente. Ela também é um animal à noite,

como me disse que era aquele tio Irido que foi para Andara e nunca mais
voltou. Não vejo os seus olhos, ela não tem olhos para se ver

Jacinto está ouvindo e pensa, Agora é uma irmã minha quem me fala\\

E a voz:

--- Agora ela anda no jardim. A velha. Um inseto a vê. Se olham. Não há o
que entender

\breakk

\vspace*{4cm}

Ventos falantes, ventos falantes\\

E essa voz que agora vem contar ao cego na janela

isso:

--- Um homem está ali e olha o rosto de uma mulher que dorme e a
adormecida vai desaparecendo em sua noite. Mas no rosto dela as coisas
que hoje viveu ainda vivem,

e nessa máscara que foi seu rosto as coisas dizem

vivi mais um dia com suas dores e a alegria que cabe a cada um nesta
terra que os homens chamam minha casa.

Depois, no rosto da adormecida até a máscara das coisas vai
desaparecendo. E então para olhar já não há mais nada. O vazio.

--- Por onde ela andará, pergunta o homem olhando esse vazio. Que começa a
se tornar uma paisagem,

pois outras coisas agora, no rosto dela, estão começando a vir à tona.

A mulher se agita. Afasta algo. Não. Diz sem voz. Não.

E o homem sabe que ela sonha com alguma coisa que não quer, que recusa.

E o que ela não quer é essa paisagem que surge, onde o homem vê um
menino e o menino está vendo as figuras que gostava de achar nos muros,
as manchas,

para nelas ver bichos, armas, um pouco de sangue correndo no início,
depois mais sangue ainda, correndo mais forte. É o homem e sua fome,
quer um animal para a sua mesa, dizia o menino olhando o muro. A mancha.
Era só deixar os olhos irem e se deixar ir, junto, para ver

homens atrás, perseguindo, e o animal fugindo sem ter para onde, preso
na mancha, no muro. A armadilha

\breakk

\vspace*{4cm}

--- A fome vem na hora inicial da vida, quando é manhã e os olhos se abrem\\

diz uma voz e a manhã ainda nasce.

Esta manhã, o tempo está estagnado

E nas suas casas os homem dormem\\

Então, mais nada irá acontecer além dessas vozes contando histórias para
o cego na janela?

Ainda não sabemos. Inquietos,

os viciados da continuidade.

\breakk

\vspace*{4cm}

--- Vendo a armadilha o homem foi indo na memória,

conta a voz que fala da adormecida.

Mas agora ele se agita.

É que para o rosto dela um outro homem se inclina.

--- Quem será, se pergunta o homem. E como tudo está escuro e não pode ver
o rosto do outro com quem ela sonha, pensa que o outro bem pode ser ele,
quem sabe

Uma noite agora está vindo sobre a região onde a mulher sonha, num fim
de tarde, e lá ela já não se recusa, não afasta a sombra.

--- Lá, onde ela está, onde eles estão, ninguém mais está, diz o homem ao
lado da cama.

Olha ainda e está vendo a mulher e o outro, ele talvez, se afastando por
entre árvores, num jardim. Ela tem a cabeça apoiada num ombro escuro, a
última coisa que ele vê

\breakk

\vspace*{4cm}

Nova voz está chegando, entra pela janela. O cego ouve

Isto não acaba. Não acaba?

Mas uma vez.

A infância não tem fim\\

--- Em Santa Maria do Grão, conta a voz

quem passa por aquela rua não sabe porque olhando a casa vê uma gaiola.
Quem passa olha a casa. Para não se atormentar mais com isso, quer
encontrar uma explicação, e pára, olha bem e depois se afasta e diz
aquilo é só uma ilusão, vem das grades que a casa tem

E a impressão vai junto, acompanha quem viu a casa durante dias

Mal se distrai, lá está de volta a casa. Vem na memória, e onde foi que
eu vi essa casa, quem passou se pergunta.

É assim.\\

A infância não tem fim?\\

Na casa mora uma mulher. Já não há homem ali, mas ficou uma menina, a
filha, e come tudo o que lhe cai do alto.

Por toda a casa se move a mãe grasnadora, arrumando coisas, varre o chão
e tem aquela pena negra que sai da abertura do vestido, atrás. Foi com
ela que agora a mulher acabou de varrer o chão.

No mais, é uma mulher como as outras. Como as outras, pode ser vista por
toda parte.

Foi assim que o homem um dia a viu.

O homem ria, se ficava por baixo da mulher sentindo aquela pena no seu
corpo agora morto.

Antes, a filha também ria. Mas a mulher sabe usar a pena com violência e
a menina nunca mais riu.

Com a pena a mulher se abana nos dias quentes, com ela abana o fogo que
quer morrer sob uma panela. Para muitas coisas a mulher usa a pena. E
agora ela está de costas fingindo que não viu mas de repente se volta e
há uma mosca a menos nesta vida

Com a pena a mulher também persegue insetos. E sabe achá-los nos buracos
e frestas e não há nenhuma frestinha para onde possam fugir

Os insetos.

Bela, a pena brilha.

O sol vem iluminá-la. Negra

À noite, porém, não se sabe para onde a mulher vai. Todas as noites sai.
Já está amanhecendo quando volta, traz comida, a filha come o que lhe
cai do alto

É uma ave. E caça ao luar.

É isso, a menina se diz, só, em casa

enquanto a mãe voa pelas ruas à noite e ela fica, fecha os olhos e assim
está vendo a mulher voar, depois vem um sono calmo e ela vai adormecendo
e dizendo, Tenho esta mãe, tenho esta mãe. Devo acordar amanhã

O rosto da mulher tem começado a anunciar a morte, porém.

--- Ela vai se cansando de estar viva, diz a menina no quarto para o
espelho. Ela vai se cansando de estar viva, disse a menina um dia
olhando o rosto da mãe adormecida, uma tarde

Ela viu que a pena ia se tornando sem luz.

E teve medo. Se ela acaba, foi dizer outra vez ao espelho.

Havia vindo o medo.

O medo.

Mas agora a menina não tem mais medo.

O medo foi embora.

Saiu daquela casa.

Foi na hora do jantar, hoje

Quando ela se sentou na mesa, havia alguma coisa embaixo dela, entre ela
e a cadeira.

Levantou, foi olhar no espelho e retorcendo-se toda para trás pode ver.
Era um início. Nela também nascia

Voltou para a mesa. E comeu com uma fome que também voltava. Olhava a
mãe sentada como sempre de lado e sentou também de lado. A herdeira. E
disse baixinho eu e ria.\\

Agora, passa alguém em frente à casa, quem passa olha a casa

\breakk

\vspace*{4cm}

--- Se vocês querem, falem mais da dor\\

É o cego quem diz isso. Fala com as vozes que tem escutado. As vozes.

Elas nunca deixando de vir contar a ele as suas histórias, no vento,

os dias não passavam\\

As vozes. Estas vozes, estas histórias

Por muito tempo ainda elas vindo, para dizer a ele que a vida, lá fora,
ainda é um lugar de rumores e um Não submete tudo.

Nele, homens se buscam sem se ver, diz o cego. Mas um dia a ave vai
voltar

Ele ri.

Na sala onde ele está há um espelho, mas o espelho não mostra esse riso

\breakk

\clearpage
\thispagestyle{empty}

\movetooddpage

\vspace*{4cm}

Ah você bateu na minha porta. Você voltou.

É o homem quem entra agora.

Ele voltou.

Veio ver como vou passando.

Eu não passo. Os dias não passam. Nada disso muda enquanto o Curau não
volta, diz o cego ao homem.

Eu tenho esta janela, e nela o tempo está parado. Nela espero a ave.
Você lembra, eu lhe contei tudo, como começou,

diz o cego ao homem e o homem estava de volta.

Havia entrado como da primeira vez.

E agora estava outra vez em frente ao cego.\\

--- Sim, tenho ouvido o vento, lhe diz Jacinto. As histórias que ele vem
me contar. Eu escuto o vento sempre. Não, você não poderá ouvir as
histórias também. Não. Você ainda tem esses olhos. Quer tê-los.

Se quiser ouvir outra voz. A minha.\\

Agora o vento parou. E o cego tem aquele homem de volta. E espera a
volta da ave. Quer ouvir asas no céu. As Asas.\\

--- Quando o vento pára, eu lembro outras histórias, ele diz. Quer ouvir
uma delas?

Quer. Já contei a história do Curau, agora você quer ouvir mais, sempre
mais? Deixe que eu fale então, enquanto espero. E não pense que a
história do Curau é só uma história. Eu estou aqui, só. Aqui é em toda
parte.\\

O homem escuta.

--- Olhe, diz o cego, o tempo é isso e então eu já não lembro como acabou.

Agora é a história de Sumiro que ele irá contar ao homem. Não pense que
a história do Curau é só uma história, ele repete.

Talvez invente um final para essa história agora, diz. E o homem ouve.
Talvez recorde à medida que for inventando. Não se sabe nunca. Há a
memória, esta coisa à noite. Não lembro mais também os nomes. De
qualquer maneira, nela, Imaginação, há coisas que crescem, fogos
enormes, e há o que se apaga. Ou vem mudado de volta, na volta, quando
se quer lembrar. Na Memória.

Por esses caminhos eu vou. Imagino. Lembro.

Eles são dois. Se misturam.

Disse que não lembro mais os nomes?

Não minto. Só o nome dele eu não esqueci. Sumiro. Era Sumiro. O
inesquecível.\\

Foi num fim de tarde aquilo.

Eles vieram e o levaram, arrastado.

Naquele tempo eu ainda tinha olhos para ver, diz o cego. Foi antes do
Curau e com eles eu não via nada, entenda.

Veja, a terra por onde o arrastavam já ia escrevendo no corpo dele esta
história que agora eu lhe conto

escrevendo ela também no seu ouvido,

e isso se dando enquanto o levavam, era uma antecipação de toda a dor
que ainda viria mais tarde para ele,

como ainda virá mais tarde para você que agora ouve a história a sua
parte de dor

Eles, porém, não iam levando o homem arrastado porque resistisse, não. É
que queriam levar ele assim. Veja.

No lugar que escolheram para fazer o que fizeram, havia um que esperava,
vigiando, um par de olhos vivos, as armas prontas para usar contra ai de
quem tentasse impedir o que eles iam fazer. Veja agora isso, assim
aquele era um Lugar fora da terra, ficava por cima do inferno. Foi lá
que aquilo se deu. Naquela hora aves começaram a voar baixo, anunciando
vem morte vem a morte, voavam e a gente entendia aquelas asas.

--- Tome um café, disse o cego ao homem.

Ah este frio, disse o homem.

É a noite, disse o cego.

Não. Eu não quero. Esta noite, mais tarde eu quero sonhar depois de ter
lhe contado tudo. Se as vozes deixarem

Ouça.

Então, eu ia dizendo: eles arrastaram aquilo como se não fosse um homem.
Sumiro. Mas era um homem.

Ainda que vivesse daquele jeito sempre, sempre tinha uns olhos baixos
procurando no chão e não se sabia o que procurava. Nunca se viu ele
levantar os olhos. Eles, os olhos daquele homem, se davam à terra. Só a
ela. Dias e dias, ele vivia assim. Era por dentro que vivia. Havia quem
dissesse que nem era mais um homem e que não estava mais entre nós, os
outros. Para mim, era.

Não sei que outra coisa podia ser aquilo, que às vezes até falava embora
não se entendesse o que dizia

Se havia perdido alguma coisa e vivia procurando, esse era um segredo
dele

Cada um terá o seu.

Não teria sido isso, nele, um segredo que não se podia medir, me
pergunto, o que eles não perdoaram,

e por isso o arrastaram naquela tarde para mostrar que todos, todos
somos iguais, homenzinhos, mistérios que têm que ser revelados custe o
que custar para que tudo, o humano, fique sob luz que faca possa contar,
que se possa negar quando for preciso fazer mais um morto?

Então.

Levado arrastado, o que era para rir era isto. Ele ia aproveitando para
continuar procurando, quieto, olhava o chão passando rente aos seus
olhos, queria usar até o último momento aqueles olhos, tinha ainda
esperança de achar o que havia perdido antes do fim. E não dizia nada.\\

Havia uma árvore no lugar para onde o levaram.

Veja isso agora: eles também tinham um segredo.

Pois fizeram aparecer lá uma mulher, taparam os olhos dela e a mandaram
amaldiçoar a árvore para que não desse mais frutos. Os frutos.

E depois, pregaram o homem na árvore.

Sumiro, o homem, ficou pregado lá

Então veio a hora da facas,

a primeira hora das facas

Haveria outras mais.

Foi nessa primeira hora que eles então tiraram o sexo dele.

Um homem pode deixar de gritar sua dor?

Então ele gritou. Ainda se ouve esse grito quando se passa por aquele
lugar à mesma hora, nos fins de tarde. Dizem. Eu não voltei mais lá, não
sei. Ele ficou pregado na árvore. E depois disso não veio mais nenhum
som dele.

Nada mudava ali.

Haviam levado Sumiro arrastado e agora ele estava lá, na árvore. Espere.
Tento lembrar

Lembro. O céu mostrava nuvens paradas. Não eram nuvens leves. Nelas,
havia um vermelho de sangue que não anda. Pesado. A gente ali sabia que
uma chuva ia cair, mas que chuva era aquela, parada, no alto, isso não
se sabia

Tínhamos o céu por cima e embaixo éramos só homens, mulheres e havia
também umas crianças ao redor da árvore.

Foi outra vez a hora das facas

Dos lados do horizonte veio vindo um barulhão.

Parecia que a chuva ia cair. Mas não caiu. Não caía. E essa foi a hora
em que eles tiraram os olhos do homem.

É que ele não havia parado de olhar o chão, de procurar, e era preciso
que isso acabasse. Não teria adiantado nada fazer aquilo com ele se não
parava de procurar.

Nós olhávamos. Não fazíamos nada.

As testemunhas.

Eles queriam que nem os olhos do homem, sozinhos, ainda procurassem,
rolando na terra, e, vimos, um deles tirou uma bolsinha, juntou os olhos
de Sumiro e guardou nela, fechados,

no escuro.

Como é que ele ia procurar agora?

Eu não sabia.

É que naquele tempo eu ainda olhava, tinha meus olhos e nada via. Via o
homem sem olhos na árvore e me perguntava como é que agora ele vai
procurar, sem entender nada. Mas ele ainda procurava, agora eu sei,
mesmo que estivesse tão no escuro quanto os seus olhos guardados na
bolsa do outro.\\

--- Não enjoe. Parece que é com os olhos da gente, não é?

Mas eram os olhos dele, não os seus, lembre-se

O homem olhou pela janela e disse:

--- Só pode ser artista quem tiver uma visão original do infinito.

Foi Schlegel quem disse

--- Não sei, disse o cego. Não sei nada. Eu estou aqui. Cego.

Só posso lhe dizer isso que lhe digo, estamos todos juntos nesta vida,

homens e deuses,

esses que nos fizeram de terra

Os da Água, os do Ar e os do Fogo

E os da Ausência.

Agora eu sei isso\\

No vento estava passando outra vez a voz que dizia\\

--- Vem Curau. Vem levar os homens para os teus jardins\\

E o cego disse:

Continuo.

Mais tarde já tinham ele nu, haviam arrancado as roupas, queimado. As
mulheres não viraram as caras, aquilo era uma outra vergonha

Lhe cortaram as mãos, na terceira hora das facas. E isso aliviou nele
uma dor, uma dor apagando outra dor, veja: a dor que ele sentia antes,

a dos seus olhos arrancados

Mas mal eles viram a nova dor nascendo e a dor mais antiga sumindo no
rosto de Sumiro, fizeram uma outra dor. Meteram um prego grande, Negro,
na sua boca. E assim ele não podia mais falar, mesmo não tendo dito nada
durante tudo isso e não fosse mesmo dizer nada enquanto aquilo durasse.
Isso eu sabia\\

E durou.

Olhando, mais tarde, não se podia saber onde ele estava acabando e onde
começava a árvore.

Isso foi assim.

É que eles tinham aberto o tronco, na quarta hora das facas, e metido o
homem dentro da árvore,

uma parte aparecendo, a outra oculta.

Queriam desse modo fazer dele cada vez menos um homem. Tirar dele todo
mistério. Fazer dele árvore.

Cada um entenderá como quiser. Sim. É. Cada um é um outro.

Mas veja, era o que queriam.

É como eu entendo.

Eu estou aqui, cego. Aqui é em toda parte

As vozes da terra vêm de longe

Um inseto me olhara

O que há para ver?\\

Ficamos lá. Eu olhava. Os outros também olhavam.

E lá se iam os dias não sei para onde

Estou lá. Olho. Não fazíamos nada.

Tinham o homem na árvore.

Decidiram dar comida a ele para que durasse mais. E aquele homem então,
ex-nosso vizinho, Sumiro, aceitou a comida deles.

As nuvens continuavam no alto, a chuva não caía, aquela chuva, e ele não
morria, aquele homem na árvore, nem deixava de ser homem nem virava
árvore de uma vez. Nem tinha nada para nos dizer. Olhávamos\\

No dia em que quis pôr para fora a comida, não pôde. Tinham fechado a
saída.

Teve que sofrer também as dores comuns.

Então parecia ainda mais um homem.

Isso eles notaram. Viram que não deviam dar comida a ele, e não deram
mais

Estava ali, perto. O homem na árvore. E no entanto bem longe daquele
lugar alguma coisa mudava nele, ficava nos seus sonhos onde nada daquilo
estava acontecendo

Nele, nos sonhos que tinha, às vezes deviam entrar uma ou outra as vozes
deles, de guarda junto à árvore

Mas isso acabou quando veio outra hora das facas e furaram os ouvidos. E
o deixaram inteiramente só, num branco,

e se fez um silêncio como não se sabia que existisse.

E assim ele já não podia mais ouvir quando novas horas vieram para as
facas, em ondas, umas após outras, soando,

e arrancaram suas pernas para que não andasse humano nunca mais,

e isso, acho, já lhe acontecia como se fosse a um outro, numa outra
história. E nessa história eles queriam talvez arrancar dele a sua alma,
a sua ave. Para trancá-la numa caixa de madeira. Madeira da árvore.
Árvore em que seu corpo desaparecia como coisa humana. Mas na caixa
aquela alma não ia parar de voar, voaria como antes no corpo do homem. É
que ele e a árvore já eram então uma só coisa.

No alto, o céu se mexeu. Era o sangue dele que finalmente ia cair?

Era aquela chuva. Que ameaçava cair e não caía. Não caía. Não caiu nem
cairá.

As horas das facas iam e vinham\\

Tiraram a cabeça dele para jogar com ela. E ele entendeu o vazio

Já havia entendido a escuridão, não era?

Já havia entendido o silêncio

E agora entendia o vazio.

Os que faziam guarda ao redor da árvore se divertiram, riram,

é que o homem é sempre uma criança? Me pergunto. Riam e atiravam a
cabeça uns para os outros. Era para passar o tempo, ali. Jogavam ela de
um lado para outro, a bola mágica, voava entre aqueles homens,

e isso sem que ele se sentisse humilhado com nada.

Mas dias nasceram. Morreram\\

Tome mais um café. Tome. Já vai acabar. É preciso que acabe. Agora
estamos indo para a parte onde ela é uma outra, a vida\\

Ouça.

Eles trouxeram a mulher que havia feito Sumiro. Puseram ela na frente da
árvore. É que então tinha descido sobre aqueles homens um desespero. O
de ter perdido Sumiro nesta vida. Queriam trazê-lo de volta. Onde
estaria ele agora?

Se reuniram então e falaram. Embaixo de uma outra árvore. Afastados.

Depois voltaram.

Vamos fazer assim, se disseram. E fizeram. Para que Sumiro voltasse,
foram buscar onde achavam que ainda estava, e arrancaram a roupa da
mulher, e procuraram no corpo dela

E tudo diante da árvore para que ele visse sem os olhos.

Trouxeram também um menino e diziam, É filho.

Procuraram sinais de Sumiro nele.

E trouxeram uns outros, assustados, e diziam, São irmãos.

Fizeram com todos a mesma coisa. E outras. As sem nome.

Sem nome foi também o que acabaram fazendo com ele.

Como não queriam ficar sem Sumiro,

se transformaram, ali diante de nós, em homens-areias, secos, e beberam
o sangue dele,

se transformaram em homens-cachorros, tinham raiva, e roeram os ossos,

e na volta aos homens que eram outra vez, trabalharam, trabalharam,
organizados, exatos. E vimos que haviam feito com a pele de Sumiro uma
tecida rede, e secaram ela sob um sol violento que surgiu no céu por
entre as nuvens daquela chuva que não caía não caía não caiu nem cairá e
se mostrou para os nossos olhos. As testemunhas.\\

Foi assim.

E uma roda girou. Lentamente

Primeiro para a esquerda.

Depois para a direita.

Depois ficou indo uma vez para a esquerda outra vez para a direita, como
uma dúvida. E a cada volta que ela não completava, víamos

Nasciam as frutas na árvore.

Então dançamos. As testemunhas.

E cantávamos. Batíamos palmas e bebíamos bebida amarga, rodando em torno
dele enfim inquieto, ou sereno, ou desesperado e sem saber por quê

\breakk

\vspace*{4cm}

Agora vá embora,

disse o cego ao homem depois. E o homem foi

Eu quero ficar só outra vez.

Ficar aqui. Ele disse cego.

\breakk

\vspace*{4cm}

A memória.

É ela outra vez. Mas é a memória de um outro nesta voz que vem falar a
Jacinto,\\

e vem no vento\\

--- Quando nosso irmão nasceu o relógio da sala parou. Eu lembro

Aquela ficou sendo a hora marcada.

Eu lembro

Nosso pai disse, Dele, deste filho que nasce de mim um dia virá a
alegria. Disse isso e empalidecendo, de repente mais velho, se afastou,
se arrastou para o seu quarto e tem se mantido mudo desde então,

uma outra máquina quebrada nesta casa.

Aquele pai.

Era um menino. Não chorava. Não ria.

Só olhava para nós, distante. O Distante. Estava lá entre nós e não
estava. Era se como continuasse no lugar de onde veio

Ele veio da nossa mãe, diz um dos irmãos que eu tenho.

Mas eu sei que veio de mais longe quando vou vê-lo.

Ainda é o nosso recém-nascido. O Eterno. Nunca cresce. E os anos passam.
Esperamos

Nosso pai disse aquilo. E agora estamos à espera da alegria.

E esta espera vai nos tornando assustados. Tudo é um espanto. É essa
espera

Um rumor, um vidro se quebra

Os desastres, os dias. Não é isso a alegria.

Fomos ver o copo quebrado na cozinha e não era ela.

E então, se nada aconteceu, temos procurado por toda a casa um indício
da vinda dela.

Mas nunca estamos de acordo.

Virá num dia dos mortos, diz minha mãe. A alegria.

Uma irmã quer que venha num dia de festa.

Temos esse pai e ele procura nos bolsos das roupas, revira tudo, quer
achar o outro lado das coisas, uma carta perdida, uma nome anotado que
não lembra onde guardou. E seus dedos tremem. Tem medo de tocar de
repente a coisa e isso não se dará num dia como os outros se ela vier,
quando vier e um de nós a achar.

Eu sei.

Nossa alegria virá quando ele começar a crescer

Ele fica lá no quarto. E não cresce. Esperamos.

Rondando o lugar onde ele está, deitado, está no fundo de um poço de
onde queremos tirar alguma coisa

Um de nós tenta adivinhar. Faz jogos. Nada. E se impacienta.

Às vezes um outro chora.

Minha mãe grita.

Na sala, o relógio ainda está parado. É o primeiro movimento dele que eu
espero. Será quando os ponteiros voltarem a andar que terei o aviso. Ela
vem. A alegria

o menino continua no lugar de onde veio. E por isso eu fico aqui também
parada, na sala. E olho o relógio.

No quarto o nosso recém-nascido, eterno, ainda espera.

Ele é quem sabe quando virá. A Alegria.

Só espero que ela venha antes da nossa morte

\breakk

\vspace*{4cm}

Lá fora agora é noite.

--- Esta noite não passa, diz o cego na janela.

Quando é noite, lá fora Jacinto acha que está ouvindo asas vermelhas
vindo no céu. Ainda agora ele ouviu de novo. As asas

Elas vêm

Elas estão vindo

Não era o Curau de volta, porém.

E ele espera. Essa espera. E os anos passam

\breakk

\clearpage
\thispagestyle{empty}

\movetooddpage

\vspace*{4cm}

Escuto.

É do fundo da cabeça. E vem no vento:\\

Ontem voltamos à ponte.

Nela a madeira está partindo da sua forma de tábuas, estou partindo
desta forma humana que me deram, a madeira diz\\

Fica cada vez mais velha esta ponte. Ela diz estou caindo aos pedaços,

e quando passamos sobre ela um de nós também cai na água, lá embaixo.
Nem todos voltam à tona.

--- Há um mal com olhos de criança e dentes lá embaixo, dizemos uns aos
outros e rimos. Mas ficamos na ponte.

Há um mal lá embaixo, repetimos. E rimos.

E fazemos a ponte tremer pulando sobre ela.

Quando chegamos à ponte já está anoitecendo.\\

--- Eu olho esta ponte e isso me deixa triste, diz o velho.

Ele sempre vem conosco.

Eu vejo um de vocês cair lá embaixo e não voltar,

ele diz. E isso me deixa triste.

E se às vezes vejo um outro, os pulos de alegria que dá por toda parte,
fico alegre também.

Vejam, não somos mais que isso, e isso vai sendo mudado pela vida.

Tudo vem de fora e entra.

E já não se é o mesmo, mais tarde.

E há também as coisas que saem de mim e entram na vida.

E seguimos em frente.

É assim.

Vivemos. É talvez o pior.\\

Quando o velho fala, sem sempre se entenderá o que ele diz.

Escutamos. Ele falava. Depois se afastou de nós.

Fica lá sentado na margem, só, e olha a água passar sob a ponte.\\

É quando a tarde está acabando que vamos à ponte. Reunidos numa rua da
cidade, de repente andamos, para a ponte, para a ponte

E uns já desaparecem pelo caminho, talvez tenham ido para as suas casas
talvez não não se sabe\\

outros esperam para cair da ponte quando chegarmos, que é uma outra
maneira de sumir.

Nem tudo porém são esses desaparecimentos.

Acontece também de alguns desaparecerem só em parte. Um acidente. E
temos um mutilado para rir, terá de andar agora numa perna só pela vida.
Isto pula. Já não é um homem inteiro. Nós somos assim. Nos fazemos aos
pedaços. Ficamos debruçados na ponte olhando a água escurecer\\

Hoje, até agora ainda nada aconteceu. Esperamos que a noite esteja aqui
plenamente

Então aí vai um de nós.

Entra na água agora.

Vemos ele entrar lembrando como outros antes também entraram na água
escura pela última vez

E eu digo:

--- Pedras. As pedras.

E vamos pegar as pedras. Elas estão por aí, pesam pela vida toda. Mas
ficarão leves no ar

E dou o exemplo, no que os outros já me imitam e o que entrou na água
sai correndo, foge, tem sangue na cabeça, vai ficar sentado também lá na
margem ao lado do velho. Há um outro lá. Olhamos. Agora na margem estão
aqueles dois.

Estamos nesta ponte.

Esperamos. E vemos lá na margem aqueles outros.

Estão sós. Lado a lado. Olham agora a água passar, escura

Na margem o atingido pelas pedras chora. Um apedrejado

Nós somos assim, não há nada a fazer\\

Então: Está para acontecer, diz um de nós agora.

Sim

Isso eu senti neste instante também.

Lá na margem, aquele que apedrejamos pára de chorar. Levantou os olhos.
Vê ao seu redor as árvores paradas, não há vento. E ele corre, vem para
estar novamente junto de nós, em cima da ponte, nos apertamos uns contra
os outros e esperamos o que virá. Mas não saímos da ponte. E só o velho
fica lá na margem olhando a água que passa

Nunca sabemos com qual de nós será.

Depois estaremos prontos para voltar amanhã, e nos outros dias da nossa
vida também.

E para ir embora daqui.

E já está acontecendo

Um de nós não tem mais onde pisar, onde manter a vida à tona, a madeira
se quebra embaixo dele, nada faremos para agarrá-lo. E as águas passam
sob a ponte. Somos assim. Não há nada a fazer. Desta vez era ele então

Já se foi. Um de nós.

Olhamos. E a água está calma.

E voltamos a nos afastar uns dos outros. O amontoado se desfaz. E já
somos cada um por si nesta vida que alguns chamam a ponte. Outros, a
água. Outros, um afundamento.

Amanhã é dia de vir outra vez, no fim da tarde.

Na margem o velho se levanta.

Voltamos.

Ninguém diz nada. Voltamos sem falar, dentro da noite, e nos despedimos
nas esquinas, para casa. Para casa. E sumimos nas ruas. Não há nada para
lamentar

\breakk

\clearpage
\thispagestyle{empty}

\movetooddpage

\vspace*{4cm}

O vento com a história de Sumiro que o cego contou já deu a volta a toda
a terra

e agora está passando outra vez pela janela de Jacinto\\

--- Eles vieram e o levaram, arrastado. Veja, a terra por onde o
arrastavam já ia escrevendo no corpo dele esta história que agora eu lhe
conto escrevendo ela também no seu ouvido

Diz o vento, só uma rajada dele que vem e passa pela janela de Jacinto
trazendo um pedaço da história de Sumiro, que permanece no ar depois que
o cego a contou.

E já passou. E daquele pedaço de voz de vento não fica mais nada\\

Uma outra voz, depois, está vindo dizer:

Do lugar onde estou, só posso ver o horizonte e um rio. Esta casa eu
mesmo fiz

Há um homem no rio. Ele me vê. Está vindo para cá. Conversamos. Me conta
a sua vida, e depois este homem está me dizendo para ir com ele.

--- Este rio vai para lá, aponta.

Diz:

--- Vamos.

Eu sei que esta ilha não existe, mas não devo ir. Não vou.

E quando ele rema outra vez para longe, desvio os olhos, tenho medo de
ver como o horizonte vai devorá-lo. É assim que a floresta agora também
está se fechando sobre mim. Aos poucos. Logo a noite virá a esta ilha. E
há animais que só então irão acordar. Eu sei que esta ilha não existe.

Do lugar onde estou só posso ver o horizonte e o rio. Vejo um homem lá.
Está vindo para cá este outro, ele também já me viu\\

Sim, Andara é mesmo a nossa África, da minha janela, cego, eu sei, diz
Jacinto.

E quero ouvir o resto.

Escuto.

O resto\\

Escuto.\\

Saindo da floresta, vi a casa.

Aquilo, a casa.

Um animal astucioso que vai abrir uma porta e olha para dentro de uma
casa com olhos selvagens como eu olhei não se encontra entre os homens
todos os dias

Entre os homens a minha lei é um pé na frente e só bem mais tarde o
outro guardado atrás avança. Não me arrisco. Fico pronto para escapar.
Eu sou assim.

Foi desse modo que então abri a porta e entrei

Ismael não estava em casa. E tive todos os gestos que quis para armar a
armadilha. Depois, saí. Fiquei do lado de fora

Atrás de uma árvore, eu queria estar lá para ver tudo. Oculto. E agora
espero a volta de Ismael.

Mas ele me vê quando chega.

Tem bons olhos esse Ismael, digo, mas não verá a armadilha. Tive todos
os gestos que quis

--- Não vejo mais rancor nos teus olhos, ah

me diz ele

Entra.

E entramos.

Ismael vai na frente. Esta casa é sua

--- É tua também, quer começar a me dizer

mas logo um estalar vem se misturar à sua voz e a interrompe, estala
outra vez e então ah tenho Ismael no ar. Usei cordas fortes. Grossas.
Suspenso assim ele é uma ave sem asas, muda. O espanto.

Ismael, Ismael, grito. Agora tu estás acabado.

E voava pela casa ao redor dele. Batia palmas. Ri e dancei, assim,
esquecido, me arriscava, sem Ismael no meu caminho posso deixar meus pés
irem para onde quiserem, dizia,

e também o resto da armadilha entra em ação, o resto que eu armara para
ele escapar talvez do primeiro mas receber um segundo ataque. Se Ismael
tivesse asas suas asas sangrariam,

e a armadilha me levou também para balançar no ar ao seu lado. Pendurado
junto com Ismael, agora somos duas aves. Isso não devia ser assim, digo

Não devia.

E Ismael agora está dizendo, É preciso que tu acordes. Escuto a sua voz,
longe

Tudo isso, vê, faz mal a nós dois, ele diz

--- Sim, respondi.

E acordei.

Mas agora eu esfrego os olhos sentado na rede e lembro o sonho, e este
sim quer se transformar em não. E eu quero é sonhar outra vez\\

--- E por isso o arrastaram naquela tarde para mostrar que todos, todos
somos iguais, homenzinhos, mistérios que têm que ser revelados custe o
que custar

É o vento. Nova rajada dele está passando pela janela de Jacinto outra
vez.

Depois, cessa

E cego se diz reconheço essa voz.

A história de Sumiro não quer se extinguir no ar?\\

Vem no vento. Escuto. Cego\\

Quando se está com medo dos homens é preciso ir à casa de um amigo

Por isso, esta manhã procurei Fabiano.

Andamos no seu jardim.

Ele me mostrava seus animais, as sombras que seus corpos faziam, tudo
isso esteve lá em torno de nós

Na volta para esta casa onde me oculto, oculto, mais tarde eu já não
lembrava o que fizemos juntos.

O que se esconde da luz dos dias?

E o que se esconde na luz dos dias.

À noite, vejo mãos sujas de sangue. E revejo tudo num sonho

Fabiano me deixando matar todos aqueles animais. E ele me dizia, Quando
o teu medo voltar, também podes voltar. Eles renascerão\\

Jacinto escuta.

É mais uma voz. É do fundo da vida:

--- Hoje eu vou tentar outra vez

Lentamente, terrivelmente, desta vez eu vou embora daqui

--- Eu vou embora,

grito isso com o meu grito mais vermelho e ele não cessa nem quando dou
com a boca torcida por um riso no espelho onde me vejo pendurado para
não me sentir só nestes dias parados.

É as ruas o que eu quero E desde já estremeço ao me ver, antecipando, e
as coisas que irão me acontecer quando estiver nelas. As ruins e as
outras. Todas elas. E não me falem do medo

Repito. Eu vou embora daqui.

E já me apresso.

Mesmo que haja um risco, o de sair pelo fundo da casa e só chegar ao
quintal

E ficar preso entre os muros, se perder a direção da porta da rua.

Para isso, devo ter uns olhinhos vivos. E para que não aconteça outra
vez, manter a direção. E ir daqui. Agora. Os pés já se movem. Estou
indo. Para as ruas. As ruas. Mas a porta da rua só parece me levar de
volta para o quarto. E até mesmo para o mais fundo dele e sob os olhos
daquele que não me perde de vista e ri, no espelho, mais ainda, para
baixo da cama. Onde me meto e estou bem agora.

Amanhã vou tentar outra vez\\

--- Nós olhávamos. Não fazíamos nada. As testemunhas, diz o vento

--- E cantávamos. Batíamos palmas, diz o vento

--- E bebíamos bebida amarga, diz o vento\\

passando novamente pela janela de Jacinto com a história de Sumiro.

E esta última rajada que agora passa está indo outra vez para longe.

O vento vai dar de novo a volta à terra inteira

O vento está indo contar a história de Sumiro a outros homens

\breakk

\clearpage
\thispagestyle{empty}

\movetooddpage

\vspace*{4cm}

De uma ponte

De uma ilha

Me falavam de uma ilha que não existe, se diz Jacinto. Quanto tempo já
se passou desde então? Os anos\\

Temos Jacinto, e Jacinto é o homem na janela

\breakk

\vspace*{4cm}

Há um outro tempo para a infância agora.

Foi por aí, a infância, esse tempo de espantos por toda parte, que isto
começou. A vida

Se diz Jacinto na janela\\

E uma nova voz está vindo no vento. É a voz de Jacinto um menino e nessa
voz em que o vento traz a infância, Jacinto vai voltar\\

Lembro Andara como era antes, diz a voz.

Ah a infância, se diz Jacinto.\\

Foi em Andara que esta cidade, Santa Maria do Grão

Andara é um lugar que mete medo

Andara foi a primeira parte da cidade que apareceu. Depois a cidade foi
aumentando e hoje Andara é um lugar quase esquecido, uma lembrança para
afogados,

ficou lá na margem de um rio,

há águas fundas, lentas, elas passam.

Andara é um amontoado de casas que dão portas e janelas para a floresta.
Você sai de casa e mal saiu já está sendo rondado por dentes, olhos

Ouve-se respirações. Um medo vem e vai ficando imenso. No alto, está a
lua, sempre. Branca, se ainda não é noite. Então se volta e se põe
trancas. Nesse dia não se sai mais de casa Andara é isso. Foi onde esta
cidade, ou uma outra, talvez a oculta, começou. Santa Maria.

Andara foi onde Santa Maria do Grão começou. No emaranhado.

Um emaranhado é sempre verde, me dizem.

Às vezes porém ele fica todo escuro.

Este não tem fim, seus rios que não existem e estas árvores ausentes ao
meu redor se estendem a perder de vista. Isto vai até onde um homem
puder ir. E vai mais longe ainda. Esta é a região. E Andara é mais:
Andara é o emaranhado inteiro.

Eu me pergunto, um menino deve saber tudo?

As histórias de Andara que tenho ouvido

Aqui também acontecem coisas.

Santa Maria do Grão também enlouquece.

Mas em Andara me dizem é pior. Lá se vai andando por uma rua e, então,
se vê, de uma janela estão nos olhando. Quem é aquele na janela que me
olha querendo não ser visto e querendo saber quem sou,

a pessoa pára na rua e se pergunta.

É só o vidro da janela, e no vidro um outro, depois se sabe. Você mesmo
é que estava lá e se olhava. Na hora, porém, não há como saber. Há um
outro ali, nos olha. E isso é tudo. E o outro é real\\

Andara é um lugar assim.

Como todos os outros lugares, todos,

me diz minha tia.

--- Isso tem um nome, ela diz.

Às vezes porém ela, a loucura, também pode vir se mostrar com uma cara
mais de caretas, risos indefinidos e também dá um desespero de rolar no
chão com gritos de se ouvir longe

Mas em Andara. Aí está essa tia me contando coisas.

--- Andara, diz ela, está lá. Entre a floresta de um lado e um rio do
outro lado, com uma vocação para a morte que não se vê por aqui. Está lá
com essa vocação e um querer ir sempre para um mistério mais fundo, e
outros, os sem fundo. Foi isso que fez Irido ir embora, por isso ele foi
morar lá. Nele também havia essa vocação para a morte.

Irido é o homem que a deixou.

Então ela conta a história do cemitério.

Em Andara foi que esta cidade teve o seu primeiro lugar de guardar os
mortos, diz. Se a pessoa não abrir todos os olhos e os do instinto de
fugir da morte também vai acabar se perdendo, entra numa rua que não
conhece e sai numa outra que nunca viu antes, e se atormenta, e sente
que está sendo puxada, levada, e então de rua em rua ela vai parar no
cemitério de Andara,

que já foi invadido pela floresta, que se mistura com ela, onde os
mortos e os vegetais estão juntos,

e a pessoa fica andando entre as casinhas de terra dos mortos, diz minha
tia, e ouvindo como a terra tem vozes antigas por ali, rumores. Depois
de dias e os dias não passam, entre os mortos, uma pessoa a quem isso
acontecer nunca mais será ela mesma.

Irido foi para lá porque não me queria mais, ela repete todas as vezes
para acabar a história. E se cala. A mulher.

Irido. O homem.

--- Tio Irido quer que eu vá para lá com ele, eu disse a ela um dia.

--- Não. Não vai.

Isso ela me diz assim, seca.

E essa tem sido a resposta quando falo que o homem quer que eu vá para
ficar lá com ele. Aquele homem lá em Andara.

Ele, porém, insiste.

Manda pedir que eu vá. Tem pedido muito.

Ainda hoje veio um homem de Andara, entrou na nossa casa e disse:

--- Irido quer saber se o menino vai ou não

Naqueles dias eu tinha as minhas irmãs e os olhos delas, de medo

Ficaram com mais medo ouvindo o homem falar, mais do que já tinham desde
que o nosso pai morreu nossa mãe morreu e ficamos assim, vivos

O homem jantou em casa e foi embora.

Minhas irmãs têm medo que o Andara me leve para o cemitério

Para elas, Andara não é só aquelas casas vazias lá, aquele começo de
cidade esquecido. É o Andara, e ele pega a gente, leva para a morte. E
aí não há mais volta para quem o Andara leva, elas dizem. E me olham. E
murmuram. E choram. E me olham como se já tivesse acontecido. É assim
que se dá com quem tem medo. A gente antecipa

Deve ter sido por isso que a minha tia acaba de decidir.

Agora ela deixou eu ir.

Ela também deve ter estado antecipando o que iria me acontecer se eu
fosse, só deixando poderá se livrar disso

Fui.

A infância. Ela é bem esse tempo de espantos por toda parte. E ela não
acaba nunca, eu sei.

Lá estava aquele homem.

Meu tio, que tinha uma noite nos olhos.

Me abraçou. E me levou para comer a carne de um animal que havia
apanhado, apanhei esta manhã, me disse, para te esperar. Aquela noite
nos olhos dele

Na mesa, enquanto comíamos, me disse amanhã, se tu gostas da carne,
iremos apanhar um outro.

Come mais.

Tu gostas?

Falava do animal e me dava mais para comer

Hoje, porém, quando acordei, tive medo de ir com ele apanhar outro
animal. E não fui

Como tudo passa rápido em Andara.

Tenho pedido tio me diga que animal é esse, quando comemos, todo dia ele
apanha um e traz para casa. Esta casa não existe, é muito velha, tem
paredes furadas e através delas podemos ver, lá fora, as árvores. E tudo
que vive nelas. Vamos ver o animal de perto, responde meu tio. Mas não
vou. Nesta casa comemos todo dia a carne do animal, que agora meu tio
põe outra vez na mesa. O jantar.

--- Hoje, e hoje tu queres ir comigo apanhar um?

Todas as manhãs me pergunta. E sai.

E volta com um animal. Já morto. E sem pele. E cortado em pedaços, é
para que eu nunca saiba que animal é.

É uma carne boa. Não há outra igual

E hoje?

Nunca irei com meu tio acho.

E também não saio desta casa para não me perder nas ruas e ir parar no
cemitério.

E hoje? Me perguntou esta manhã outra vez, antes de sair para apanhar
mais um animal.

É no cemitério que apanha todos eles.

Na parte do cemitério que vai sendo invadida pela floresta, ali onde a
floresta está bem viva, e avança sempre

E hoje?

Não respondi. Já não respondo mais. Ele entende que quando não respondo
estou dizendo não, não vou. E sai de casa

E hoje? Ele me pergunta sai.

Não levanta os olhos para fazer a pergunta. E tem sempre a noite
naqueles olhos. Eu sei.

Não vou, nunca irei

E hoje. Me pergunta.

O que eu não quero é ver o animal vivo e, depois, morto. Nunca
esquecerei o seu gosto porém

\breakk

\vspace*{4cm}

Fim para a infância agora.\\

E essa voz que diz no vento\\

--- Vem Curau. Vem levar os homens para os teus jardins\\

\breakk

\vspace*{4cm}

Mas uma outra voz está chegando.

Que diz:\\

--- Em Santa Maria do Grão quem passa por aquela rua não sabe porque
olhando a casa vê uma gaiola. Quem passa olha a casa. Para não se
atormentar mais com isso, quer encontrar uma explicação, e pára, olha
bem e depois se afasta e diz aquilo é só uma ilusão, vem das grades que
a casa tem.

E a impressão vai junto, acompanha quem viu a casa durante dias.

Mal se distrai, lá está de volta a casa. Vem na memória, e onde foi que
eu vi essa casa, quem passou se pergunta\\

A infância não tem fim

\breakk

\vspace*{4cm}

Espero a volta da ave. E enquanto isso escuto as vozes

As vozes da terra vêm de longe. Para ouvi-las basta se deixar ficar, não
ir embora nunca. Eu fico. O vento trará todas elas\\

Há outros por toda parte? Não ouvirão como eu estas vozes?\\

isso havia dito Jacinto e aquele homem fora embora, fora, voltara e
havia ido embora outra vez.\\

Isto ainda é um homem. Um inseto talvez me olhe. E isto insiste em
entender\\

Eu estou aqui. Cego.

Aqui é em toda parte\\

Diz Jacinto. E Jacinto é o homem na janela.

\breakk

\clearpage
\thispagestyle{empty}

\movetooddpage

\vspace*{4cm}

Se tudo continuar indefinidamente assim

por dias e dias e dias, um dia virá em que Jacinto já não será mais
Jacinto.

E haverá um outro na janela, ainda à espera

Bu. Um fantasma.

Uns ossos brancos

bu, um som para amedrontar crianças\\

De manhã, ele aparecerá na janela.

A janela estará muito velha.

Todos os dias, bu aparecerá na janela. Nela, a madeira não terá mais
idade, terá rachaduras que dirão somos as coisas mortas de um fantasma\\

E a vida, como será se o Curar não vier

\breakk
\pagecolor{black}

\chapter*{}
\pagecolor{black}\afterpage{\nopagecolor}


\movetoevenpage

\vspace*{4cm}

Bu, diz o vento passando agora por Jacinto ainda um homem.

Pois bu será Jacinto só se o Curau não voltar

\breakk

\vspace*{4cm}

Por muitos anos ainda Jacinto vai ouvi-las da janela. As vozes.

E os dias passaram velozes por ele. E não passavam

\breakk
\clearpage
\thispagestyle{empty}

\movetooddpage

\vspace*{4cm}

O homem estava lá, cego.

Quando o outro veio ele disse:

--- Ah você voltou outra vez. Me chame Jacinto. Eu estou aqui cego. Aqui é
em toda parte

Trouxe um menino desta vez. É seu filho, quer que ele escute também a
história do Curau. Sim. Você voltou

Eu conto, eu conto outra vez.

Sim. Em Santa Maria do Grão acontecem essas coisas.

Venha cá, menino. Você é novo para mim. Foi há muito tempo que o Curau
veio, você ainda não tinha nascido. E depois ele foi embora. Mas um dia
vai voltar. Você também precisa ficar sabendo como tudo se deu.

O menino ficou escutando.

O cego falava.

--- Foi há muito tempo. Mas eu lembro tudo. Eu nunca esquecerei.

O menino escutava.

Venha cá, disse o cego.

Deixe eu tocar nos seus olhos.

Deixou.

Os olhos. Estes. Os seus.

Não tenha medo. Toco. Sei.

Você é uma criança, as crianças não precisam ter medo do Curau. Que você
não seja mais um para correr pelas ruas quando a ave voltar. Tapando os
olhos, procurando um lugar para se esconder gritando o Curau, Curau. Lá
vem ele. Como eles gritavam. Tanto. Os gritos. E mais tarde nem fechavam
mais os olhos à noite com medo de não ter mais olhos para abrir de manhã\\

O menino ouvia e o pai quis olhar para a janela.

O cego contava

Tudo começou como eu conto agora. Conto para você outra vez. Eu fui
avisado antes dos outros.

A ave era toda vermelha. Estava lá, parada. Parecia doente. Dei a ela o
nome que quis, Curau. Foi o primeiro nome que saiu da minha boca. É só
assim, aprenda, que se pode achar o nome oculto de uma coisa oculta. E
aquela não era uma ave como as outras\\

O menino ouvia.

O medo só veio para aqueles que tinham as suas velhas razões para ter
medo, e esses passaram a ter medo então do Curau. Eles têm medo de tudo,
dizia Jacinto. E o menino ouvia. Quando a ave veio, aquele medo andava
pelas ruas com passos que só nos levarão a uma terra não-sagrada\\

Veja, tirando as crianças que têm medo reais, dizia Jacinto\\

Entenda, quando um menino grita o Curau, o Curau, ele só está fazendo
como os outros fazem. Imita. Se os adultos não tiverem medo, as crianças
também não terão medo da ave. Vão ficar nas suas redes, calmas, e as
noites serão sem espreitas\\

O Curau só furou os olhos dos adultos, quando veio da primeira vez,
dizia Jacinto\\

Todos devem pedir a vinda do Curau. A sua volta. Devem pedir todas as
noites, e dizer como quem reza vem Curau, e me cega. Me livra desses
olhos que não querem mais ver as coisas assim iguais

Só você não precisa pedir, menino, dizia Jacinto.

Pedir que o Curau venha cegá-lo.

Mas peça por seu pai.

Peça à noite antes de dormir.

O Curau é um bem que nos aconteceu, menino, e um dia ele vai voltar,
dizia Jacinto\\

Empurrado pelo cego, depois.

O menino se afastou.

Agora vá embora. Eu já disse tudo. Agora eu quero ficar só aqui, sentado
junto à janela. Cego. Disse Jacinto.\\

Na rua o homem dizia ao menino

--- Não

O menino quis olhar outra vez para o céu.

No céu havia nuvens, havia uma mancha. Era grande. Vermelha\\

Talvez agora um inseto me olhe para entender, diz Jacinto na sua janela.

Na sua janela, ele agora está dizendo, vem Curau

\breakk

\vspace*{4cm}

E essa voz que diz no vento\\

Ouvimos, ainda uma vez:

--- Vem Curau. Vem levar os homens para os teus jardins


\vfill
Fim de Os jardins e a noite\\

A viagem a Andara não tem fim