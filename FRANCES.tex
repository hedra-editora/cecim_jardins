\addcontentsline{toc}{chapter}{Les jardins et la nuit, \emph{traduction de Cláudia Vitalouca}}
\part*{Voyage a Andara oO livre invisible\\
\bigskip
\bigskip
\bigskip
\bigskip
\HUGE{Les jardins et la nuit}\\
\Large{Vicente Franz Cecim}\\
\bigskip
\bigskip
\normalsize{La liberté est une nuit noire?}}

%\chapter*{}

%\addcontentsline{toc}{chapter}{Tradução para o francês, de Cláudia Vitalouca}

\selectlanguage{french}

\forceindent{}\textbf{Que contiennent les animaux?}

Qu'est-ce qui contient les animaux?\\

L'étrange oiseau qui aveugle les hommes. Ne protège pas ses yeux. La
liberté est-elle une nuit obscure?

Une fois encore, bêtes et hommes

\breakk

\vspace*{4cm}

Le labyrantre.

C'est Andara. Andara c'est là où Santa Maria do Grão est née, comme on
le verra maintenant dans ces jardins, en faisant s'ouvrir la forêt, puis
en la faisant reculer, et c'est par Andara que la forêt revient.

Tous ces enfants ici

Andara est-elle le lieu d'un mythe entrevu?

Peut-être, mais pas seulement cela peut-être.

De toute façon, c'est là que s'enlacent, que se nient la nature et cette
autre chose inquiétante qui a un nom. Civilisation.

Tous ces enfants écoutent

La voix. Du labiantre

\breakk

\vspace*{4cm}

Seras-tu aussi attentif à ce qu'Andara n'est pas?

Andara c'est le voyage hors de soi et elle devra continuer à être ainsi,
un geste sans geste, elle sera ailleurs

Cela restera en blanc. Le vertige. Elle retire la terre sous nos pieds
et pourtant nous ne la perdons pas de vue.

Les yeux qui autrefois ont lu l'histoire du Nazaréen et le livre de
l'aveugle Dias, eux aussi lisaient un livre qu'ils ne lisaient pas.

Peut-être savent-ils maintenant, non sans surprise, ou déjà
savaient-ils? Que sous les livres il y a les non-livres.

Voyage à Andara, le livre invisible.

\breakk

\vspace*{4cm}

Bon.

Dans le labyrinthe, on doit dire aux autres: sans un texte, il n'y pas
de temps

Et ainsi, il y a d'infinies manières de lire Andara.

Un jeu de dislocations et, parfois, de rapprochements arbitraires, selon
chacun, et en chacun, le temps d'une émotion.

Une danse pour les intuitions.

\breakk

\vspace*{4cm}

Dans cette danse, la trame des livres d'Andara est abolie. Et on peut
commencer le voyage par le chemin qui s'ouvrira en premier

--- O fleur Andara, où les rêves

N'ont pas de limites.

Où tout est déréglé. Plus tard, une lecture d'Andara trouvera d'autres
combinaisons dans la mémoire. Voilà un idéal: l'imagination dans la
pénombre, libérée par l'absence physique d'un texte

Les événements curieux qui alors se produiront: qui vole à côté de
Caminá? Est-ce l'oiseau Curau? Et: existe-t-il un livre appelé Les
jours aveugles, où l'on raconte l'histoire de l'aveugle Dias?

Mais il y a plus: ayant arraché les pages de tous les livres,

et ce serait lumineux si cela arrivait dans un moment de révolte,

le lecteur, avec l'aide d'un vent inattendu qui d'abord égare puis
réorganise le voyage à Andara. Je m'arrête. Les fenêtres doivent rester
toujours ouvertes pour que les vents entrent sans cesse et qu'aucun
ordre final ne s'installe. Rigoureux.

\breakk

\clearpage
\thispagestyle{empty}
\movetooddpage

\vspace*{4cm}

Ce voyage vers la chose humaine la nuit\\

Et j'ai ouvertement voué mon cœur à la terre grave et souffrante et,
souvent, dans la nuit sacrée, j'ai promis de l'aimer fidèlement jusqu'à
ma mort, sans crainte, avec son pesant fardeau de fatalité, et de ne
négliger aucune de ses énigmes. Hoelderlin dit

Cette nuit une fois niée,

Il y en aura une autre

Et j'ai ouvertement voué mon cœur à la terre et souvent, dans la nuit
sacrée, j'ai promis de l'aimer fidèlement et de ne négliger aucune de
ses énigmes\\

La fardeau d'une histoire avec une histoire.

Comment serait une histoire sans une seule histoire, le fardeau éclaté
en morceaux?

Le non.

Le labyrinthe.

Le non. Non à une histoire unique. Même si elle est l'histoire d'un
homme. Faire en sorte que par elle passent autant d'histoires d'autres
hommes que l'oreille en perçoit et que le vent en raconte.

Le labyrinthe. Dans celui qui s'ouvre maintenant, il y a les voix que
cet homme écoute à sa fenêtre, réelles,

elles viennent de documents anciens sur la douleur et se mêlent à la
mise en scène de la douleur imaginée faite par la voix d'Andara\\

Alors, le voici.

Le non labyrinthe.

\breakk

\vspace*{4cm}

Une sorte de voix murmure dans la nuit.

C'est cela l'imagination.

Elle arrive. On ne sait pas bien d'où

Voilà ce que nous avons. Un homme à sa fenêtre. On ne peut pas l'éviter.
Et des voix. Il suffit alors de se laisser aller,

ce voyage parle de la vie et ne s'arrêtera pas avant la fin.\\

Une autre fois.

Ce sera comment?

\breakk

\vspace*{4cm}

L'homme était là. Aveugle.

Quand l'autre est arrivé, il a dit:

--- Je m'appelle Jacinto. Je suis ici, aveugle. Ici c'est partout.

Oui. Ce genre de choses arrivent à Santa Maria.

Tu es venu pour entendre l'histoire. On t'a dit là-bas que je sais
comment tout est arrivé, alors tu es venu. Ils t'ont dit où j'habitais.
Ils t'ont mené jusqu'à cette rue et t'ont montré la porte de cette
maison. Alors entre.

L'autre homme est entré.

Il est resté à écouter.

Pendant une partie de l'après-midi, l'aveugle a parlé. Et l'autre
écoutait.

Lui, l'aveugle, a dit:

--- Il n'y a pas si longtemps.

Et je me souviens de tout. Je n'oublierai jamais.

Ecoute.

Ça s'est passé comme je te le dis maintenant, avec ces mots. Tu peux ne
pas croire que j'ai été averti avant les autres, mais c'est vrai. Il n'y
pas de quoi le jurer. Il y a seulement cette voix telle qu'elle est, et
ne prête pas l'oreille au vent de la nuit., de cette nuit sans la
moindre lueur pour éclairer les énigmes, cela se tordra avec des lames
partout, ne prête l'oreille qu'à cette voix et laisse le vent emporter,
aussi, la part qui lui revient dans ce que tu vas entendre. D'autres ont
aussi besoin de savoir comment tout cela est arrivé.\\

Ecoute.

L'oiseau était tout rouge. Il était là, arrêté. Il semblait malade.

C'est ainsi que tout a commencé.

Je lui ai donné le nom que j'ai voulu, puisque les choses qu'on connaît
ont toutes un nom commun, mais les autres, celles qui viennent on ne
sait d'où, on ne le sait jamais, on peut les nommer comme on veut, il
suffit d'ouvrir la bouche et de retenir le premier mot qui sort, il n'y
a que comme cela qu'on peut trouver le nom caché d'une chose cachée. Et
celui-ci n'était pas un oiseau comme les autres, ça je l'ai vu tout de
suite.\\

--- Curau.

C'est le premier mot qui est sorti de ma bouche. Cette bouche qui de
cette façon a pris vie. Curau. Et ce nom~lui est resté depuis ce jour de
rumeurs et de changement, le mien, où j'ai ouvert la porte et qu'il
était là. Il était vraiment malade et ne s'est pas enfui quand je me
suis approché. Il a voulu entrer dans ma maison. J'ai pris soin de lui.
Très vite il s'est rétabli. Il était fort. Grand. L'oiseau.

Il faut que je te dise: auparavant, j'avais les yeux fermés, et j'ai
tout vu.

C'était avant de trouver l'oiseau. J'ai fait un rêve.

Quand je me suis réveillé, je n'ai pas compris ce que j'avais vu sans
mes yeux. Dans mon rêve.

Maintenant je sais que cela avait été un présage. J'avais vu les jours
de peur qui plus tard allaient s'installer parmi nous

Quand je me suis réveillé, j'ai revu ces ailes immenses,

des yeux de feu qui m'épiaient du fond de la nuit,

et dans le rêve que j'avais fait il y avait une nuit dans laquelle les
hommes couraient, fuyant avec les femmes par les rues vides, des rues où
les pierres n'avaient pas de poids,

fuyant comme ils allaient fuir aussi dans les rues bien réelles de cette
ville. En criant. En cherchant à se cacher.

En me réveillant de ce rêve, je ne pouvais pas encore comprendre, vivant
comme je vivais dans l'illusion que seul ce que l'on voit les yeux
ouverts est réel, et miroir trompeur tout ce qui nous apparaît quand on
ferme les yeux. Non je ne pouvais pas. Il y avait cet oiseau qui volait
au-dessus de la ville, il plongeait sur une rue et on entendait des
cris. Les cris.

Ce fut le début. Et je l'ai vu.

C'est ainsi que la vie m'avertit que le Curau allait arriver.

Et ensuite, vois-tu, elle, la vie, est venue poser le Curau exactement
devant la porte de cette maison

Plus tard, j'ai compris

Nous avons des yeux, comme j'en ai eu, pour regarder partout. Un miroir
nous aveugle en dépit du soleil.

Est-il encore là, tout là-haut?

Existe-t-il encore?

La nuit hésite et nous rejette avec ces yeux qui ne nous servent à rien

Tout s'est passé comme je l'ai dit.

Avant de voler parmi les hommes, le Curau a volé en Moi, entre les yeux
que j'avais, fermés, et la vie là dehors. Il a d'abord volé dans un ciel
humain.

Et alors j'ai trouvé l'oiseau et je l'ai emmené chez moi.\\

La peur n'est arrivée que plus tard chez les autres.

C'est plus tard qu'ils ont commencé à entendre ses cris. Alors, le Curau
volait déjà au-dessus de la ville.

Il volait et ils criaient, le Curau. Curau. Il est revenu. Et les enfant
aussi criaient.

Pourtant ils ne devaient pas avoir peur.

La peur ne vient que chez ceux qui ont déjà de vieilles raisons d'avoir
peur,

Ceux-là avaient déjà la peur en eux et ils se sont mis alors à avoir
peur du Curau. Ceux-là ont peur de tout. Et l'oiseau était la seule
chose maintenant visible qui faisait surgir leur peur, cette peur qui
allait par les rues au pas qui nous mènera, qui mène toujours, à une
terre non-sacrée. Ils allaient ainsi dans la contrée d'un dieu sans
Visage, et dans son œil gauche ils avaient planté une Epine.\\

Vois, sauf les enfants qui ont des peurs réelles

Non. Je veux dire cela d'une autre façon. Comme je l'ai compris

Quand un enfant criait le Curau, le Curau, il ne faisait que ce que les
autres faisaient, il imitait

Si les adultes n'avaient pas peur, les enfants eux non plus n'auraient
pas peur de l'oiseau

Ils resteraient dans leurs hamacs, calmes. Et le soir ils n'auraient pas
d'appréhension pour entreprendre sans effroi leurs vols enfantins

Un Curau n'attaque jamais, jamais il ne fait du mal à un enfant. Jamais
on n'a parlé d'une attaque d'oiseau contre un enfant. Il perce seulement
les yeux des adultes. Tu n'as jamais entendu dire qu'il avait aveuglé un
enfant.\\

Les autres cependant,

ceux-là ne dormaient plus depuis l'arrivée de l'oiseau. Les adultes. Ils
tremblaient la nuit. Ils avaient des frissons d'effroi. Une chose était
certaine: à tout instant l'oiseau pouvait entrer par leurs fenêtres et
les aveugler, dans ces maisons où ensuite ils erreraient, sans but, en
se cognant contre les objets\\

Et les jours passaient\\

Il y avait la peur. Elle était à l'intérieur. Et tout autour.

J'ai entendu des choses s'écrouler.\\

Eux ne comprenaient pas.

Un Curau ne fait pas de mal.

Il fallait même demander sa venue.

--- Il faut demander qu'il vienne, c'est ce que je disais à tous.

Je leur disais:

--- Tout le monde doit demander au Curau de venir.

Tout le~monde doit le demander chaque nuit et dire comme dans une
prière:

Viens Curau, et aveugle-moi

Délivre moi des choses immuables,

Mes yeux ne veulent plus voir ces jours monotones

Curau, fais-moi tomber

dans la nuit que je suis moi-même, je le sais,

pour que tout change

pour que je retrouve le goût de la vie

Curau,

c'est ce que tous devraient dire, les yeux fermés,

moi je ne vois plus rien

Je ne vois plus les autres hommes, il y a un masque sur chaque visage

Fais aussi que ces autres se perdent dans ta nuit, perdent leurs yeux
avec moi, fais-leur du bien

Aveugle la femme qui dort à mes côtés, aveugle ces hommes autour de moi,

eux aussi ne voient qu'un masque sur mon visage et ne peuvent voir,
comme je ne peux voir en eux, sous le masque, le visage que j'avais Et
que j'ai encore. Il est caché. Souterrain. Un espace sans lumière.

Viens, et je te le demande, ne vois pas le masque sur le visage d'un
enfant. Je te le demande. Parce qu'en eux vivent les jours

Laisse pour cela leurs yeux en paix

Et ne perce que les miens, Curau, en entrant par cette fenêtre avec ta
lumière noire.

Moi je ne sais plus voir\\

Voilà ce que vous tous devriez demander

C'est ce que je leur disais.

Ils devaient se réunir dans les églises et sur les places pour demander.
Et aussi demander seuls, comme le font les saints à voix basse

Demandez la venue de l'oiseau et attendez qu'il vienne, leur disais-je.

Et qu'il vous exauce vite.\\

Regarde. Moi.

Tu es entrain d'écouter un homme dont les yeux ont déjà été percés par
le Curau, dit l'aveugle à l'homme.

Et l'autre écoutait.

Maintenant ce que je sais c'est ce que je sais. Rien.

L'autre écoutait et l'aveugle a dit la première chose que l'oiseau a
fait c'est de m'aveugler. En finir avec les yeux que j'avais. Détruits.

L'homme en face de l'aveugle a regardé par la fenêtre.

Il a eu du mal à retrouver ses forces, continua l'aveugle, après avoir
mangé ce que je lui avais donné, il a bondi sur moi et alors est arrivée
cette nuit dans laquelle maintenant je suis plongée, ici. Ici c'est
partout. Cette nuit, j'attends ce qui va arriver, les jours sans nom, je
le sais

J'attends et je sais

J'attends et j'écoute. Les voix viennent dans le vent

J'attends et je marche en tâtant les choses. Et je comprends.

Je passe cette main sur le visage de quelqu'un et je comprends qu'il est
triste et d'où vient sa tristesse, je passe cette même main sur le
visage d'un autre et je comprends sa peur et d'où vient la Peur et où va
la Peur. Et de quoi est faite la Peur.

Alors je dis d'une voix lente pour ne pas l'effrayer

Prends patience attends un jour l'oiseau va revenir\\

L'homme devant l'aveugle a eu envie de regarder à nouveau par la
fenêtre.

Il y est allé, a écouté ce que l'aveugle disait.

Il a voulu savoir la suite, ce qui était arrivé ensuite.

Un jour il s'en alla, disait l'aveugle.

On ne sait pas où. Il a dû retourner d'où il était venu.

Il sera maintenant dans un nid inamical, dans un lieu caché dans une
quelconque partie de la vie, loin ou près de nous. On ne sait pas

Maintenant, à cette fenêtre, j'attends le Jour.

Le Jour où il reviendra, dit l'aveugle. Ce sera par un après-midi comme
celui-ci peut-être. Il écartera les nuages. Quand l'oiseau reviendra,
disais-je. Et alors. Les cris reprendront. Les fuites. Moi

j'attends ici.\\

L'homme a entendu ce que l'aveugle lui disait. Reste, tu viens de loin,
et attends son retour. Le retour de l'oiseau.

L'aveugle demandait à toucher ses yeux. Il l'a laissé faire.

Viens ici, disait l'aveugle.

Laisse-moi toucher tes yeux.

Les yeux.

Ceux-là aussi. Les tiens.

Tous pareils.

Les tiens sont comme ceux que j'avais. Ils regardent hors de la vie.

Délivre-toi d'eux aussi.

Ah, mais tu as peur. Je t'ai touché. Je le sais.

Tu en verras plus d'un courir par les rues quand l'oiseau va revenir.
Cachant leurs yeux, cherchant un lieu pour se cacher, criant le Curau,
le Curau. Il arrive. Comme ils criaient. Tellement. Les cris. Et plus
tard aussi tu ne fermeras même pas les yeux la nuit, par peur de n'avoir
plus d'yeux à ouvrir le matin. De ne plus rien voir.\\

Repoussé par l'aveugle.

L'homme s'est écarté.

L'aveugle lui disait pars maintenant. J'ai tout dit. Maintenant je veux
rester seul. Rester assis à côté de la fenêtre. Aveugle.

\breakk

\vspace*{4cm}

J'écoute.

Je suis ici. Aveugle.

\breakk

\vspace*{4cm}

Une fois l'homme parti, l'aveugle était resté seul à nouveau.

\breakk

\vspace*{4cm}

Et cette voix qui dit dans le vent\\

--- Viens Curau. Viens emmener les hommes vers tes jardins

\breakk

\vspace*{4cm}

J'écoute.

Je suis ici. Aveugle.

Ici c'est partout.\\

C'est la voix de l'aveugle qui parle. Mais il y en aura d'autres, c'est
dit\\

J'attends le retour de l'oiseau. Et j'écoute les voix, dit-il. Les voix
de la terre viennent de loin. Pour entendre les voix de la terre il
suffit de se laisser aller, dit l'aveugle près de sa fenêtre.

Il dit, ceci est un homme. Un insecte me regardera sans comprendre.

Il y a toutes ces rumeurs\\

Maintenant l'aveugle écoute.

Et il dit: C'est du fond de la tête que me viennent ces histoires. Tout
vient dans le vent aussi

pour ceux qui insisteront à avancer dans dette nuit, et elle, la vie, à
son tour avancera aussi. Parfois elle crie. D'autre fois elle murmure

\breakk

\vspace*{4cm}

Maintenant

c'est l'heure où les ombres s'approchent des choses.

D'ici peu, personne ne pourra plus voir ses pieds, qui buteront sur les
objets, hésiteront dans les maisons impuissants à savoir où diriger le
corps. Les bras aussi s'effaceront et on pourra les étendre en avant
sans voir ce que les mains iront toucher, et tous ainsi, mutilés,
découvriront avec surprise, bien que cela arrive tous les soirs, qu'ils
disparaissent. L'obscurité envahira aussi tous les yeux jusqu'à ce qu'il
ne reste plus rien à donner à la nuit. Mais ce n'est pas le Curau qui
revient, non. C'est seulement le Curau de la vie qui fait disparaître la
vie pour un certain temps à nos yeux

La nuit. Maintenant elle arrive.\\

Un cri. C'est un oiseau.

Il dit à Jacinto qu'à Andara tout cesse d'être humain. C'est d'Andara
que vient ce cri qui arrive à sa fenêtre, à Santa Maria.

Santa Maria do Grão le soir.

Maintenant la peur, la Peur va sortir d'ici et courir urbi et orbi
emportée par le vent.\\

A sa fenêtre Jacinto attend

Un bruit de chute arrive dans le vent.

Cela commence, dit Jacinto à sa fenêtre. Et il attend.

Il y a un homme à sa fenêtre qui écoute

Et la vie lui parle

Arrivée avec le vent, elle, la vie, pour que moi qui porte le fardeau de
toutes ces histoire je l'écoute aussi\\

Plus tard, il entendra un aie

C'est le malheur

Il est en train de s'installer.

Alors Jacinto sent que tout recommence et que cette nuit va être une
nuit d'insomnie, et que seul lui, l'homme à sa fenêtre, va rester calme,
muet, tandis que tout parle et que les autres parlent

Il arrive dans le vent:

--- Je te dis qu'il a arraché les yeux avec ses mains. Il est tombé, il a
roulé sur le sol et a appelé quelqu'un. La femme est arrivée. Et alors,
pour qu'il ne souffre plus, elle a utilisé la corde pour le tuer.

Une voix dit cela.

Et à sa fenêtre l'aveugle sait que certains ne résisteront pas jusqu'au
lever du jour.

Demain, quand la lumière reviendra, ce sera un jour de plus pour les
enterrements. Mais un jour les Ailes vont revenir et cela n'arrivera
plus, dit l'homme à sa fenêtre.

Il attend.

Et il sait qu'à mesure que la nuit avance,

la force du vent augmentera jusqu'à arracher les fenêtres, tirer les
hommes des hamacs, frapper à la porte comme pour entrer de force tandis
que de l'autre côté du bois s'amoncellent les choses, les tables, les
morts, les chaises, tout pour résister au vent. Et le vent ira aussi au
bord du fleuve et coulera les barques amarrées au ponton et renversera
les chandeliers et éteindra les bougies, et ne ménagera aucun refuge de
lumière où l'on puisse courir, poursuivis, tous, par une chose sans nom
qui vient du fond de chacun de nous et se trouve dehors aussi, et dans
les rêves de ceux qui dorment et tentent ainsi de s'échapper en feignant
de ne rien savoir, les endormis, qui ne sont pas vivants, et il entre
par leurs oreilles, et même au fond des rêves personne n'est sauvé car
maintenant le vent va faire en sorte qu'ils rêvent qu'il bouleverse tout
en eux, dans leurs têtes, embroussaillant leurs cheveux et, pâles, ils
se réveilleront en voulant fuir ici, dehors, vers la vie, mais c'est
ici, en elle, qu'un véritable enfer les attend et le vent agrippe les
femmes et veut soulever leurs jupes et arracher les enfants de leurs
ventres dès qu'ils sortent leur tête et regardent dehors. Ceux qui
tenteraient de naître cette nuit. Epient la vie qui s'élance, s'élance
mais vers quoi,

Se demande Jacinto à sa fenêtre.

Il a déjà vécu des nuits comme celle-là.

La vie pourrait être autre chose, dit l'aveugle. Mais seulement pour
ceux qui sauront attendre que le jour se lève.

Il se lèvera bientôt, pense Jacicnto à sa fenêtre.

Et il attend.\\

Cette nuit, d'autres voix viendront à nouveau dans le vent\\

Il écoute.

Elle lui diront:

Que maintenant les eaux du fleuve sont devenues folles, elles coulent en
sens inverse. Elles montent par là, signalent les voix

Et elles disent, Notre Dame illumine de ta lame de lumière cette nuit

Et elles disent aussi, O merveille, O merveille. Les hallucinations
arrivent au port, elles apportent un espoir, venez tous

Voilà ce que disent les voix à l'homme à sa fenêtre.

Et elles disent plus encore, elles disent, Un fils de dieu a été enlevé
par le vent et jeté sur le pavé de l'église quand il disait ses prières.
La petite église se lézarde, elle ne résistera pas à cette nuit

Je ne supporte plus d'attendre le lever du jour,

Disent les voix.

Et l'homme à sa fenêtre écoute. Il attend.

Peut-être l'une d'elles demande-t-elle maintenant qu'on sacrifie un
innocent, se dit Jacinto.

Elle dira, C'est pour nous sauver.

Ils boiront son sang s'il se répand.

Et plus tard ils allumeront des feux dans la nuit. Ils feront une fête.
Et ensuite ils tenteront de s'endormir, écrasés par l'alcool.\\

Le jour va arriver.\\

Cette nuit va finir, murmure Jacinto.

Alors arrivent dans le vent des voix plus anciennes,

parlent d'un autre temps de tortures.

Et le vent dit: Il y a des gens, condamnés à être tirés par quatre
chevaux dans quatre directions.

Oui, dit Jacinto.

Et il pense, il va bientôt faire jour.

Et avec le vent arrive, d'autres pays, cette histoire: Là-bas, dans un
pays inconnu, une femme en a tué une autre. C'est pour cela que
maintenant elle va mourir, maintenant c'est son tour. Elle va mourir en
face de la chaise où était assise l'autre femme quand elle l'a tuée.

Jacinto peut voir cela de sa fenêtre

C'est la vengeance. Dit le vent.

Il dit et menée sur le lieu où elle a tué l'autre, sa main sera coupée
et jetée au feu pour qu'elle voie. C'est avec la main droite qu'elle a
tué et c'est précisément elle qu'elle voit se transformer en cendres

Pour qu'elle voie, répète l'aveugle à la fenêtre.

Elle mourra du même couteau, dit le vent.

L'homme à sa fenêtre répète c'est la vengeance

Le même couteau, dit le vent.

Et il s'arrête.\\

Les voix qui viennent de Santa Maria do Grão dans le vent s'arrêtent
aussi. Et plus aucune voix n'arrive jusqu'à l'homme, pas même d'Andara.

A sa fenêtre il n'entend plus rien maintenant

Ensuite, le vent revient. Et un oiseau tout noir se fracasse sur le
visage de l'homme. Il sait que celui qui a jeté l'oiseau c'est le vent.
Et cela lui importe peu.

--- Ils ont tué Mariana, ils on tué Mariana, crie quelqu'un sous la
fenêtre.

A sa fenêtre l'homme ne croit pas cette voix.

Il sait que c'est un oiseau que le vent a jeté sur son visage. Et que
Mariana n'est pas noire. Il sait aussi que de nombreuses nuits viendront
encore pour qu'il ressente cette peur d'être irrémédiablement vivant. Et
il attend que le jour arrive\\

Le matin arrive maintenant

Je le sens venir.

Il vient de ce côté, entre par la fenêtre, va éclairer d'abord les mains
des hommes pour qu'ils puissent trouver un verre d'eau et éloigner cette
nuit. Ensuite, il éclairera leurs pieds pour qu'ils vivent un jour de
plus avant que les ombres ne reviennent toucher les choses blanches

Je suis ici.

Ici c'est partout, dit l'aveugle à sa fenêtre.\\

Le jour se lève.\\

Et si l'enfance venait jusqu'à lui maintenant?

L'enfance arrive alors

Et une voix vient dire dans le vent:

--- La constellation du chien aboie à nouveau

\breakk

\vspace*{4cm}

--- Fin pour l'enfance,

crie, muet, l'aveugle à côté de la fenêtre.

Il est là. Aveugle.

Et il attend le retour de son oiseau. Le premier mot qui est sorti de sa
bouche. Curau.

--- Nous devons veiller nos morts, dit Jacinto. Avant j'étais avec eux.
Maintenant je ne veux plus quitter ce lieu, avancer dans l'obscur. Les
morts sont des poissons partis pour d'autres eaux, on ne sait où\\

Et dans le vent une autre voix vient lui dire:\\

cette voix venait de l'autre côté de la pièce.

C'était comme un travail d'insecte.

Ensuite elle s'est arrêtée de parler

Les uns grands les autres petits approchaient leur visage de l'homme
couché pour un ultime baiser. Il a entendu des voix d'enfants. Là dehors
un oiseau de la nuit a chanté. La brise entre maintenant par la fenêtre
et frôle ses cheveux qui semblent retrouver un restant de vie aux yeux
de ceux qui sont autour de lui

Dans la pièce quelqu'un murmure

--- Alors passons la frontière, et il n'a pas écouté la fin de ce qu'il
disait

Pendant toute la nuit nous avons veillé dans la maison

Et ensuite le mort n'était plus là.

Mais ils l'embrassaient encore sans savoir qu'il était parti avec les
lèvres froides de l'aube. Les parents

--- Jamais nous n'avons été aussi doux qu'en cet instant, répétait la voix
de l'autre côté de la pièce. Dans mon pays, il y a des pécheurs qui
amarrent les morts par les pieds pour qu'ils ne se lèvent pas et ne
retournent pas à la mer, dit cette voix

\breakk

\vspace*{4cm}

Je suis ici.

C'est l'aveugle.

Ceci est un homme, dit-il. Et ceci veut aussi comprendre

\breakk

\vspace*{4cm}

--- La constellation du chien aboie\\

--- Nous n'avons jamais été aussi doux qu'en cet instant\\

Les voix errent encore dans l'air de ce matin naissant

Tandis que dans leurs maisons les hommes dorment, soignant dans leurs
rêves les blessures de la nuit passée\\

Et une voix arrive dans le vent jusqu'à Jacinto à sa fenêtre pour dire:

--- Andara. Là-bas, les rues sont toujours vides et Cela, la forêt,
avance, s'approche de plus en plus de nous\\

Jacinto a faim.

Ceci aussi est un homme, dit-il.\\

Andara c'est l'Afrique que nous avons en nous

\breakk

\vspace*{4cm}

Et si une autre voix arrivait encore?

Et cette voix arrive. Et elle dit:

--- La vieille. Elle mastique lentement. Elle aussi est un animal la nuit,

comme on m'a dit qu'était cet oncle Irido qui est allé à Andara et n'en
est jamais revenu. Je ne vois pas ses yeux, elle n'a pas d'yeux pour se
voir

Jacinto écoute et pense, Agora est une de mes sœurs qui me parle\\	

Et la voix:

--- Maintenant elle marche dans le jardin. La vieille. Un insecte la voit.
Ils se regardent. Il n'y a rien à comprendre

\breakk

\vspace*{4cm}

Des vents qui parlent, des vents qui parlent\\

Et cette voix qui maintenant vient raconter à l'aveugle à sa fenêtre

ceci:

--- Un homme est là et regarde le visage d'une femme qui dort et
l'endormie disparaît dans sa nuit. Mais sur son visage les choses
qu'elle a vécues aujourd'hui vivent encore,

et sur ce masque qui fut son visage les choses disent

j'ai vécu encore un jour avec ses douleurs et la joie qui revient à
chacun sur cette terre que les hommes appellent ma maison.

Ensuite, sur le visage de l'endormie se dissipe jusqu'au masque des
choses. Et alors il n'y a plus rien à regarder. Le vide.

--- Où a-t-elle pu aller, demande l'homme en regardant ce vide. Qui
commence à se changer en paysage,

car d'autres choses commencent maintenant à prendre forme sur le visage
de l'endormie.

La femme s'agite. Elle écarte quelque chose. Non. Dit-elle sans voix.
Non.

Et l'homme sait qu'elle rêve à quelque chose qu'elle ne veut pas,
qu'elle refuse.

Et ce qu'elle ne veut pas c'est ce paysage qui surgit, où l'homme voit
un enfant et l'enfant voit les figures qu'il aimait retrouver sur les
murs, les taches,

Pour y voir des animaux, des armes, un peu de sang coulant au début,
puis plus de sang encore, coulant plus fort. C'est l'homme et sa faim,
il veut un animal pour sa table, disait l'enfant en regardant le mur. La
tache. Il suffisait de laisser les yeux aller et se laisser aller, avec
eux, pour voir

des hommes derrière, à sa poursuite, et l'animal fuyant sans savoir où,
pris dans la tache, sur le mur. Le piège

\breakk

\vspace*{4cm}

--- La faim arrive à l'heure initiale de la vie, quand vient le matin et
que les yeux s'ouvrent\\

dit une voix et le matin arrive encore.

Ce matin, le temps stagne

Et dans leurs maisons les hommes dorment\\

Alors, plus rien d'autre n'arrivera en dehors de ces voix qui racontent
des histoires à l'aveugle à sa fenêtre?

Nous ne le savons pas encore. Inquiets,

les pervers de la continuité.

\breakk

\vspace*{4cm}

--- En voyant le piège l'homme est allé dans sa mémoire,

raconte la voix qui parle de l'endormie.

Mais maintenant il s'agite.

C'est qu'un autre homme s'incline sur le visage de l'endormie.

--- Qui cela peut-il être, se demande l'homme. Et comme tout est sombre et
qu'il ne peut voir le visage auquel elle rêve, il pense que l'autre
pourrait bien être lui, qui sait

Une nuit arrive maintenant sur le lieu où la femme rêve, une fin
d'après-midi, et là elle ne se refuse plus, elle ne repousse pas
l'ombre.

--- Là où elle est, où ils sont, personne n'est plus, dit l'homme à côté
du lit.

Il regarde encore et voit la femme et l'autre, lui peut-être, qui se
dissimulent entre les arbres dans un jardin. Elle appuie sa tête contre
une épaule sombre, c'est la dernière chose qu'il voit

\breakk

\vspace*{4cm}

Une nouvelle voix arrive, elle entre par la fenêtre. L'aveugle écoute

Cela ne finit pas. Ne finit pas?

Une fois de plus

L'enfance n'a pas de fin\\

--- A Santa Maria do Grão, raconte la voix

Celui qui passe par cette rue ne sait pas pourquoi il voit une cage en
regardant la maison. Celui qui passe regarde la maison. Pour ne pas se
tourmenter avec ça, il veut trouver une explication et s'arrête, regarde
bien et ensuite s'écarte et se dit que cela n'était qu'une illusion, que
cela vient des grilles de la maison

Et l'impression part avec lui, elle accompagne des jours durant celui
qui a vu la maison

Il n'arrive pas vraiment à s'en distraire, le voilà revenu devant cette
maison. Il cherche dans sa mémoire où il a bien pu voir cette maison, se
demande celui qui est passé.

C'est ainsi.\\

L'enfance n'a-t-elle pas de fin?\\

Dans la maison demeure une femme. Il n'y a plus d'homme ici, mais il est
resté une enfant, une fille, et elle mange tout ce qui tombe d'en haut.

La mère criaille par toute la maison, elle met de l'ordre, balaie le
sol, elle a une plume noire qui sort de l'ouverture de sa robe,
derrière. C'est avec elle que maintenant la femme a terminé de balayer
le sol.

En plus, c'est une femme comme une autre. Comme les autres on peut la
voir de partout.

C'est ainsi qu'un jour l'homme l'a vue

L'homme riait, il se tenait sous la femme et sentait cette plume sur son
corps maintenant mort.

Auparavant, la fille aussi riait. Mais la femme sait utiliser la plume
avec violence et la petite n'a plus jamais ri.

Avec la plume la femme s'évente les jours de chaleur, avec elle aussi
elle ranime le feu qui meurt sous une casserole. La femme utilise la
plume pour faire beaucoup de choses. Et maintenant elle est tourne le
dos, feignant de ne rien voir mais tout à coup elle se retourne et il y
a une mouche de moins dans ce monde

Avec la plume, la femme pourchasse aussi les insectes. Et elle sait où
les trouver dans les trous et les fentes et il n'y a aucune petite
fissure où puissent fuir

Les insectes.

Belle, la plume brille.

Le soleil l'éclaire. Noire

Le soir cependant, on ne sait pas où va la femme. Tous les soirs elle
sort. Le jour se lève déjà quand elle rentre, rapporte de quoi manger,
la fille mange ce qui lui tombe du haut

C'est un oiseau. Et elle chasse au clair de lune.

C'est cela, se dit la petite, seule, à la maison

pendant que sa mère vole par les rues la nuit et qu'elle elle reste là,
elle ferme les yeux et ainsi elle voit la femme voler, ensuite vient un
sommeil calme et elle s'endort en disant, j'ai cette mère, j'ai cette
mère. Je dois me réveiller demain

Cependant, le visage de la femme a commencé à annoncer la mort.

--- Elle est fatiguée d'être vivante, dit la fillette dans sa chambre à
son miroir. Elle est fatiguée d'être vivante, a dit la fillette un jour
en regardant le visage de sa mère endormie, un après-midi

Elle a vu que la plume se tordait sans lumière.

Et elle a eu peur. Si elle part, alla-t-elle dire une autre fois à son
miroir.

La peur était arrivée.

La peur.

Mais maintenant la fillette n'a plus peur.

La peur s'en est allée.

Elle est sortie de cette maison.

C'était à l'heure du dîner, aujourd'hui

Quand elle s'est assise à table, il y avait quelque chose sous elle,
entre elle et la chaise.

Elle s'est levée, est allée regarder dans le miroir et s'est
contorsionnée pour pouvoir voir derrière. C'était une ébauche. En elle
aussi elle apparaissait

Elle est retournée à table. Et elle a mangé avec un appétit qui revenait
aussi. Elle regardait sa mère assise comme toujours de côté et s'est
assise aussi de côté. L'héritière. Et elle a dit tout bas moi et elle
riait.

Maintenant, quelqu'un passe devant la maison, celui qui passe regarde la
maison

\breakk

\vspace*{4cm}

--- Si vous voulez, parlez davantage de la douleur\\

C'est l'aveugle qui a dit cela. Il parle avec les voix qu'il a écoutées.
Les voix.

Elles n'omettent jamais de venir lui raconter leurs histoires, dans le
vent,

les jours ne passaient pas\\

Les voix. Ces voix, ces histoires

Elles vont venir encore longtemps pour lui dire que la vie, là dehors,
est encore un lieu de rumeurs et qu'un Non tient tout sous sa coupe.

En lui, les hommes se cherchent sans se voir, dit l'aveugle. Mais un
jour l'oiseau va revenir

Il rit.

Dans la pièce où il est il y a un miroir, mais le miroir ne reflète pas
de rire

\clearpage
\thispagestyle{empty}
\movetooddpage

\vspace*{4cm}

Ah tu as frappé à ma porte. Tu es revenu.

C'est l'homme qui entre maintenant.

Il est revenu.

Il venu voir comment je passe.

Je ne passe pas. Les jours ne passent pas. Rien de tout cela ne change
tant que le Curau ne revient pas, dit l'aveugle à l'homme.

J'ai cette fenêtre et le temps s'y est arrêté. J'y attends l'oiseau. Tu
te rappelles, je t'ai tout raconté, comment cela a commencé,

Dit l'aveugle à l'homme et l'homme était de retour.

Il était entré comme la première fois.

Et maintenant il était à nouveau en face de l'aveugle.\\

--- Oui, j'ai entendu le vent, lui dit Jacinto. Les histoires qu'il vient
me raconter. J'écoute toujours le vent. Non, tu ne pourras pas écouter
aussi ces histoires. Non. Toi tu as tes yeux. Tu veux les garder.

Si tu veux entendre une autre voix. La mienne.

\breakk

\vspace*{4cm}

Maintenant le vent s'est arrêté. Et l'aveugle a près de lui cet homme
qui est revenu. Et il attend le retour de l'oiseau. Il veut entendre ses
ailes dans le ciel. Les Ailes.\\

--- Quand le vent s'arrête, je me souviens d'autres histoires, a-t-il dit.
Tu veux écouter l'une d'elles?

Oui. Je t'ai déjà raconté l'histoire du Curau, maintenant tu veux en
entendre plus, toujours plus? Laisse-moi te dire alors, pendant que
j'attends. Et ne pense pas que l'histoire du Curau n'est qu'une
histoire. Je suis ici, seul. Ici c'est partout.\\

L'homme écoute.

--- Vois, dit l'aveugle, le temps c'est ça et alors je ne me rappelle déjà
plus comme elle a fini.

Maintenant c'est l'histoire de Sumiro qu'il va raconter à l'homme. Ne
pense pas que l'histoire du Curau n'est qu'une histoire, répète-t-il.

Peut-être invente-t-il maintenant une fin à cette histoire, il dit. Et
l'homme écoute. Peut-être se rappelle-t-il à mesure qu'il l'invente. On
ne sait jamais. Il y a la mémoire, cette chose la nuit. Je ne me
souviens pas non plus des noms. De toute manière, en elle,
l'Imagination, il y a des choses qui grossissent, des feux énormes, et
il y a ce qui s'éteint. Ou revient changé, au retour, quand on veut se
rappeler. Dans la Mémoire.

Je vais par ces chemins. J'imagine. Je me rappelle.

Ils sont deux. Ils se mêlent.

J'ai dit que je ne me rappelais pas les noms?

Je ne mens pas. Il n'y a que son nom que je n'ai pas oublié. Sumiro.
C'était Sumiro. L'inoubliable.\\

C'était une fin d'après-midi.

Ils sont venus et l'ont emporté, attaché.\\

A cette époque, j'avais encore mes yeux pour voir, dit l'aveugle.
C'était avant le Curau et avec eux je ne voyais rien, tu comprends.

Regarde, la terre où ils l'attachaient écrivait déjà sur son corps cette
histoire que maintenant je te raconte

en la gravant aussi dans tes oreilles,

et cela étant arrivé tandis qu'ils l'emportaient, c'était une
anticipation de toute la douleur qui allait l'atteindre plus tard,

comme elle viendra aussi plus tard pour toi qui maintenant écoute
l'histoire et son lot de douleur

Eux cependant n'emportaient pas l'homme attaché parce qu'il résistait,
non. C'est qu'ils voulaient l'emporter comme ça. Vois.

A l'endroit qu'ils avaient choisi pour faire ce qu'ils ont fait, l'un
d'eux les attendaient, en surveillant, une paire d'yeux vifs, les armes
prêtes à servir contre ceux qui voudraient les empêcher de faire ce
qu'ils allaient faire. Vois maintenant ceci, c'était un Lieu hors de la
terre, il était au-dessus de l'enfer. C'est là que tout est arrivé. A
cette heure, les oiseaux ont commencé à voler bas, annonçant la mort
arrive la mort arrive, ils volaient et les gens entendaient le
froissement de leurs ailes.

--- Prends un café, dit l'aveugle à l'homme.

Ah, ce froid, dit l'homme.

C'est la nuit, dit l'aveugle.

Non. Je ne veux pas. Cette nuit, plus tard, je veux rêver après t'avoir
tout raconté. Si les voix le permettent

Ecoute.

Alors, je disais: ils l'attachèrent comme s'il n'était pas un homme.
Sumiro. Mais c'était un homme.\\

Bien qu'il vécût toujours de cette façon, et que toujours il eût les
yeux baissés cherchant sur le sol on ne savait pas ce qu'il cherchait.
On ne l'avait jamais vu lever les yeux. Eux, les yeux de cet homme,
s'accordaient bien avec la terre. Seulement avec elle. Jours après
jours, il vivait ainsi. Il vivait à l'intérieur. Certains disaient qu'il
n'était plus un homme et qu'il n'était plus parmi nous, les autres. Pour
moi, il était là.

Je ne sais pas quelle autre chose cela pouvait être, parfois il parlait
bien qu'on ne comprît pas ce qu'il disait

S'il avait perdu quelque chose et vivait pour la chercher, c'était son
secret.

Chacun aura le sien.

Je me demande si ce ne serait-ce pas en lui un secret qui ne peut se
partager, celui qu'ils n'ont pas pardonné,

Et c'est pour cela qu'ils l'ont attaché ce soir-là pour montrer que
tous, tous nous sommes égaux, nous petits hommes, mystères qui doivent
être révélés coûte que coûte pour que tout, l'humain, reste sous la
lumière sur laquelle la lame du couteau puisse compter, qu'on puisse
refuser quand il s'agit de faire un mort de plus?

Alors.

Emporté attaché, c'était ça qui donnait envie de rire. Il allait
profiter pour continuer à chercher, tranquille, il regardait vers le sol
qui s'étalait caressait ses yeux, il voulait utiliser ses yeux jusqu'au
dernier moment, il avait encore l'espoir de trouver ce qu'il avait perdu
jusqu'à la fin. Et il ne disait rien.

Il y avait un arbre à l'endroit où il l'avaient mené.

Vois ceci maintenant: ils avaient aussi un secret.\\

Ils avaient donc fait venir une femme, ils lui avaient bouché les yeux,
et l'avaient envoyée maudire l'arbre pour qu'il ne laisse pas de fruits.
Les fruits.

Et ensuite, ils avaient cloué l'homme sur l'arbre.

Sumiro, l'homme resta cloué là

Alors arriva l'heure des couteaux,

La première heure des couteaux

Il y en avait eu d'autres après.

C'est dans cette première heure qu'ils lui avaient enlevé son sexe.

Un homme peut-il ne pas crier sa douleur?

Alors il cria. On entend encore son cri quand on passe par ce lieu à la
même heure, en fin d'après-midi. Dit-on. Moi je ne suis jamais retourné
là-bas, je ne sais pas. Il resta cloué sur l'arbre. Et après cela plus
aucun son ne s'échappa de lui.

Rien ne changeait ici.

Ils avaient emporté, traîné Sumiro et maintenant il était là, sur
l'arbre. Attend. J'essaie de me rappeler

Je me rappelle. Dans le ciel des nuages immobiles. Ce n'étaient pas des
nuages légers. Ils portaient en eux un rouge sang indélébile. Lourd. Les
gens d'ici savaient que la pluie allait tomber, mais quelle était cette
pluie, figée, là-haut, ça on ne savait pas

Nous avions le ciel au-dessus de nous et en-bas, nous étions seulement
des hommes, des femmes et il y avait aussi quelques enfants autour de
l'arbre.

Ce fut à nouveau l'heure des couteaux

Du côté de l'horizon s'éleva un grand bruit.

Il semblait que la pluie allait tomber. Mais elle ne tomba pas. Elle ne
tombait pas. Et ce fut l'heure où ils enlevèrent les yeux de l'homme.

C'est qu'il n'avait pas cessé de regarder le sol, de chercher, et il
fallait que cela s'arrête. Cela n'aurait servi à rien de lui faire subir
toutes ces choses s'il ne s'arrêtait pas de chercher.

Nous, nous regardions. Nous ne faisions rien.

Les témoins.

Ils voulaient que les yeux de l'homme cessent de chercher en roulant sur
le sol, et, ça nous l'avons vu, l'un d'eux sortit un petit sac, et y mit
les deux yeux, fermés,

Dans l'obscurité.

Comment allait-il chercher maintenant?

Je ne le savais pas.

C'est qu'à cette époque je voyais encore, j'avais mes yeux et je ne
voyais rien. Je voyais l'homme sans yeux sur l'arbre et je me demandais
comment maintenant va-t-il chercher, sans avoir de réponse~; mais lui
cherchait encore, maintenant je le sais, même s'il était autant dans
l'obscurité que ses yeux rangés dans le sac de l'autre.

--- Ne sois pas dégoûté. Il semble que c'est avec les yeux des gens,
n'est-ce pas?

Mais c'étaient ses yeux, pas les leurs, rappelle-toi.

L'homme regarda par la fenêtre et dit:

--- Seul peut être artiste celui qui aura une vision originale de
l'infini.

C'est Schlegel qui l'a dit

--- Je ne sais pas dit l'aveugle. Je ne sais rien. Je suis ici. Aveugle.

Je ne peux te dire que ce que je te dis, nous sommes tous ensemble dans
cette vie,

Hommes et dieux,

Ceux qui nous ont faits avec de la terre

Ceux de l'Eau, ceux de l'Air et ceux du Feu

Et ceux de l'Absence.

Maintenant je sais cela\\	

Dans le vent la voix passait à nouveau et disait\\

--- Viens Curau, viens emporter les hommes vers tes jardins\\

Et l'aveugle dit:

Je continue.

Plus tard, ils l'avaient mis nu, lui avaient enlevé ses vêtements, les
avaient brûlés. Les femmes n'avaient pas détourné leurs regards, c'était
une autre honte

Ils lui avaient coupé les mains, à la troisième heure des couteaux. Et
cela soulagea en lui une douleur, une douleur effaçant l'autre, tu
comprends: la douleur qu'il ressentait avant,

Celle des yeux arrachés

Mais à peine avaient-ils vu la nouvelle douleur naître et l'ancienne
disparaître du visage de Sumiro, qu'ils lui en causèrent une autre. Ils
mirent un grand clou, Noir, dans sa bouche. Et ainsi il ne pouvait plus
parler, même s'il n'avait rien dit pendant tout le temps que cela avait
duré. Cela, je le savais\\

Et cela dura.

En regardant, plus tard, on ne pouvait savoir où lui finissait et où
commençait l'arbre.

Ce fut ainsi.

C'est qu'ils avaient ouvert le tronc, à la quatrième heure des couteaux,
et mis l'homme dans l'arbre,

Une partie visible, l'autre cachée.

Ils voulaient de cette façon lui faire perdre de plus en plus sa qualité
d'homme. Tirer de lui tout mystère. Faire de lui un arbre.

Chacun comprendra à sa façon. Oui. C'est cela. Chacun est un autre.

Mais vois, c'était ce qu'ils voulaient.

C'est ainsi que je le comprends.

Je suis ici, aveugle. Ici c'est partout

Les vois de la terre viennent de loin

Un insecte m'a regardé

Qu'est-ce qu'il y a à voir?\\

Nous sommes restés là. Je regardais. Les autres aussi regardaient.

Et là, les jours allaient je ne sais où

Je suis là. Je regarde. Nous ne faisions rien.

Ils avaient mis l'homme dans l'arbre.

Ils décidèrent de lui donner à manger pour qu'il dure plus longtemps. Et
cet homme, notre ex-voisin, Sumiro, a accepté alors leur nourriture.

Les nuages étaient toujours là-haut, la pluie ne tombait pas, et il ne
mourait pas, cet homme dans l'arbre, il ne cessait pas d'être un homme
et ne devenait pas arbre une fois pour toutes. Il n'avait rien à nous
dire. Nous regardions\\

Le jour où il voulut rejeter la nourriture, il ne le put pas. Ils
avaient fermé la sortie.

Il dut souffrir aussi les douleurs communes.

Alors il ressemblait encore davantage à un homme.

Ça, ils le notèrent. Ils virent qu'ils ne devaient pas lui donner à
manger et ne lui donnèrent plus

Il était là, tout près. L'homme dans l'arbre. Et pourtant, bien loin de
ce lieu, une chose changeait en lui, elle restait dans ses rêves où rien
de tout cela n'arrivait

En lui, dans ses rêves, devaient parfois entrer l'une ou l'autre des
voix de ces hommes, de garde près de l'arbre

Mais cela se termina quand vint l'autre heure des couteaux et qu'ils
percèrent ses oreilles. Et qu'ils le laissèrent absolument seul, dans un
blanc,

Et il se fit un silence dont on ne connaissait pas l'existence.

Et ainsi il ne pouvait plus entendre quand de nouvelles heures
arrivèrent pour les couteaux, par vagues, les unes après les autres, en
résonnant,

Et ils lui arrachèrent ses jambes pour qu'il ne soit plus jamais humain

Mais, je crois, il recevait cela comme s'il était un autre, dans une
autre histoire. Et dans cette histoire ils voulaient peut-être lui
arracher son âme, son oiseau. Pour l'enfermer dans une caisse en bois.
En bois de l'arbre. L'arbre dans lequel son corps disparaissait en tant
que chose humaine. Mais dans cette caisse cette âme n'allait pas cesser
de voler, elle allait voler d'abord dans le corps de l'homme. C'est que
lui et l'arbre étaient alors devenus une seule chose.

Là-haut, le ciel s'agita. Etait-ce son sang qui finalement allait
tomber?

C'était cette pluie. Qui menaçait de tomber et qui ne tombait pas. Elle
ne tombait pas. Elle ne tomba pas et ne tombera pas

Les heures des couteaux allaient et venaient\\

Ils avaient coupé sa tête pour jouer avec. Et il comprit ce qu'était le
vide

Il avait déjà compris ce qu'était l'obscurité, n'est-ce pas?

Il avait déjà compris ce qu'était le silence

Et maintenant il comprenait ce qu'était le vide.

Ceux qui montaient la garde autour de l'arbre s'amusaient, riaient,

Est-ce que l'homme reste toujours un enfant? Je me demande. Ils riaient
et s'envoyaient la tête les uns aux autres. C'était pour passer le
temps, là. Ils la lançaient d'un côté à l'autre, la boule magique, elle
volait parmi ces hommes,

Et cela sans qu'il ne se sentît en rien humilié.

Mais les jours étaient nés. Ils étaient morts.\\

Prends encore un café. Prends. Cela va finir. Il faut que cela finisse.
Maintenant nous allons vers la partie où elle est autre, la vie\\

Ecoute.

Ils amenèrent la femme qui avait fait Sumiro. Ils la placèrent devant
l'arbre. C'est que maintenant le désespoir était tombé sur ces hommes.
Celui d'avoir perdu Sumiro dans cette vie. Ils voulaient le faire
revenir. Où pouvait-il être maintenant?

Ils se réunirent alors et parlèrent. Sous un autre arbre. A l'écart.

Ensuite ils revinrent.

Nous allons faire ainsi, dirent-ils. Et ainsi firent-ils. Pour que
Sumiro revienne, ils allèrent le chercher là où il était encore et ils
arrachèrent les vêtements de la femme, et ils cherchèrent dans son corps

Et tout cela devant l'arbre pour qu'il voie sans les yeux.

Ils allèrent chercher un enfant aussi et ils dirent, C'est un garçon.

Ils cherchèrent des traces de Sumiro en lui.

Et ils allèrent en chercher d'autres, effrayés, et ils disaient, Ce sont
des frères.

Ils firent la même chose avec tous. Et d'autres choses aussi. Celles qui
n'ont pas de nom.

Sans nom aussi ce qu'ils finirent par faire de lui.

Comme ils ne voulaient pas rester sans Sumiro,

ils se transformèrent, là devant nous, en hommes-sables, secs, et ils
burent son sang,

ils se transformèrent en hommes-chiens, ils étaient enragés, et
rongèrent ses os,

et revenant à leur état d'hommes qu'ils étaient, ils travaillèrent,
travaillèrent, organisés, précis. Et nous vîmes qu'avec la peau de
Sumiro ils avaient tissé un hamac, et qu'ils l'avaient fait sécher sous
un soleil violent qui était apparu dans le ciel entre les nuages de
cette pluie qui ne tombait pas, ne tombait pas, ne tomba pas et ne
tombera jamais, et se montra à nos yeux. Les témoins.\\

C'est ainsi que cela se passa.

Et une roue tourna. Lentement

D'abord vers la gauche.

Puis vers la droite.

Ensuite elle alla une fois vers la droite une fois vers la gauche, comme
un doute. Et à chaque tour qu'elle finissait, nous voyions

Les fruits naître dans l'arbre.

Alors nous avons dansé. Les témoins.

Et nous avons chanté. Nous applaudissions et nous buvions une boisson
amère, en tournant autour de lui enfin calme, ou serein, ou désespéré et
sans savoir pourquoi

\breakk

\vspace*{4cm}

Maintenant va-t-en,

Dit ensuite l'aveugle à l'homme. Et l'homme s'en alla

Je veux rester à nouveau seul.

Rester ici. Dit l'aveugle.

\breakk

\vspace*{4cm}

La mémoire.

C'est à nouveau elle. Mais c'est la mémoire d'un autre dans cette voix
qui vient parler à Jacinto,\\

et arrive dans le vent\\

Quand notre frère est né, l'horloge du salon s'est arrêtée. Je me
souviens

Elle est restée à l'heure qu'elle marquait.

Je me souviens

Notre père a dit, de Lui, de ce fils qui naît de moi viendra
l'allégresse. Il a dit cela et en pâlissant, tout à coup plus vieux, il
s'est retiré, s'est enfermé dans sa chambre et reste muet depuis lors,

une autre machine cassée dans cette maison.

Ce père.

C'était un enfant. Il ne pleurait pas. Il ne riait pas.

Il nous regardait seulement, distant. Le Distant. Il était là parmi nous
et n'y était pas. C'était comme s'il était toujours à l'endroit d'où il
était venu

Il est venu de notre mère, dit l'un de mes frères.

Mais moi je sais qu'il est venu de plus loin quand je vais le voir.

C'est notre dernier-né. L'Eternel. Il ne grandit jamais. Et les années
passent. Attendons

Notre père a dit cela. Et maintenant nous attendons l'allégresse.

Et cette attente nous rend inquiets. Tout est effroi. C'est cette
attente

Une bruit, une vitre qui se casse

Les accidents, les jours. Ce n'est pas cela l'allégresse.

Nous sommes allés voir le verre cassé à la cuisine et ce n'était pas
elle.

Et alors, comme rien ne s'est passé, nous avons cherché dans toute la
maison un indice de son arrivée.

Mais nous ne sommes jamais d'accord.

Elle arrivera le jour des morts, dit ma mère. L'allégresse.

Une de mes sœurs veut qu'elle arrive un jour de fête.

Nous avons ce père et il cherche dans les poches des vêtements, renverse
tout, veut trouver l'autre côté des choses, une lettre perdue, un nom
noté dont il ne sait plus où il l'a rangé. Et ses doigts tremblent. Il a
peur de toucher tout à coup la chose et cela n'arrivera pas un jour
comme les autres si elle vient, quand elle viendra et c'est à l'un de
nous de trouver.

Moi je sais.

Notre allégresse viendra quand il commencera à grandir

Il reste là dans la chambre. Et il ne grandit pas. Nous attendons.

Entourant le lieu où il est, couché, au fond d'un puits d'où nous
voulons tirer quelque chose

L'un de nous tente de deviner. On parie. Rien. Et on s'impatiente.

Parfois un autre pleure.

Ma mère crie.

Dans le salon, l'horloge est encore arrêtée. C'est son premier mouvement
que j'attends. Ce sera quand les aiguilles se remettront à bouger que je
serai avisé. Elle arrive. L'allégresse.

l'enfant est toujours à l'endroit d'où il est venu. Et pour cela je
reste aussi arrêtée dans le salon. Et je regarde l'horloge.

Dans la chambre notre nouveau-né, éternel, attend encore.

Lui seul sait quand elle arrivera. L'Allégresse.

J'espère seulement qu'elle arrivera avant notre mort

\breakk

\vspace*{4cm}

Là dehors maintenant c'est la nuit.

--- Cette nuit ne passe pas, dit l'aveugle à sa fenêtre.\\

Quand il fait nuit, là dehors Jacinto pense qu'il entend les ailes
rouges qui viennent du ciel. Maintenant encore il les a entendues à
nouveau. Les ailes

Elles arrivent

Elles sont en train d'arriver\\

Ce n'était pas le Curau de retour cependant.

Et il attend. Cette attente. Et les années passent

\clearpage
\thispagestyle{empty}
\movetoevenpage

\vspace*{4cm}

J'écoute.

C'est du fond de la tête. Et arrive dans le vent:\\

Hier nous sommes retournés au pont.

Sur lui le bois a perdu sa forme de planches, j'ai perdu cette forme
humaine qu'on m'a donnée, dit le bois\\

Ce pont vieillit de plus en plus. Il dit je tombe en morceaux,

et quand nous passons dessus l'un de nous tombe aussi dans l'eau, là en
bas. Pas tous ne remontent à la surface.

--- Il y a un mal avec des yeux d'enfants et des dents là en bas,
disons-nous les uns aux autres, et nous rions. Mais nous restons sur le
pont.

Il y a un mal là en bas, répétons-nous. Et nous rions.

Et nous faisons trembler le pont en sautant dessus.

Quand nous sommes arrivés au pont la nuit tombait déjà.

--- Je regarde ce pont et cela me rend triste, dit le vieux.

Il vient toujours avec nous.

Je vois l'un de vous tomber là en bas et ne plus revenir,

dit-il. Et cela me rend triste.

Et si parfois j'en vois un autre sauter de joie en tous sens, je suis
joyeux moi aussi.

Voyez, nous ne sommes pas plus que cela, et cela va être changé par la
vie.

Tout vient de dehors et entre.

Et il y a aussi des choses qui sortent de moi et entrent dans la vie.

Et nous avançons

C'est ainsi.

Nous vivons. C'est peut-être le pire.\\

Quand le vieux parle, sans qu'on comprenne toujours ce qu'il dit.

Nous écoutons. Il parlait. Ensuite il s'est éloigné de nous.

Il reste là assis sur la rive, seul, et il regarde l'eau passer sous le
pont.

C'est quand l'après-midi se termine que nous allons vers le pont. Réunis
dans une rue de la ville, tout à coup nous partons, vers le pont, vers
le pont

Et les uns disparaissent dans le chemin, peut-être sont-ils allés chez
eux peut-être pas, on ne sait

d'autres attendent pour tomber du pont quand nous arriverons, ce qui est
une autre façon de s'échapper

Ces disparitions ne sont cependant pas tout.

Il arrive aussi que quelques-uns disparaissent seulement en partie. Un
accident. Et nous avons un mutilé pour rire, il devra maintenant marcher
sur une seule jambe pour la vie entière. Celui-la saute. Ce n'est plus
un homme entier. Nous sommes ainsi. Nous nous faisons par morceaux. Nous
restons penchés sur le pont à regarder l'eau s'assombrir\\

Aujourd'hui, jusqu'à présent rien n'est arrivé. Nous attendons que la
nuit soit tombée complètement

Alors l'un de nous y va.

Maintenant il entre dans l'eau.

Nous le voyons entrer en se rappelant comment auparavant les autres sont
entrés aussi dans l'eau sombre pour la dernière fois

Et je dis:

--- Pierres. Les pierres.

Et nous allons prendre les pierres. Elles sont là, elles pèsent pour
toute la vie. Mais elles seront légères dans l'air

Et je donne l'exemple, que les autres imitent et celui qui est entré
dans l'eau sort en courant, fuit, il a du sang sur la tête, il va
s'asseoir à l'écart à côté du vieux. Il y en a un autre là. Nous
regardons. Maintenant à l'écart ils sont deux.

Nous sommes sur ce pont.

Nous attendons. Et nous voyons ces autres là à l'écart.

Ils sont seuls. Côte à côte. Maintenant ils regardent l'eau couler,
sombre

Sur la rive celui qui a reçu les pierres pleure. Un lapidé

Nous sommes tous ainsi, il n'y a rien à faire\\

Alors: cela doit arriver, dit l'un de nous maintenant.

Oui

Je l'ai senti à cet instant aussi.

Là sur la rive, celui que nous avons bombardé de pierres s'arrête de
pleurer. Il a levé les yeux. Il voit autour de lui des arbres immobiles,
il n'y a pas de vent. Et il court, vient à nouveau vers nous, sur le
pont, nous nous serrons les uns contre les autres et nous attendons
qu'il arrive. Mais nous ne quittons pas le pont. Et seul le vieux reste
là-bas sur la rive à regarder l'eau couler

Nous ne savons pas avec lequel de nous ce sera.

Ensuite nous serons prêts à revenir demain, et tous les autres jours de
notre vie aussi.

Et pour partir d'ici.

Et cela est déjà en train d'arriver

L'un de nous ne sait plus où mettre les pieds, comment rester à la
surface, le bois se casse sous lui, nous ne ferons rien pour le retenir.
Et les eaux passent sous le pont. Nous sommes ainsi. Il n'y a rien à
faire. Cette fois c'était lui alors

Il s'en est allé. L'un de nous.

Nous regardons. Et l'eau est calme.

Et nous nous écartons à nouveau les uns des autres. Le groupe se défait.
Chacun pour soi dans cette vie que certains appellent un pont. D'autres
l'eau. D'autres une noyade.

Demain il faudra venir à nouveau, en fin d'après-midi.

Sur la rive, le vieux se lève .

Nous rentrons.

Personne ne dit rien. Nous rentrons sans parler, dans la nuit, et nous
nous séparons aux angles des rues, pour rentrer chez nous. Et nous
disparaissons dans les rues. Il n'y pas de quoi se lamenter

\clearpage
\thispagestyle{empty}
\movetoevenpage

\vspace*{4cm}

Le vent a déjà fait le tour de toute la terre avec l'histoire que
l'aveugle a racontée

Et maintenant il passe à nouveau par la fenêtre de Jacinto\\

--- Ils sont venus et l'ont enlevé, attaché. Vois, par où ils l'ont traîné
en écrivant sur son corps cette histoire que je te raconte maintenant en
l'inscrivant aussi dans ton oreille

Dit le vent, une seule rafale qui passe par la fenêtre de Jacinto en
apportant un morceau de l'histoire de Sumiro, qui reste dans l'air après
que l'aveugle l'a contée.

Et il est passé. Et de ce morceau de voix de vent il ne reste plus rien\\

Une autre voix arrive ensuite en disant:

De l'endroit où je suis, je ne peux voir que l'horizon et un fleuve.
C'est moi qui ai fait cette maison

Il y a un homme dans le fleuve. Il me voit. Il vient par là. Nous
parlons. Il me raconte sa vie et ensuite cet homme me demande de partir
avec lui.

--- Ce fleuve va par là, il montre du doigt.

Il dit:

--- Allons

Je sais que cette île n'existe pas, mais je ne dois pas y aller. Non je
n'irai pas.

Et quand il rame à nouveau au loin, je détourne mes yeux, j'ai peur de
voir comment l'horizon va le dévorer. C'est ainsi que la forêt
maintenant se referme sur moi aussi. Peu à peu. Bientôt la nuit viendra
sur cette île. Et il y a des animaux qui seulement alors se
réveilleront. Je sais que cette île n'existe pas.

De l'endroit où je suis je ne peux voir que l'horizon et le fleuve. Je
vois un homme là. Et cet autre qui arrive par là, lui aussi m'a vu\\

Oui, Andara est vraiment notre Afrique, de ma fenêtre, aveugle, je le
sais, dit Jacinto.

Et je veux entendre le reste.

J'écoute.

Le reste.\\

J'écoute.

En sortant de la forêt, j'ai vu la maison.

Ça, la maison.

Un animal astucieux qui va ouvrir la porte et regarde à l'intérieur
d'une maison avec des yeux féroces comme ceux avec lesquels moi j'ai
regardé ne se rencontre pas tous les jours parmi les hommes

Chez les hommes ma loi c'est avance un pied et n'avance l'autre, resté
en arrière, que bien plus tard. Je ne prends pas de risque. Je suis prêt
à m'échapper. Je suis ainsi.\\

C'est de cette façon qu'alors j'ai ouvert la porte et que je suis entré

Ismaël n'était pas chez lui. Et j'ai fait tous les gestes qu'il fallait
pour préparer le piège. Ensuite, je suis sorti. Je suis resté dehors

Derrière un arbre, je voulais être là pour tout voir. Caché. Et
maintenant j'attends le retour d'Ismaël.

Mais il me voit en arrivant.

Il a de bons yeux Ismaël, me dis-je, mais il ne verra pas le piège. J'ai
f ait tous les gestes qu'il fallait

--- Je ne vois plus de rancune dans tes yeux , ah

Me dit-il

Entre.

Et nous entrons.

Ismaël passe devant. Cette maison est la sienne

--- C'est aussi la tienne, veut-il commencer à me dire

mais un bruit vient se mêler à sa voix et l'interrompt, un autre bruit
et alors ah voilà Ismaël dans les airs. J'ai utilisé de grosses cordes.
Grosses. Suspendu ainsi il ressemble à un oiseau sans ailes, muet.
L'effroi.

Ismaël, Ismaël, je crie. Maintenant tu es foutu.

Et je volais à travers la maison autour de lui. J'applaudissais. J'ai ri
et j'ai dansé, ainsi, oublié, je me risquais, sans Ismaël sur mon chemin
je peux laisser mes pieds aller où ils voudront, disais-je,

Et le reste du piège entra en action, le reste que j'avais préparé, une
seconde attaque au cas où il échapperait à la première. Si Ismaël avait
eu des ailes, elles saigneraient,

et le piège me souleva aussi pour me balancer dans les airs à ses côtés.
Pendus tous les deux, maintenant nous sommes deux oiseaux. Cela ne
devait pas se passer comme ça, me dis-je

Non.

Et Ismaël est maintenant en train de me dire, Il faut que tu te
réveilles. J'entends sa voix au loin

Tu vois, tout cela nous fait du mal à tous les deux

--- Oui, ai-je répondu

Et je me suis réveillé.

Mais maintenant je me frotte les yeux assis dans mon hamac et je me
rappelle mon rêve, et ce oui veut se transformer en non. Et ce que je
veux c'est rêver encore une fois

--- Et c'est pour ça qu'ils l'ont traîné cet après-midi-là pour montrer
que nous tous, tous, sommes égaux, de petits hommes, des mystères qui
doivent être révélés, coûte que coûte

C'est le vent. Une de ces nouvelles rafales est en train de passer par
la fenêtre de Jacinto encore une fois.

Après, il cesse

Et l'aveugle se dit, je reconnais cette voix.

L'histoire de Sumiro ne veut-elle pas s'évanouir dans l'air?\\

Elle arrive dans le vent. J'écoute. Aveugle\\

Quand on a peur des hommes il faut aller chez un ami

Ainsi j'ai cherché Fabiano ce matin.

Nous avons marché dans son jardin.

Il me montrait ses animaux, les ombres que leurs corps projetaient, tout
ça était là autour de nous

De retour à cette maison où je m'occulte, occulte, plus tard je ne me
rappelais plus ce que nous avons fait ensemble.

Que se cache-t-il de la lumière des journées?

Que se cache-t-il dans la lumière des journées?

La nuit, je vois des mains tachées de sang. Et je revois tout ça dans un
rêve

Fabiano qui me laisse tuer tous ces animaux-là. Et il me disait, Quand
ta peur reviendra, toi aussi, tu peux revenir. Ils renaîtront tous\\

Jacinto écoute.

C'est une voix de plus. C'est du fond de la vie:\\

--- Aujourd'hui je vais essayer encore une fois

Lentement, affreusement, cette fois-ci je m'en vais d'ici

--- Je m'en vais,

je le crie avec mon cri le plus rouge et il ne cesse ni même quand je me
trouve face à la bouche tordue par un rire sur le miroir et où je me
vois pendu pour ne pas me sentir seul lors des jours où rien n'arrive.\\

C'est les rues que je veux. Et dès maintenant, je tremble quand je me
vois, en train d'anticiper, et les choses qui vont se passer dès le
moment où j'y serai. Les mauvaises choses et les autres. Toutes. Et ne
me parlez pas de la peur

Je répète. Je m'en vais d'ici.

Et déjà je me hâte.

Même si je prends un risque, celui de sortir par l'arrière de la maison
et d'arriver juste à l'arrière-cour

Et de rester coincé entre les murs, si je rate la direction de la porte
d'entrée.

Pour cela, je dois avoir de petits yeux vifs. Et pour que ça n'arrive
pas à nouveau , je dois garder la voie. Et m'en aller. Maintenant. Les
pieds bougent déjà. Je suis en train d'y aller. Vers les rues. Les rues.
Mais la porte d'entrée semble me conduire seulement à la chambre. Voire
vers sa plus grande profondeur et sous les yeux de celui qui ne me perd
pas de vue et sourit, dans le miroir, plus que jamais, sous le lit. Où
je me mets et je me sens bien maintenant.

Demain, je vais essayer encore une fois.\\

--- Nous regardions. Nous ne faisions rien. Les témoins, dit le vent

--- Et nous chantions. Nous applaudissions, dit le vent

--- Et nous buvions un breuvage amer, dit le vent\\

en passant, à nouveau, par la fenêtre de Jacinto avec l'histoire de
Sumiro.

Et cette dernière rafale qui passe en ce moment va encore une fois vers
le lointain.

Le vent va refaire le tour complet de la terre

Le vent va raconter l'histoire de Sumiro à d'autres hommes

\clearpage
\thispagestyle{empty}
\movetoevenpage

\vspace*{4cm}

D'un pont

D'une île

On me parlait d'une île qui n'existe pas, se dit Jacinto. Combien de
temps est déjà passé depuis? Les années\\

Nous avons Jacinto, et Jacinto est l'homme à la fenêtre

\breakk

\vspace*{4cm}

Il y a un autre temps pour l'enfance maintenant.

Là, pendant l'enfance, ce temps d'étonnement partout, où tout a
commencé. La vie

Se dit Jacinto à la fenêtre\\

Et une voix de plus arrive dans le vent. C'est la voix de Jacinto un
garçon et dans cette voix où le vent apporte l'enfance, Jacinto va
revenir\\

Je me rappelle Andara comment c'était auparavant, dit la voix.

Ah l'enfance, se dit Jacinto.\\

C'est à Andara où cette ville, Santa Maria do Grão

Andara est un lieu qui fait peur

Andara a été la première partie de la ville à apparaitre. Puis, la ville
a augmenté de plus en plus et aujourd'hui Andara est un lieu presque
oublié, un souvenir pour les noyés,

elle est restée là sur une rive d'une rivière

il y a des eaux profondes, lentes, elles passent.

Andara est un enchevêtrement de maisons dont les portes et fenêtre
donnent sur la forêt. Vous sortez de chez vous et à peine vous êtes
sortis il y a des dents, des yeux autour de vous

On entend des souffles. Une peur arrive et elle devient immense. En
haut, la lune, toujours. Blanche, s'il ne fait pas encore nuit. Donc, on
rentre et on verrouille la porte. Ce jour-là, on ne sort plus de la
maison. Andara, c'est ça. Dans cette ville, ou dans une autre, peut-être
celle qui est occulte, où elle a commencé. Santa Maria.

C'est Andara où Santa Maria do Grão a commencé. Dans l'enchevêtrement.

Un enchevêtrement est toujours vert, me dit-on.

Mais, parfois, il devient tout noir.

Il n'a pas de fin, ses fleuves qui n'existent pas et ces arbres absents
autour de moi s'étendent à perte de vue. Cela s'étend jusqu'où l'homme
peut aller. Et il va encore plus loin. Voilà la région. Et Andara est
bien plus~encore : Andara est tout l'enchevêtrement.

Je me demande, un garçon doit-il tout savoir?

Les histoires d'Andara que j'entends

Ici des choses se passent aussi.

Santa Maria do Grão aussi devient folle.

Mais à Andara on me dit c'est pire. Là on marche dans une rue et, alors,
on voit, d'une fenêtre, on nous regarde. Qui est celui à la fenêtre qui
me regarde sans vouloir être vu et voulant savoir qui suis-je,

on s'arrête dans la rue et se demande.

C'est juste la vitre de la fenêtre, et dans la vitre un autre, après on
le sait. C'est vous-même\sout{s} qui étiez là et vous regardiez. A ce
moment-là, cependant, il n'était pas possible de le savoir. Il y un
autre là-bas, il nous regarde. Et voilà tout. Et l'autre est réel

Andara est un lieu comme ça.

Comme tous les autres lieux, tous,

me dit ma tante.

--- Ça a un nom, dit-elle.

 Mais, parfois, elle, la folie, elle aussi peut venir se montrer d'un air
plutôt de grimaces, rires indéfinis et l'on est pris de désespoir de se
rouler par terre avec des cris qu'on peut entendre de loin

Mais à Andara. Voilà cette tante en train de me raconter des choses.

--- Andara, dit-elle, est là. Entre la forêt d'un côté et un fleuve de
  l'autre, avec une vocation pour la mort qui n'est pas vue par ici.
  Elle est là avec cette vocation et un souhait d'aller toujours vers un
  mystère plus profond, et vers d'autres, sans fond. C'est ça qui a fait
  partir Irido, c'est pour ça qu'il est allé habiter là-bas. Il portait,
  lui-aussi, cette vocation pour la mort.

Irido est l'homme qui l'a abandonnée.

Donc elle raconte l'histoire du cimetière.

C'est à Andara que cette ville a eu son premier endroit pour garder les
morts, dit-elle. Si on n'ouvre pas tous les yeux et ceux de l'instinct
de s'enfuir de la mort on finira aussi par se perdre, entrant dans une
rue inconnue et sortant dans une autre jamais vue auparavant, et on est
troublés, et sent qu'on est entraînés, emportés, et alors, de rue en rue
on finit par se retrouver dans le cimetière d'Andara,

qui a déjà été envahi par la forêt, se mêlant l'un à l'autre, où les
morts et le végétaux sont ensemble,

et on marche parmi les petites maisons de terre des morts, dit ma tante,
et on écoute comment la terre a des anciennes voix, par là, rumeurs.
Après quelques jours, et les jours ne passent pas, parmi les morts, une
personne à qui tout ça arrive ne sera plus jamais elle-même.

Irido y est allé parce qu'il ne me voulait plus, elle le répète toutes
les fois pour finir l'histoire.

Et se tait. La femme.

Irido. L'homme.

--- Mon oncle Irido veut que j'y aille avec lui, je lui ai dit un jour.

--- Non. Ne va pas.

Ça, elle me l'a dit ainsi, sèchement.

Et c'est toujours la même réponse quand je dis que l'homme veut que j'y
aille pour rester avec lui.

Cet homme-là à Andara.

Mais lui, il insiste.

Il me demande d'y aller. Il le demande souvent dernièrement.

Aujourd'hui encore, un homme est venu d'Andara, il est entré chez nous
et a dit:

--- Irido veut savoir si le garçon y va ou pas

Ces jours-là j'avais mes soeurs et leurs yeux, effrayés

Elles ont eu encore plus peur en entendant ce que l'homme disait, une
peur plus grande que celle à l'époque où notre père est mort et notre
mère est morte et nous sommes restés ainsi, vivants

L'homme a dîné chez nous et s'en est allé.

Mes soeurs ont peur qu'Andara, lui, m'emporte vers le cimetière

Pour elles, Andara n'est pas uniquement ces maisons-là vides, ce début
oublié de ville. C'est Andara et il nous attrape, nous entraîne vers la
mort. Et alors, il n'y a plus de retour pour ceux qui sont emportés par
Andara, disent-elles. Et elles me regardent. Et murmurent. Et pleurent.
Et me regardent comme si tout s'était déjà passé. C'est comme ça pour
ceux qui ont peur. On anticipe

Peut-être c'est pour cela que ma tante vient de décider.

Elle m'a laissé partir maintenant.

Elle aussi, elle a dû anticiper ce qui m'arriverait si j'y étais allé,
il fallait s'y soumettre pour s'en débarasser

Je m'en suis allé.

L'enfance. Elle est exactement ce temps d'étonnements partout. Et elle
ne finit jamais, je sais.

Il était là, cet homme.

Mon oncle, qui avait une nuit dans les yeux.

Il m'a pris dans ses bras. Et il m'a emmené manger la viande d'un animal
qu'il avait attrapé, je l'ai attrapé ce matin, m'a-t-il dit, pour
t'attendre. Cette nuit-là dans ses yeux

À table, pendant que nous mangions, il m'a dit, demain, si tu aimes
cette viande, nous allons en attraper un autre.

Manges-en plus.

Tu aimes~ça?

Il me parlait de l'animal et m'offrait plus de viande.

Mais aujourd'hui, quand je me suis réveillé, j'ai eu peur d'aller avec
lui attraper un autre animal. Et je n'y suis pas allé

Comme tout passe vite à Andara.

Je lui demande souvent, quel animal est-ce, quand nous mangeons, tous
les jours il en attrape un et l'apporte à la maison. Cette maison
n'existe pas, elle est trop vieille, ses murs sont troués et on peut y
voir à travers, dehors, les arbres. Et tout ce qui y vit. Allons voir de
près l'animal, mon oncle me répond. Mais je n'y vais pas. Dans cette
maison nous mangeons la viande de l'animal tous les jours, que mon oncle
met sur la table encore une fois. Le dîner.

--- Aujourd'hui, et aujourd'hui tu veux aller avec moi en attraper un?

Il me le demande tous les matins. Et il sort.

Il revient avec un animal. Déjà mort. Et sans la peau. Et coupé en
morceaux, c'est pour que je ne sache jamais quel animal est-ce.

C'est une bonne viande. Il n'y en a aucune pareille.

Et aujourd'hui?

Je n'irai jamais avec mon oncle je pense.

Je ne sors pas non plus de cette maison pour ne pas me perdre dans les
rues et finir au cimetière.

Et aujourd'hui? M'a-t-il demandé encore une fois ce matin, avant de
sortir attraper encore un autre animal.

C'est dans le cimetière qu'il les attrape tous.

Dans la partie du cimetière qui est envahie par la forêt, là où la forêt
est bien vivante, et avance toujours

Et aujourd'hui?

Je n'ai pas répondu. Je ne réponds plus. Il comprend que si je ne
réponds pas je dis non, je ne vais pas. Et il sort.

Et aujourd'hui? Me demande-t-il et il sort.

Il ne lève pas les yeux pour poser la question. Et il a toujours la nuit
dans ses yeux. Je sais.

Je n'y vais pas, je n'irai jamais

Et aujourd'hui. Me demande-t-il.

Ce que je ne veux pas c'est voir l'animal vivant et, après, mort. Je
n'oublierai jamais son goût pourtant

\breakk

\vspace*{4cm}

C'est la fin pour l'enfance maintenant.\\

Et cette voix qui dit dans le vent

--- Viens Curau. Viens emporter les hommes vers tes jardins

\breakk

\vspace*{4cm}

Mais, à nouveau, voilà une voix qui revient.

Qui dit:

--- A Santa Maria do Grão, celui qui passe par cette rue ne sait pas
  pourquoi il voit une cage en regardant la maison. Celui qui passe
  regarde la maison. Pour ne plus se tourmenter avec ça, il veut trouver
  une explication, et s arrête, regarde bien et ensuite il s'écarte et
  se dit que cela n'était qu'une illusion, que cela vient des grilles de
  la maison

--- Et l'impression part avec lui, elle accompagne des jours durant celui
  qui a vu la maison

Il n'arrive pas vraiment à s'en distraire, le voilà revenu devant cette
maison. Il cherche dans sa mémoire où il a bien pu voir cette maison, se
demande celui qui est passé.

L'enfance n'a pas de fin\\

J'attends le retour de l'oiseau. Et en attendant, j'entends les voix

Les voix de la terre viennent de loin. Pour les entendre il suffit de se
permettre de rester, ne jamais partir. Je reste. Le vent les apportera
toutes.\\

Il y a en d'autres, comme moi, partout? N'entendent-ils, comme moi, ces
voix?\\

ça, avait dit Jacinto, et cet homme était parti, parti, il était revenu,
parti encore une fois.\\

Ceci est encore un homme. Un insecte peut-être me regarde. Et ceci
insiste à comprendre

Je suis là. Aveugle.

Ici c'est partout\\

Dit Jacinto. Et Jacinto est l'homme à la fenêtre.

\clearpage
\thispagestyle{empty}
\movetooddpage

\vspace*{4cm}

Si tout continue ainsi indéfiniment

pendant des jours et des jours, un jour viendra où Jacinto ne sera plus
Jacinto.

Et il y aura un autre à sa fenêtre, qui attend encore

Bouh. Un fantôme.

Quelques os blancs

bouh, un son pour effrayer les enfants\\

Le matin, il apparait à sa fenêtre.

La fenêtre sera trop vieille.

Tous les jours, bouh apparaitra à sa fenêtre. Le bois de cette fênetre
n'aura plus d'âge, il aura des craquelures qui diront nous sommes des
choses mortes d'un fantôme\\

Et la vie, comment sera-t-elle si le Curau ne vient pas

\breakk
\pagecolor{black}

\chapter*{}
\pagecolor{black}\afterpage{\nopagecolor}


\movetoevenpage

\vspace*{4cm}

Bouh, dit le vent qui passe par Jacinto encore un homme.

Car bouh sera Jacinto, sauf si le Curau revient

\breakk

\vspace*{4cm}

Pendant plusieurs années Jacinto les entend encore de sa fenêtre. Les
voix.

Et les jours sont passés vite par lui. Et ils ne passaient pas

\clearpage
\thispagestyle{empty}
\movetooddpage

\vspace*{4cm}

L'homme était là, aveugle.

Quand l'autre est arrivé, il a dit:

--- Ah, tu es revenu. Appelle-moi Jacinto. Je suis là, aveugle. Ici c'est
  partout

Tu as ramené un garçon cette fois-ci. C'est ton fils, tu veux qu'il
écoute aussi l'histoire du Curau. Oui. Tu es revenu

Je la raconte, je la raconte encore une fois.

Oui. A Santa Maria do Grão ces choses arrivent.

Viens ici, garçon. Tu es une nouveauté pour moi. Le Curau est venu il y
a longtemps, tu n'étais pas encore né. Puis, il s'en est allé. Mais un
jour il reviendra. Toi aussi, tu dois savoir comment tout cela est
arrivé.

Le garçon écoutait.

L'aveugle parlait.

--- Il y a longtemps. Mais je me rappelle tout. Je n'oublierai jamais.

Le garçon écoutait.

Viens ici, dit l'aveugle.

Laisse-moi toucher tes yeux.

Il l'a laissé faire.

Les yeux. Ceux-là. Les tiens.

N'aie pas peur. Je touche. Je sais.

Tu es un enfant, les enfants ne doivent pas avoir peur du Curau. Que tu
ne sois un de plus à courir dans les rues, au moment où le Curau
reviendra. Cachant leurs yeux, cherchant un lieu pour se cacher, criant
le Curau, Curau. Il arrive. Comme ils criaient. Tellement. Les cris. Et
plus tard, ils ne fermaient plus les yeux la nuit par peur de n'avoir
plus d'yeux à ouvrir le matin.\\

Le garçon écoutait et le père a voulu regarder par la fenêtre.

L'aveugle racontait

Tout a commencé comme je te raconte maintenant. Je te le raconte encore
une fois. J'ai été averti avant les autres.

L'oiseau était tout rouge. Il était là, arrêté. Il semblait malade. Je
lui ai donné le nom que j'ai voulu, Curau. Le premier nom qui est sorti
de ma bouche. Il n'y a que comme cela, apprends, qu'on peut trouver le
nom caché d'une chose cachée. Et celui-ci n'était pas un oiseau comme
les autres\\

Le garçon écoutait.\\

La peur ne vient que chez ceux qui ont déjà de vieilles raisons d'avoir
peur, et ceux-là ont commencé donc à avoir peur du Curau. Ceux-là ont
peur de tout, disait Jacinto. Et le garçon écoutait. Quand l'oiseau est
arrivé, cette peur-là marchait dans les rues avec des pas qui ne nous
conduiront qu'à un terre non-sacrée

Regarde, sauf les enfants qui ont peur de choses réelles, disait Jacinto\\

Comprends, quand un garçon crie le Curau, le Curau, il ne fait qu'imiter
les autres. Si les adultes n'ont pas peur, les enfants n'auront pas peur
de l'oiseau. Ils vont rester dans leurs hamacs, calmes, et les soirs ne
se seront pas aux aguets.\\

Le Curau n'a percé que les yeux des adultes, quand il venu pour la
première fois, disait Jacinto\\

Tout le monde doit demander la venue du Curau. Son retour. On doit lui
demander tous les soirs, et dire, comme dans une prière, viens Curau et
aveugle-moi. Délivre-moi de ces yeux ne voulant plus voir les choses
ainsi, de la même manière

Il n'y a que toi qui n'ait pas besoin de demander, mon garçon, disait
Jacinto.

Demander que le Curau vienne t'aveugler.

Mais demande-le pour ton père.

Demande le soir avant de dormir.

Le Curau a été un bien qui nous est arrivé, mon garçon, et un jour il va
revenir, disait Jacinto\\

Repoussé par l'aveugle, ensuite.

Le garçon s'est écarté.

Maintenant, va t'en. J'ai déjà tout dit. Maintenant je veux rester seul
ici, assis à côté de la fenêtre. Aveugle. A dit Jacinto.\\

Dans la rue l'homme disait au garçon

--- Non

Le garçon a voulu regarder encore une fois vers le ciel.

Dans le ciel il y avait des nuages, il y avait une tâche. Elle était
grande. Rouge.\\

Peut-être maintenant un insecte me regarde pour comprendre, dit Jacinto
à sa fenêtre.

A sa fenêtre, il est en train de dire, viens Curau

\breakk

\vspace*{4cm}

Et cette voix qui dit dans le vent\\

Nous avons entendu, encore une fois

--- Viens Curau. Viens emporter les hommes vers tes jardins

\vfill
Fin de «Les jardins et la nuit»\\

Le voyage à Andara n'a pas de fin.