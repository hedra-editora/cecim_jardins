\chapter*{Introdução\\
\emph{Romance onírico ambientado na região de~Andara,~na Amazônia brasileira}}

\addcontentsline{toc}{chapter}{Introdução}


Em uma Amazônia de sonhos, um cego escuta as histórias que o vento traz
de longe, de outros lugares, de outras pessoas, para lhe contar. São
histórias fantásticas, breves ou longas -- e o livro tem muitas delas --
por isso, \emph{Os jardins e a noite} vai além da convenção formal de um
romance linear, e em sua livre estrutura, fragmentada, pode ser definido
como um \emph{récit --} com a história central do livro, a história do
cego, sendo constantemente entrecortada pelas outras histórias que vêm a
ele no vento. \emph{Como esse homem ficou cego?} Ele foi cegado pelo
pássaro Curau, que achou na porta de sua casa, um dia. Mas para ele, ter
sido cegado pelo pássaro não foi um mal, e sim um grande bem. E desde
então o cego agradece ao pássaro, que depois de tê-lo cegado desapareceu
nos céus -- porque, graças a ele, na escuridão de sua cegueira, ele pode
ver \emph{mais} -- e ver coisas que, antes, quando tinha olhos, não via.
Em sua janela, de onde jamais sai, o cego ouve as histórias do vento e
espera a volta do Curau, e faz orações para que ele um dia volte -- e
pede ao pássaro que cegue também todos os habitantes do lugar (a região
de Andara\textbf{*}, em que o autor transfigura oniricamente a Amazônia
onde nasceu e vive) para que eles \emph{passem a ver mais, como agora
ele vê}. Um trecho do livro é bem esclarecedor, exatamente suas últimas
palavras: \emph{- Vem Curau. Vem levar os homens para os teus jardins
--} é a oração do cego, solitário, sempre em sua janela, sempre ouvindo
as vozes do vento - e o livro \emph{Os jardins e a noite} se encerra com
a sua oração.\\


\textbf{*} Andara -- onde o autor ambienta toda sua obra literária - é a
Amazônia vista com olhos mágicos, por ele recriada. Segundo o autor,
toda a sua obra é uma \emph{Viagem a Andara} -- sendo este o título
geral de sua obra literária, viagem essa que ele resume com estas
palavras:

\emph{``A Viagem que se faz, que se inicia a partir do Mistério que é
nascer -- então, desde antes de se escrever a primeira Palavra - é
Demanda do Um através do Vários. A Literatura é essa Demanda em espelho.
}

\emph{Enquanto a Vida viaja em mim, escrevo para me ver nela vivendo.}

\emph{Andara é Coisa que viaja por dentro e no sentido inverso: quer
retornar dos dedos dos pés ao calcanhar de Aquiles do homem, ali onde
ele é mais sensível à Hipótese Onírica e Lúdica e Naturalmente Sagrada
da vida.''}

\chapter*{Introduction\\
\emph{Roman onirique se passant dans la région d'Andara, dans l'Amazonie brésilienne}}

\addcontentsline{toc}{chapter}{Introduction}

Dans une Amazonie des rêves, un aveugle entend les histoires que le vent
apporte d'endroits lointains, de personnes proches ou non, qu'il
racontera à qui veut bien les entendre. Ce sont plusieurs histoires
fantastiques, courtes ou longues, présentées selon une structure libre,
fragmentée -- \emph{Les jardins et la nuit} serait donc mieux défini
comme un \emph{récit}, fort éloigné de la convention formelle d'un roman
linéaire. L'histoire centrale, l'histoire de l'aveugle, est souvent
entrecoupée par des mémoires apportées par le vent. \emph{Comment est-il
devenu aveugle}? Il a été aveuglé par l'oiseau Curau, trouvé un jour
devant sa porte. Et pourtant, pour lui, être devenu aveugle par l'oiseau
n'a pas été un mal, mais cela a été tout à fait un bien. Depuis lors, il
remercie l'oiseau, disparu dans le ciel après l'avoir aveuglé, parce
que, grâce à lui, dans les ténèbres de son aveuglement, il a pu voir
\emph{davantage} et voir des choses qu'il ne voyait pas quand il avait
des yeux. A sa fenêtre, d'où il ne sort jamais, l'aveugle entend les
histoires du vent et attend le retour du Curau, en priant pour qu'il
revienne un jour. Il demande dans sa prière que l'oiseau aveugle aussi
tous les habitants de la région (Andara\textbf{*}, dans laquelle l'auteur
transfigure oniriquement l'Amazonie où il est né et où il habite), pour
qu'il \emph{voient davantage, comme lui voit maintenant}. Un passage du
livre, où se trouvent justement ses derniers mots, en est bien
illustratif : - \emph{Viens Curau. Viens emmener les hommes à tes
jardins} -- cela est la prière de l'aveugle, toujours à sa fenêtre,
tandis qu'il entend les voix du vent. Et le livre \emph{Les jardins et
la nuit} se termine par sa prière.\\

\noindent{}\textbf{*} Andara -- où l'auteur fait passer toute cette histoire, c'est
l'Amazonie vue avec des yeux magiques, par lui récréée. Selon l'auteur,
toute sa production est un \emph{Voyage à Andara}, étant en outre le
titre général de son oeuvre, qu'il résume par ces mots~:

\noindent{}\emph{«~Le voyage que l'on fait, déjà initié dans le Mystère qui est la
naissance -- alors, avant même l'écriture du premier Mot -- est la
Demande de l'Un à travers le Plusieurs. La littérature est cette demande
en miroir.}

\noindent{}\emph{Pendant que la vie voyage en moi, j'écris pour m'y voir vivant.}

\noindent{}\emph{Andara est Chose qui voyage dedans et dans le sens inverse ~: il
veut retourner des orteils au tendon d'Achille de l'homme, là où il est
plus sensible à l'Hypothèse Onirique et Ludique et Naturellement Sacrée
de la vie.~»}
