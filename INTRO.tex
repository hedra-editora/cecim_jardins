\selectlanguage{french}
\chapter*{Note d'introduction\\
\emph{Cecim: Andara. Visible. Invisible.}}

\addcontentsline{toc}{chapter}{Note d'introduction, \emph{par Cláudia Vitalouca}}

\begin{flushright}
\emph{Cláudia Vitalouca}
\end{flushright}

\bigskip
\bigskip

\noindent{}``Les jardins et la nuit'' c'est la poétique de la création de
l'écrivain brésilien Vicente Franz Cecim.

\noindent{}C'est comment se construit l'écriture littéraire de l'auteur par le
biais de l'encre (in)visible, dans sa relation avec la littérature brésilienne contemporaine.

\noindent{}C'est l'univers créatif de Cecim.

\noindent{}C'est la composition littéraire du cycle de Andara, qui a comme vecteurs,\\
\noindent{}entre autres, des réflexions au sujet de l'Être et des perceptions de mondes\\
\noindent{}(in)visibles, l'image du labyrinthe comme espace imaginaire et du métalangage de l'écriture de Vicente Franz Cecim.

\noindent{}l'inachevé de l'écriture littéraire et le silence comme espace ouvert à l'imagination créative.

\noindent{}C'est la poétique d'une écriture qui est centrée entre le visible et l'invisible.

\noindent{}``Les jardins et la nuit'' c'est Andara. Visible. Invisible.
%Dans une Amazonie des rêves, un aveugle entend les histoires que le vent
%apporte d'endroits lointains, de personnes proches ou non, qu'il
%racontera à qui veut bien les entendre. Ce sont plusieurs histoires
%fantastiques, courtes ou longues, présentées selon une structure libre,
%fragmentée -- \emph{Les jardins et la nuit} serait donc mieux défini
%comme un \emph{récit}, fort éloigné de la convention formelle d'un roman
%linéaire. L'histoire centrale, l'histoire de l'aveugle, est souvent
%entrecoupée par des mémoires apportées par le vent. \emph{Comment est-il
%devenu aveugle}? Il a été aveuglé par l'oiseau Curau, trouvé un jour
%devant sa porte. Et pourtant, pour lui, être devenu aveugle par l'oiseau
%n'a pas été un mal, mais cela a été tout à fait un bien. Depuis lors, il
%remercie l'oiseau, disparu dans le ciel après l'avoir aveuglé, parce
%que, grâce à lui, dans les ténèbres de son aveuglement, il a pu voir
%\emph{davantage} et voir des choses qu'il ne voyait pas quand il avait
%des yeux. A sa fenêtre, d'où il ne sort jamais, l'aveugle entend les
%histoires du vent et attend le retour du Curau, en priant pour qu'il
%revienne un jour. Il demande dans sa prière que l'oiseau aveugle aussi
%tous les habitants de la région (Andara\textbf{*}, dans laquelle l'auteur
%transfigure oniriquement l'Amazonie où il est né et où il habite), pour
%qu'il \emph{voient davantage, comme lui voit maintenant}. Un passage du
%livre, où se trouvent justement ses derniers mots, en est bien
%illustratif : - \emph{Viens Curau. Viens emmener les hommes à tes
%jardins} -- cela est la prière de l'aveugle, toujours à sa fenêtre,
%tandis qu'il entend les voix du vent. Et le livre \emph{Les jardins et
%la nuit} se termine par sa prière.\\

%\noindent{}\textbf{*} Andara -- où l'auteur fait passer toute cette histoire, c'est
%l'Amazonie vue avec des yeux magiques, par lui récréée. Selon l'auteur,
%toute sa production est un \emph{Voyage à Andara}, étant en outre le
%titre général de son oeuvre, qu'il résume par ces mots~:

%\noindent{}\emph{«~Le voyage que l'on fait, déjà initié dans le Mystère qui est la
%naissance -- alors, avant même l'écriture du premier Mot -- est la
%Demande de l'Un à travers le Plusieurs. La littérature est cette demande
%en miroir.}

%\noindent{}\emph{Pendant que la vie voyage en moi, j'écris pour m'y voir vivant.}

%\noindent{}\emph{Andara est Chose qui voyage dedans et dans le sens inverse ~: il
%veut retourner des orteils au tendon d'Achille de l'homme, là où il est
%plus sensible à l'Hypothèse Onirique et Ludique et Naturellement Sacrée
%de la vie.~»}
